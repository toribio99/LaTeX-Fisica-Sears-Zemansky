\documentclass[11pt,a4paper]{article}

% Paquetes necesarios
\usepackage[utf8]{inputenc}
\usepackage[T1]{fontenc}
\usepackage[spanish]{babel}
\usepackage[margin=2.5cm]{geometry}
\usepackage{amsmath}
\usepackage{amssymb}
\usepackage{xcolor}
\usepackage{tcolorbox}
\usepackage{graphicx}
\usepackage{tikz}
\usepackage{pgfplots}
\pgfplotsset{compat=1.18}
\usetikzlibrary{arrows.meta,patterns,decorations.pathmorphing,calc}
\usepackage{wrapfig}
\usepackage[export]{adjustbox} % (opcional) claves extra para \includegraphics
\usepackage{xparse}

% Definición de colores
\definecolor{azuloscuro}{RGB}{0,51,102}
\definecolor{azulclaro}{RGB}{230,240,250}
\definecolor{verdeoscuro}{RGB}{0,100,0}
\definecolor{rojoclaro}{RGB}{255,230,230}

% Configuración de cajas
\tcbuselibrary{theorems,skins,breakable}

\newtcolorbox{datosbox}{
    colback=azulclaro,
    colframe=azuloscuro,
    fonttitle=\bfseries,
    title=Datos del Problema,
    sharp corners,
    boxrule=1pt
}

\newtcolorbox{solucionbox}{
    colback=white,
    colframe=verdeoscuro,
    fonttitle=\bfseries,
    title=Desarrollo de la Solución,
    sharp corners,
    boxrule=1pt,
    breakable
}

\newtcolorbox{resultadobox}{
    colback=rojoclaro,
    colframe=red!70!black,
    fonttitle=\bfseries,
    title=Resultado Final,
    sharp corners,
    boxrule=2pt
}

% Título y autor
\title{\textbf{Solución del Ejercicio 3.31} \\
\large Movimiento Circular: Prueba de Traje G}
\author{Física Universitaria - Sears y Zemansky \\ Capítulo 3: Movimiento en Dos o Tres Dimensiones}
\date{\today}

\begin{document}

\maketitle

\section{Enunciado del Problema}

\begin{wrapfigure}[8]{r}{.55\textwidth}  % Ajusta el ancho según necesites
	\centering
	\vspace{-\baselineskip}
	%\begin{center}
\begin{tikzpicture}[scale=0.65]
    % Vista superior del círculo de rotación
    \draw[blue!40!black, thick] (0,0) circle (4);
    \draw[blue!40!black, dashed] (0,0) circle (4);

    % Centro de rotación
    \filldraw[black] (0,0) circle (0.1);
    \node[below] at (0,-0.3) {Centro};

    % Voluntario en la circunferencia
    \filldraw[red!70!black] (4,0) circle (0.25);
    \node[red!70!black, right] at (4.3,0) {Voluntario};

    % Radio
    \draw[-{Latex[length=2mm]}, thick, orange!80!black] (0,0) -- (4,0);
    \node[orange!80!black, above] at (2,0.1) {$R = 7.0$ m};

    % Vector velocidad
    \draw[-{Latex[length=3mm]}, green!60!black, ultra thick] (4,0) -- (4,1.5);
    \node[green!60!black, right] at (4,1) {$\vec{v}$};

    % Vector aceleración centrípeta
    \draw[-{Latex[length=3mm]}, purple!70!black, ultra thick] (4,0) -- (2.5,0);
    \node[purple!70!black, above] at (3.2,-0.9) {$\vec{a}_{\text{rad}}$};

    % Flecha de rotación
    \draw[-{Latex[length=4mm]}, red!70!black, very thick]
        (2.8,2.8) arc (45:100:4);
    \node[red!70!black] at (0.5,3.5) {Rotación};

    % Brazo mecánico (opcional)
    \draw[thick, gray!60] (0,0) -- (4,0);
    \foreach \x in {0.5,1,...,3.5} {
        \draw[gray!60] (\x,-0.1) -- (\x,0.1);
    }

    % Etiqueta
    \node[below, align=center]  at (0,-5.2) {\textbf{Vista superior: voluntario} \\ (girando en círculo horizontal)};

\end{tikzpicture}
%\end{center}
\vspace{-\baselineskip} % Reduce espacio después
\end{wrapfigure}
En una prueba de un "traje g", un voluntario se gira en un círculo horizontal de 7.0 m de radio. \\[.4mm]

?`Con qué periodo de rotación la aceleración centrípeta tiene magnitud de: \\[.4mm]

a) 3.0$g$? \\[.4mm]

b) 10$g$? \\[.4mm]

\vspace{0.5cm}

\section{Datos del Problema}

\begin{datosbox}
\begin{itemize}
    \item \textbf{Radio de la trayectoria circular:} $R = 7.0$ m
    \item \textbf{Aceleración gravitacional:} $g = 9.8$ m/s$^2$
    \item \textbf{Aceleraciones a probar:}
    \begin{itemize}
        \item Parte a): $a_{\text{rad}} = 3.0g = 3.0 \times 9.8 = 29.4$ m/s$^2$
        \item Parte b): $a_{\text{rad}} = 10g = 10 \times 9.8 = 98$ m/s$^2$
    \end{itemize}
    \item \textbf{Incógnita:} Periodo $T$ en cada caso
\end{itemize}
\end{datosbox}

\section{Marco Teórico}

\subsection{Aceleración Centrípeta y Periodo}

La aceleración centrípeta en movimiento circular uniforme está relacionada con el periodo por:

\begin{equation*}
    a_{\text{rad}} = \frac{4\pi^2 R}{T^2}
\end{equation*}

\subsection{Despeje del Periodo}

Despejando $T$ de la ecuación anterior:

\begin{align*}
    T^2 &= \frac{4\pi^2 R}{a_{\text{rad}}} \\
    T &= \sqrt{\frac{4\pi^2 R}{a_{\text{rad}}}} \\
    T &= 2\pi\sqrt{\frac{R}{a_{\text{rad}}}}
\end{align*}

Esta es la fórmula que usaremos para encontrar el periodo.

\subsection{Contexto: Trajes G}

Los pilotos de aviones de combate y astronautas usan "trajes g" que aplican presión a las piernas y abdomen para evitar que la sangre se acumule en las extremidades inferiores durante maniobras de alta aceleración. Estos entrenamientos simulan las condiciones que experimentarán en vuelo.

\section{Desarrollo de la Solución}

\begin{solucionbox}

\subsection*{Parte a) Periodo para $a_{\text{rad}} = 3.0g$}

\subsubsection*{Paso 1: Calcular la aceleración en m/s$^2$}

\begin{equation*}
    a_{\text{rad}} = 3.0g = 3.0 \times 9.8 = 29.4 \text{ m/s}^2
\end{equation*}

\subsubsection*{Paso 2: Aplicar la fórmula del periodo}

\begin{equation*}
    T = 2\pi\sqrt{\frac{R}{a_{\text{rad}}}} = 2\pi\sqrt{\frac{7.0 \text{ m}}{29.4 \text{ m/s}^2}}
\end{equation*}

\subsubsection*{Paso 3: Calcular la fracción dentro de la raíz}

\begin{equation*}
    \frac{R}{a_{\text{rad}}} = \frac{7.0}{29.4} = 0.238 \text{ s}^2
\end{equation*}

\subsubsection*{Paso 4: Calcular la raíz cuadrada}

\begin{equation*}
    \sqrt{0.238} = 0.488 \text{ s}
\end{equation*}

\subsubsection*{Paso 5: Multiplicar por $2\pi$}

\begin{equation*}
    T = 2\pi \times 0.488 = 3.07 \text{ s}
\end{equation*}

\textbf{Resultado:} $T \approx 3.1$ s

\textbf{Interpretación:} Para experimentar una aceleración de 3$g$, el voluntario debe completar una vuelta completa cada 3.1 segundos.

\subsection*{Parte b) Periodo para $a_{\text{rad}} = 10g$}

\subsubsection*{Paso 1: Calcular la aceleración en m/s$^2$}

\begin{equation*}
    a_{\text{rad}} = 10g = 10 \times 9.8 = 98 \text{ m/s}^2
\end{equation*}

\subsubsection*{Paso 2: Aplicar la fórmula del periodo}

\begin{equation*}
    T = 2\pi\sqrt{\frac{R}{a_{\text{rad}}}} = 2\pi\sqrt{\frac{7.0 \text{ m}}{98 \text{ m/s}^2}}
\end{equation*}

\subsubsection*{Paso 3: Calcular la fracción dentro de la raíz}

\begin{equation*}
    \frac{R}{a_{\text{rad}}} = \frac{7.0}{98} = 0.0714 \text{ s}^2
\end{equation*}

\subsubsection*{Paso 4: Calcular la raíz cuadrada}

\begin{equation*}
    \sqrt{0.0714} = 0.267 \text{ s}
\end{equation*}

\subsubsection*{Paso 5: Multiplicar por $2\pi$}

\begin{equation*}
    T = 2\pi \times 0.267 = 1.68 \text{ s}
\end{equation*}

\textbf{Resultado:} $T \approx 1.7$ s

\textbf{Interpretación:} Para experimentar una aceleración de 10$g$, el voluntario debe completar una vuelta completa cada 1.7 segundos. ¡Esto es casi el doble de rápido que en el caso anterior!

\subsection*{Verificación}

Verifiquemos los resultados calculando $a_{\text{rad}}$ con los periodos obtenidos:

\textbf{Para $T = 3.1$ s:}
\begin{align*}
    a_{\text{rad}} &= \frac{4\pi^2 \times 7.0}{(3.1)^2} = \frac{276.5}{9.61} = 28.8 \text{ m/s}^2 \approx 2.94g \quad \checkmark
\end{align*}

\textbf{Para $T = 1.7$ s:}
\begin{align*}
    a_{\text{rad}} &= \frac{4\pi^2 \times 7.0}{(1.7)^2} = \frac{276.5}{2.89} = 95.7 \text{ m/s}^2 \approx 9.8g \quad \checkmark
\end{align*}

Las pequeñas diferencias se deben al redondeo.

\end{solucionbox}

\section{Resultados Finales}

\begin{resultadobox}

\subsection*{Parte a) Para $a_{\text{rad}} = 3.0g$}

\begin{equation*}
    \boxed{T = 3.1 \text{ s}}
\end{equation*}

El voluntario debe completar una revolución cada \textbf{3.1 segundos}.

\subsection*{Parte b) Para $a_{\text{rad}} = 10g$}

\begin{equation*}
    \boxed{T = 1.7 \text{ s}}
\end{equation*}

El voluntario debe completar una revolución cada \textbf{1.7 segundos}.

\vspace{0.3cm}

\textbf{Observación:} Para triplicar la aceleración (de 3$g$ a 10$g$), el periodo debe reducirse por un factor de $\sqrt{10/3} \approx 1.83$, no por 3. Esto se debe a la relación cuadrática entre aceleración y periodo.

\end{resultadobox}

\section{Análisis Adicional}

\subsection{Relación entre aceleración y periodo}

\begin{center}
\begin{tikzpicture}
\begin{axis}[
    width=0.9\textwidth,
    height=7cm,
    xlabel={Periodo $T$ (s)},
    ylabel={Aceleración / $g$},
    xmin=0, xmax=5,
    ymin=0, ymax=12,
    grid=major,
    title={Aceleración centrípeta vs periodo para $R = 7.0$ m}
]

% Curva arad/g vs T
\addplot[blue, thick, domain=0.8:5, samples=100] {4*pi^2*7.0/(x^2*9.8)};

% Puntos calculados
\addplot[red, only marks, mark=*, mark size=3pt] coordinates {(3.1, 3.0)};
\addplot[orange, only marks, mark=*, mark size=3pt] coordinates {(1.7, 10)};

% Líneas de referencia
\draw[dashed, gray] (axis cs:0,3) -- (axis cs:3.1,3);
\draw[dashed, gray] (axis cs:3.1,0) -- (axis cs:3.1,3);
\draw[dashed, gray] (axis cs:0,10) -- (axis cs:1.7,10);
\draw[dashed, gray] (axis cs:1.7,0) -- (axis cs:1.7,10);

\node[red] at (axis cs:3.5,3.5) {3.0$g$, 3.1 s};
\node[orange] at (axis cs:2.2,10.5) {10$g$, 1.7 s};

\end{axis}
\end{tikzpicture}
\end{center}

\subsection{Velocidades lineales correspondientes}

Para completar, calculemos las velocidades lineales en cada caso:

\textbf{Para 3.0$g$ ($T = 3.1$ s):}
\begin{equation*}
    v = \frac{2\pi R}{T} = \frac{2\pi \times 7.0}{3.1} = 14.2 \text{ m/s} = 51 \text{ km/h}
\end{equation*}

\textbf{Para 10$g$ ($T = 1.7$ s):}
\begin{equation*}
    v = \frac{2\pi R}{T} = \frac{2\pi \times 7.0}{1.7} = 25.9 \text{ m/s} = 93 \text{ km/h}
\end{equation*}

\subsection{Comparación de condiciones}

\begin{center}
\begin{tabular}{|c|c|c|c|c|}
\hline
\textbf{Aceleración} & \textbf{Periodo} & \textbf{Frecuencia} & \textbf{Velocidad} & \textbf{Aplicación} \\
\hline
3.0$g$ & 3.1 s & 0.32 Hz & 14.2 m/s & Entrenamiento básico \\
\hline
10$g$ & 1.7 s & 0.59 Hz & 25.9 m/s & Simulación de combate \\
\hline
\end{tabular}
\end{center}

\subsection{Efectos fisiológicos}

\begin{itemize}
    \item \textbf{3$g$:} La persona siente que pesa 3 veces más. Incomodidad pero manejable.

    \item \textbf{5-6$g$:} Dificultad para mover brazos y piernas. Sin traje g, posible pérdida de visión periférica.

    \item \textbf{9-10$g$:} Sin traje g, pérdida de conciencia en segundos. El traje g es esencial.

    \item \textbf{$>$ 10$g$:} Riesgo de daño físico incluso con traje g.
\end{itemize}

\section{Fórmula General}

\begin{tcolorbox}[colback=yellow!10!white,colframe=orange!75!black,title=Relación General]

Para un círculo de radio $R$, si queremos una aceleración de $n \times g$:

\begin{equation*}
    T = 2\pi\sqrt{\frac{R}{ng}}
\end{equation*}

O simplificando:

\begin{equation*}
    T = \frac{2\pi}{\sqrt{g}}\sqrt{\frac{R}{n}}
\end{equation*}

Para $R$ en metros:
\begin{equation*}
    T \approx 2.01\sqrt{\frac{R}{n}} \text{ segundos}
\end{equation*}

\end{tcolorbox}

\section{Conceptos Clave}

\begin{enumerate}
    \item El periodo necesario para lograr cierta aceleración depende inversamente de la raíz cuadrada de la aceleración: $T \propto 1/\sqrt{a}$

    \item Para aumentar la aceleración al doble, el periodo debe reducirse por un factor de $\sqrt{2} \approx 1.41$

    \item A mayor aceleración, menor periodo (rotación más rápida)

    \item Los entrenamientos con traje g preparan a pilotos y astronautas para condiciones extremas

    \item La relación $T = 2\pi\sqrt{R/a}$ es fundamental en movimiento circular
\end{enumerate}

\end{document}
