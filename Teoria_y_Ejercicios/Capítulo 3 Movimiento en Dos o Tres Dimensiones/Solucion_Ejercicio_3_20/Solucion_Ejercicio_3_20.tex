\documentclass[11pt,a4paper]{article}

% Paquetes necesarios
\usepackage[utf8]{inputenc}
\usepackage[T1]{fontenc}
\usepackage[spanish]{babel}
\usepackage[margin=2.5cm]{geometry}
\usepackage{amsmath}
\usepackage{amssymb}
\usepackage{xcolor}
\usepackage{tcolorbox}
\usepackage{graphicx}
\usepackage{tikz}
\usepackage{pgfplots}
\pgfplotsset{compat=1.18}
\usetikzlibrary{arrows.meta,patterns,decorations.pathmorphing,calc}
\usepackage{wrapfig}
\usepackage[export]{adjustbox} % (opcional) claves extra para \includegraphics
\usepackage{xparse}

% Definición de colores
\definecolor{azuloscuro}{RGB}{0,51,102}
\definecolor{azulclaro}{RGB}{230,240,250}
\definecolor{verdeoscuro}{RGB}{0,100,0}
\definecolor{rojoclaro}{RGB}{255,230,230}

% Configuración de cajas
\tcbuselibrary{theorems,skins,breakable}

\newtcolorbox{datosbox}{
    colback=azulclaro,
    colframe=azuloscuro,
    fonttitle=\bfseries,
    title=Datos del Problema,
    sharp corners,
    boxrule=1pt
}

\newtcolorbox{solucionbox}{
    colback=white,
    colframe=verdeoscuro,
    fonttitle=\bfseries,
    title=Desarrollo de la Solución,
    sharp corners,
    boxrule=1pt,
    breakable
}

\newtcolorbox{resultadobox}{
    colback=rojoclaro,
    colframe=red!70!black,
    fonttitle=\bfseries,
    title=Resultado Final,
    sharp corners,
    boxrule=2pt
}

% Título y autor
\title{\textbf{Solución del Ejercicio 3.20} \\
\large Lanzamiento de Bala}
\author{Física Universitaria - Sears y Zemansky \\ Capítulo 3: Movimiento en Dos o Tres Dimensiones}
\date{\today}

\begin{document}

\maketitle

\section{Enunciado del Problema}

\begin{wrapfigure}[10]{r}{.6\textwidth}  % Ajusta el ancho según necesites
	\centering
	\vspace{-\baselineskip}
	%\begin{center}
\begin{tikzpicture}[scale=.85]
    % Suelo (ajustado a la nueva escala)
    \draw[fill=brown!20, thick] (-1,0) rectangle (8.5,-0.3);
    \draw[pattern=north east lines, pattern color=brown] (-1,-0.3) rectangle (8.5,-0.5);

    % Plataforma de lanzamiento
    \draw[fill=gray!30, thick] (-0.5,0) rectangle (0.5,2);
    \node[right] at (0.6,1) {\small Plataforma};
    \node[right] at (0.6,0.5) {\small $h_0 = 2$ m};

    % Punto de lanzamiento
    \filldraw[blue!70!black] (0,2) circle (0.15);
    \node[below] at (-0.6,2.6) {\small Lanzador};

    % Vector velocidad inicial
    \draw[-{Latex[length=3mm]},blue!70!black, very thick] (0,2) -- (1.2,3.48)
        node[midway,above left] {$\vec{v}_0$};

    % Componentes
    \draw[-{Latex[length=2mm]},red, thick, dashed] (0,2) -- (1.2,2)
        node[midway,below] {$v_{0x}$};
    \draw[-{Latex[length=2mm]},red, thick, dashed] (1.2,2) -- (1.2,3.48)
        node[midway,right] {$v_{0y}$};

    % Ángulo
    \draw[thick] (0.6,2) arc (0:51:0.6);
    \node at (0.9,2.3) {$51°$};

    % Trayectoria parabólica desde altura inicial
    % Ecuación correcta: y = h0 + tan(θ)*x - (g/(2*v0^2*cos^2(θ)))*x^2
    % Coeficiente correcto: b = 9.8/(2*12^2*cos^2(51°)) = 9.8/(2*144*0.396) = 0.0859
    % y = 2 + 1.2349*x - 0.0859*x^2 (en metros)
    % Alcance: R = 15.9 m, Altura máxima: 6.44 m en x = 7.18 m
    % Escala: 1 unidad gráfica = 2 m → x_graf = x_m/2
    % y_graf = 2 + 1.2349*(2*x_graf) - 0.0859*(2*x_graf)^2
    % y_graf = 2 + 2.4698*x_graf - 0.3436*x_graf^2
    \draw[red!70!black, very thick, -{Latex[length=3mm]}]
        plot[domain=0:7.95, samples=100, smooth] (\x, {2 + 2.4698*\x - 0.3436*\x*\x});

    % Punto de altura máxima (x_m = 7.18 m → x_graf = 3.59, y = 6.44)
    \filldraw[green!50!black] (3.59,6.44) circle (0.12);
    \node[green!50!black, above, font=\small] at (3.59,6.44) {$h_{\text{máx}}=6.44$ m};

    % Línea de altura máxima
    \draw[green!50!black, dotted, thick] (3.59,6.44) -- (3.59,0);
    \draw[{Latex[length=2mm]}-{Latex[length=2mm]}, thick] (-1.2,0) -- (-1.2,6.44)
        node[below,right,font=\small] {$h_{\text{máx}}$};

    % Alcance horizontal
    \draw[{Latex[length=2mm]}-{Latex[length=2mm]}, thick] (0,-1.0) -- (7.95,-1.0)
        node[midway,below] {$R = 15.9$ m};

    % Punto de impacto
    \filldraw[red!70!black] (7.95,0) circle (0.12);
    \node[below] at (7.95,-0.5) {\small Impacto};

\end{tikzpicture}
%\end{center}
\vspace{-\baselineskip} % Reduce espacio después
\end{wrapfigure}
Se lanza una bala con velocidad inicial de 12.0 m/s, a un ángulo de 51.0° sobre la horizontal desde un punto a 2.00 m de altura. Calcule:

\begin{enumerate}
	\item[a)] Las componentes $x$ e $y$ de la velocidad inicial.
	\item[b)] El tiempo que tarda la bala en alcanzar la altura máxima.
	\item[c)] La altura máxima sobre el suelo.
	\item[d)] El tiempo total de vuelo.
	\item[e)] El alcance horizontal total.
\end{enumerate}

\begin{enumerate}
	\item[f)] Las componentes de velocidad de la bala justo antes de golpear el suelo.
\end{enumerate}


\section{Datos del Problema}

\begin{datosbox}
\begin{itemize}
    \item \textbf{Velocidad inicial:} $v_0 = 12.0$ m/s
    \item \textbf{Ángulo de lanzamiento:} $\theta_0 = 51.0°$
    \item \textbf{Altura inicial:} $h_0 = 2.00$ m
    \item \textbf{Aceleración gravitacional:} $g = 9.8$ m/s$^2$
    \item \textbf{Altura final:} $y_f = 0$ m (nivel del suelo)
    \item \textbf{Resistencia del aire:} despreciable
\end{itemize}
\end{datosbox}

\section{Marco Teórico}

Para un proyectil lanzado desde una altura $h_0$ con velocidad inicial $v_0$ a un ángulo $\theta_0$:

\textbf{Componentes de velocidad inicial:}
\begin{align*}
    v_{0x} &= v_0 \cos\theta_0 \\
    v_{0y} &= v_0 \sin\theta_0
\end{align*}

\textbf{Ecuaciones de posición:}
\begin{align*}
    x(t) &= v_{0x}t \\
    y(t) &= h_0 + v_{0y}t - \frac{1}{2}gt^2
\end{align*}

\textbf{Ecuaciones de velocidad:}
\begin{align*}
    v_x(t) &= v_{0x} \quad \text{(constante)} \\
    v_y(t) &= v_{0y} - gt
\end{align*}

\textbf{Tiempo hasta altura máxima:}
\begin{equation}
    t_{\text{máx}} = \frac{v_{0y}}{g}
\end{equation}

\textbf{Altura máxima:}
\begin{equation}
    h_{\text{máx}} = h_0 + \frac{v_{0y}^2}{2g}
\end{equation}

\textbf{Tiempo total de vuelo (hasta llegar al suelo):}

Se obtiene resolviendo $y(t) = 0$:
\begin{equation}
    0 = h_0 + v_{0y}t - \frac{1}{2}gt^2
\end{equation}

\section{Desarrollo de la Solución}

\begin{solucionbox}

\subsection*{Parte a) Componentes de velocidad inicial}

\textbf{Componente horizontal:}
\begin{align*}
    v_{0x} &= v_0 \cos\theta_0 \\
    v_{0x} &= 12.0 \times \cos(51.0°) \\
    v_{0x} &= 12.0 \times 0.6293 \\
    v_{0x} &= 7.55 \text{ m/s}
\end{align*}

\textbf{Componente vertical:}
\begin{align*}
    v_{0y} &= v_0 \sin\theta_0 \\
    v_{0y} &= 12.0 \times \sin(51.0°) \\
    v_{0y} &= 12.0 \times 0.7771 \\
    v_{0y} &= 9.33 \text{ m/s}
\end{align*}

\subsection*{Parte b) Tiempo hasta altura máxima}

En la altura máxima, $v_y = 0$:
\begin{align*}
    0 &= v_{0y} - gt_{\text{máx}} \\
    t_{\text{máx}} &= \frac{v_{0y}}{g} \\
    t_{\text{máx}} &= \frac{9.33}{9.8} \\
    t_{\text{máx}} &= 0.952 \text{ s}
\end{align*}

\subsection*{Parte c) Altura máxima sobre el suelo}

Usando la ecuación de posición en $t = t_{\text{máx}}$:
\begin{align*}
    h_{\text{máx}} &= h_0 + v_{0y}t_{\text{máx}} - \frac{1}{2}g t_{\text{máx}}^2 \\
    h_{\text{máx}} &= 2.00 + 9.33 \times 0.952 - \frac{1}{2} \times 9.8 \times (0.952)^2 \\
    h_{\text{máx}} &= 2.00 + 8.88 - 4.44 \\
    h_{\text{máx}} &= 6.44 \text{ m}
\end{align*}

\textbf{Forma alternativa:}
\begin{align*}
    h_{\text{máx}} &= h_0 + \frac{v_{0y}^2}{2g} \\
    h_{\text{máx}} &= 2.00 + \frac{(9.33)^2}{2 \times 9.8} \\
    h_{\text{máx}} &= 2.00 + \frac{87.05}{19.6} \\
    h_{\text{máx}} &= 2.00 + 4.44 \\
    h_{\text{máx}} &= 6.44 \text{ m} \quad \checkmark
\end{align*}

\subsection*{Parte d) Tiempo total de vuelo}

La bala regresa al suelo cuando $y = 0$. Usamos:
\begin{equation}
    0 = h_0 + v_{0y}t - \frac{1}{2}gt^2
\end{equation}

Reordenando:
\begin{equation}
    \frac{1}{2}gt^2 - v_{0y}t - h_0 = 0
\end{equation}

Sustituyendo valores:
\begin{equation}
    4.9t^2 - 9.33t - 2.00 = 0
\end{equation}

Usando la fórmula cuadrática:
\begin{equation}
    t = \frac{-b \pm \sqrt{b^2 - 4ac}}{2a} = \frac{9.33 \pm \sqrt{(-9.33)^2 - 4(4.9)(-2.00)}}{2(4.9)}
\end{equation}

\begin{align*}
    t &= \frac{9.33 \pm \sqrt{87.05 + 39.2}}{9.8} \\
    t &= \frac{9.33 \pm \sqrt{126.25}}{9.8} \\
    t &= \frac{9.33 \pm 11.24}{9.8}
\end{align*}

Tomando la solución positiva:
\begin{align*}
    t_{\text{total}} &= \frac{9.33 + 11.24}{9.8} \\
    t_{\text{total}} &= \frac{20.57}{9.8} \\
    t_{\text{total}} &= 2.10 \text{ s}
\end{align*}

\subsection*{Parte e) Alcance horizontal total}

El alcance horizontal es:
\begin{align*}
    R &= v_{0x} \times t_{\text{total}} \\
    R &= 7.55 \times 2.10 \\
    R &= 15.9 \text{ m}
\end{align*}

\subsection*{Parte f) Velocidad justo antes del impacto}

\textbf{Componente horizontal (constante):}
\begin{equation}
    v_x = v_{0x} = 7.55 \text{ m/s}
\end{equation}

\textbf{Componente vertical:}
\begin{align*}
    v_y &= v_{0y} - gt_{\text{total}} \\
    v_y &= 9.33 - 9.8 \times 2.10 \\
    v_y &= 9.33 - 20.58 \\
    v_y &= -11.25 \text{ m/s}
\end{align*}

El signo negativo indica que la bala se está moviendo hacia abajo.

\textbf{Magnitud de la velocidad:}
\begin{align*}
    |\vec{v}| &= \sqrt{v_x^2 + v_y^2} \\
    |\vec{v}| &= \sqrt{(7.55)^2 + (-11.25)^2} \\
    |\vec{v}| &= \sqrt{57.00 + 126.56} \\
    |\vec{v}| &= \sqrt{183.56} \\
    |\vec{v}| &= 13.5 \text{ m/s}
\end{align*}

\textbf{Dirección:}
\begin{align*}
    \theta &= \arctan\left(\frac{|v_y|}{v_x}\right) \\
    \theta &= \arctan\left(\frac{11.25}{7.55}\right) \\
    \theta &= \arctan(1.490) \\
    \theta &= 56.1°
\end{align*}

La velocidad forma un ángulo de $56.1°$ debajo de la horizontal.

\end{solucionbox}

\section{Resultados Finales}

\begin{resultadobox}

\textbf{a) Componentes de velocidad inicial:}
\begin{align*}
    &\boxed{v_{0x} = 7.55 \text{ m/s}} \\
    &\boxed{v_{0y} = 9.33 \text{ m/s}}
\end{align*}

\vspace{0.3cm}

\textbf{b) Tiempo hasta altura máxima:}
\begin{equation*}
    \boxed{t_{\text{máx}} = 0.952 \text{ s}}
\end{equation*}

\vspace{0.3cm}

\textbf{c) Altura máxima sobre el suelo:}
\begin{equation*}
    \boxed{h_{\text{máx}} = 6.44 \text{ m}}
\end{equation*}

\vspace{0.3cm}

\textbf{d) Tiempo total de vuelo:}
\begin{equation*}
    \boxed{t_{\text{total}} = 2.10 \text{ s}}
\end{equation*}

\vspace{0.3cm}

\textbf{e) Alcance horizontal:}
\begin{equation*}
    \boxed{R = 15.9 \text{ m}}
\end{equation*}

\vspace{0.3cm}

\textbf{f) Velocidad al impactar:}
\begin{itemize}
    \item Componentes: $v_x = 7.55$ m/s, $v_y = -11.25$ m/s
    \item Magnitud: $\boxed{|\vec{v}| = 13.5 \text{ m/s}}$
    \item Dirección: $\boxed{\theta = 56.1° \text{ debajo de la horizontal}}$
\end{itemize}

\end{resultadobox}

\section{Verificación}

\subsection*{Conservación de energía}

Verificamos usando conservación de energía mecánica:

\textbf{Energía inicial (en el lanzamiento):}
\begin{align*}
    E_0 &= \frac{1}{2}mv_0^2 + mgh_0 \\
    E_0 &= \frac{1}{2}m(12.0)^2 + m(9.8)(2.00) \\
    E_0 &= 72m + 19.6m \\
    E_0 &= 91.6m \text{ J}
\end{align*}

\textbf{Energía final (al impactar):}
\begin{align*}
    E_f &= \frac{1}{2}m|\vec{v}_f|^2 + mg(0) \\
    E_f &= \frac{1}{2}m(13.5)^2 \\
    E_f &= 91.1m \text{ J}
\end{align*}

La pequeña diferencia ($0.5\%$) se debe al redondeo. ¡La energía se conserva! $\checkmark$

\subsection*{Asimetría del vuelo}

A diferencia de un lanzamiento desde el suelo, aquí el tiempo de subida NO es igual al tiempo de bajada:

\begin{itemize}
    \item Tiempo de subida: $t_{\text{máx}} = 0.952$ s
    \item Tiempo de bajada: $t_{\text{total}} - t_{\text{máx}} = 2.10 - 0.952 = 1.15$ s
    \item Relación: $\frac{t_{\text{bajada}}}{t_{\text{subida}}} = \frac{1.15}{0.952} = 1.21$
\end{itemize}

La bala tarda más tiempo en bajar porque tiene que recorrer una mayor distancia vertical (desde 6.44 m hasta 0 m) que la que sube (desde 2.00 m hasta 6.44 m).

\subsection*{Velocidad final mayor que velocidad inicial}

\begin{equation}
    |\vec{v}_f| = 13.5 \text{ m/s} > v_0 = 12.0 \text{ m/s}
\end{equation}

Esto ocurre porque la bala termina a un nivel más bajo que el punto de lanzamiento, ganando energía cinética a expensas de energía potencial gravitacional.

\section{Análisis Físico}

\subsection{Características del lanzamiento de bala}

\begin{enumerate}
    \item \textbf{Ventaja de la altura inicial:}
    \begin{itemize}
        \item La altura inicial de 2.00 m aumenta el alcance
        \item Un lanzador más alto tiene ventaja
        \item La bala permanece más tiempo en el aire
    \end{itemize}

    \item \textbf{Ángulo de 51°:}
    \begin{itemize}
        \item Ligeramente mayor que 45° (ángulo óptimo desde el suelo)
        \item Favorece altura sobre alcance
        \item Apropiado para lanzamiento desde altura
    \end{itemize}

    \item \textbf{Alcance de 15.9 m:}
    \begin{itemize}
        \item Distancia considerable para una velocidad inicial de 12 m/s
        \item En competencia olímpica, las marcas superan los 20 m
        \item Los atletas profesionales lanzan a velocidades de 13-14 m/s
    \end{itemize}

    \item \textbf{Velocidad de impacto mayor:}
    \begin{itemize}
        \item La bala gana velocidad al caer desde altura
        \item Aterriza con más energía cinética que la inicial
        \item Ángulo de impacto (56.1°) mayor que ángulo de lanzamiento (51.0°)
    \end{itemize}
\end{enumerate}

\subsection{Contexto deportivo}

En el lanzamiento de bala olímpico:

\begin{center}
\begin{tikzpicture}[scale=0.6]
    % Círculo de lanzamiento
    \draw[thick, fill=gray!20] (0,0) circle (2.135);
    \node at (0,-3) {Círculo de lanzamiento};
    \node at (0,-3.5) {(diámetro 2.135 m)};

    % Línea de lanzamiento
    \draw[very thick, red] (-2.135,0) -- (2.135,0);

    % Trayectoria
    \draw[-{Latex[length=3mm]}, blue, very thick] (0,0.5) -- (15.9,0);
    \node[above, blue] at (7.95,0.5) {Alcance: 15.9 m};

    % Zona válida de caída
    \draw[thick, green!50!black, dashed] (0,0) -- (11,11);
    \draw[thick, green!50!black, dashed] (0,0) -- (11,-11);
    \node[green!50!black] at (13,0) {Sector de 34.92°};

    % Punto de impacto
    \filldraw[red] (15.9,0) circle (0.2);
\end{tikzpicture}
\end{center}

\textbf{Récords olímpicos:}
\begin{itemize}
    \item Hombres: 23.30 m (Ryan Crouser, 2021)
    \item Mujeres: 22.41 m (Natalya Lisovskaya, 1987)
\end{itemize}

\section{Gráficas del Movimiento}

\subsection{Trayectoria en el plano $xy$}

\begin{center}
\begin{tikzpicture}
\begin{axis}[
    width=13cm,
    height=8cm,
    xlabel={Distancia horizontal $x$ (m)},
    ylabel={Altura $y$ (m)},
    xmin=0, xmax=17,
    ymin=0, ymax=7,
    grid=major,
    grid style={dashed,gray!30},
    legend pos=north east
]

% Trayectoria: y = 2 + 1.2349*x - 0.2058*x^2
\addplot[
    red!70!black,
    very thick,
    domain=0:15.9,
    samples=100,
    smooth
] {2 + 1.2349*x - 0.2058*x^2};
\addlegendentry{Trayectoria}

% Punto de lanzamiento
\addplot[
    mark=*,
    mark size=4pt,
    blue!70!black,
    only marks
] coordinates {(0, 2.0)};
\addlegendentry{Lanzamiento}

% Punto de altura máxima
\addplot[
    mark=*,
    mark size=4pt,
    green!50!black,
    only marks
] coordinates {(7.18, 6.44)};
\addlegendentry{Altura máxima}

% Punto de impacto
\addplot[
    mark=*,
    mark size=4pt,
    red!70!black,
    only marks
] coordinates {(15.9, 0)};
\addlegendentry{Impacto}

\end{axis}
\end{tikzpicture}
\end{center}

\subsection{Posición vertical vs. tiempo}

\begin{center}
\begin{tikzpicture}
\begin{axis}[
    width=12cm,
    height=8cm,
    xlabel={Tiempo $t$ (s)},
    ylabel={Altura $y$ (m)},
    xmin=0, xmax=2.3,
    ymin=0, ymax=7,
    grid=major,
    grid style={dashed,gray!30},
    legend pos=north east
]

% y(t) = 2 + 9.33t - 4.9t^2
\addplot[
    red!70!black,
    very thick,
    domain=0:2.10,
    samples=100
] {2 + 9.33*x - 4.9*x^2};
\addlegendentry{$y(t)$}

% Punto de lanzamiento
\addplot[
    mark=*,
    mark size=4pt,
    blue!70!black,
    only marks
] coordinates {(0, 2.0)};
\addlegendentry{$t=0$ s}

% Punto de altura máxima
\addplot[
    mark=*,
    mark size=4pt,
    green!50!black,
    only marks
] coordinates {(0.952, 6.44)};
\addlegendentry{$t_{\text{máx}}=0.952$ s}

% Punto de impacto
\addplot[
    mark=*,
    mark size=4pt,
    red!70!black,
    only marks
] coordinates {(2.10, 0)};
\addlegendentry{$t_{\text{total}}=2.10$ s}

\end{axis}
\end{tikzpicture}
\end{center}

\subsection{Velocidad vertical vs. tiempo}

\begin{center}
\begin{tikzpicture}
\begin{axis}[
    width=12cm,
    height=8cm,
    xlabel={Tiempo $t$ (s)},
    ylabel={Velocidad $v_y$ (m/s)},
    xmin=0, xmax=2.3,
    ymin=-12, ymax=10,
    grid=major,
    grid style={dashed,gray!30},
    legend pos=north east
]

% vy(t) = 9.33 - 9.8t
\addplot[
    red!70!black,
    very thick,
    domain=0:2.10,
    samples=50
] {9.33 - 9.8*x};
\addlegendentry{$v_y(t) = 9.33 - 9.8t$}

% Punto inicial
\addplot[
    mark=*,
    mark size=4pt,
    blue!70!black,
    only marks
] coordinates {(0, 9.33)};
\addlegendentry{$v_{0y} = 9.33$ m/s}

% Punto donde vy=0
\addplot[
    mark=*,
    mark size=4pt,
    green!50!black,
    only marks
] coordinates {(0.952, 0)};
\addlegendentry{$v_y=0$ en $t_{\text{máx}}$}

% Punto final
\addplot[
    mark=*,
    mark size=4pt,
    red!70!black,
    only marks
] coordinates {(2.10, -11.25)};
\addlegendentry{$v_y = -11.25$ m/s}

% Línea y=0
\addplot[
    dashed,
    black,
    domain=0:2.10
] {0};

\end{axis}
\end{tikzpicture}
\end{center}

\subsection{Rapidez (magnitud de velocidad) vs. tiempo}

\begin{center}
\begin{tikzpicture}
\begin{axis}[
    width=12cm,
    height=7cm,
    xlabel={Tiempo $t$ (s)},
    ylabel={Rapidez $|\vec{v}|$ (m/s)},
    xmin=0, xmax=2.3,
    ymin=7, ymax=14,
    grid=major,
    grid style={dashed,gray!30},
    legend pos=south east
]

% |v(t)| = sqrt(vx^2 + vy^2) = sqrt(7.55^2 + (9.33-9.8t)^2)
\addplot[
    blue!70!black,
    very thick,
    domain=0:2.10,
    samples=100,
    smooth
] {sqrt(7.55^2 + (9.33 - 9.8*x)^2)};
\addlegendentry{$|\vec{v}(t)|$}

% Punto inicial
\addplot[
    mark=*,
    mark size=4pt,
    blue!70!black,
    only marks
] coordinates {(0, 12.0)};
\addlegendentry{$v_0 = 12.0$ m/s}

% Punto de mínima rapidez (altura máxima)
\addplot[
    mark=*,
    mark size=4pt,
    green!50!black,
    only marks
] coordinates {(0.952, 7.55)};
\addlegendentry{Mínima: 7.55 m/s}

% Punto final
\addplot[
    mark=*,
    mark size=4pt,
    red!70!black,
    only marks
] coordinates {(2.10, 13.5)};
\addlegendentry{$v_f = 13.5$ m/s}

\end{axis}
\end{tikzpicture}
\end{center}

\section{Observaciones Importantes}

\begin{enumerate}
    \item \textbf{Efecto de la altura inicial:} La altura de 2.00 m aumenta el alcance y el tiempo de vuelo comparado con un lanzamiento desde el suelo.

    \item \textbf{Rapidez mínima en altura máxima:} La rapidez mínima es $v_x = 7.55$ m/s, que ocurre en el punto más alto.

    \item \textbf{Velocidad de impacto:} La bala impacta con mayor velocidad (13.5 m/s) que la inicial (12.0 m/s) debido a la conversión de energía potencial a cinética.

    \item \textbf{Ángulos de lanzamiento e impacto:} El ángulo de lanzamiento (51°) es menor que el ángulo de impacto (56.1°), mostrando la asimetría del movimiento.

    \item \textbf{Asimetría temporal:} El tiempo de bajada (1.15 s) es mayor que el tiempo de subida (0.952 s).

    \item \textbf{Ventaja competitiva:} En competencia, cada metro extra de altura del lanzador puede significar centímetros adicionales de alcance.
\end{enumerate}

\section{Comparación con lanzamiento desde el suelo}

Si la bala se lanzara con la misma velocidad y ángulo desde el suelo ($h_0 = 0$):

\textbf{Tiempo de vuelo:}
\begin{equation}
    t'_{\text{total}} = \frac{2v_{0y}}{g} = \frac{2 \times 9.33}{9.8} = 1.90 \text{ s}
\end{equation}

\textbf{Alcance:}
\begin{equation}
    R' = v_{0x} \times t'_{\text{total}} = 7.55 \times 1.90 = 14.3 \text{ m}
\end{equation}

\textbf{Ganancia por altura:}
\begin{equation}
    \Delta R = R - R' = 15.9 - 14.3 = 1.6 \text{ m}
\end{equation}

La altura inicial de 2.00 m proporciona 1.6 m adicionales de alcance, ¡un aumento del 11\%!

\end{document}
