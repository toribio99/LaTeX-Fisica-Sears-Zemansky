\documentclass[11pt,a4paper]{article}

% Paquetes necesarios
\usepackage[utf8]{inputenc}
\usepackage[T1]{fontenc}
\usepackage[spanish]{babel}
\usepackage[margin=2.5cm]{geometry}
\usepackage{amsmath}
\usepackage{amssymb}
\usepackage{xcolor}
\usepackage{tcolorbox}
\usepackage{graphicx}
\usepackage{tikz}
\usepackage{pgfplots}
\pgfplotsset{compat=1.18}
\usetikzlibrary{arrows.meta,patterns,decorations.pathmorphing}
\usepackage{wrapfig}
\usepackage[export]{adjustbox} % (opcional) claves extra para \includegraphics
\usepackage{xparse}

% Definición de colores
\definecolor{azuloscuro}{RGB}{0,51,102}
\definecolor{azulclaro}{RGB}{230,240,250}
\definecolor{verdeoscuro}{RGB}{0,100,0}
\definecolor{rojoclaro}{RGB}{255,230,230}

% Configuración de cajas
\tcbuselibrary{theorems,skins,breakable}

\newtcolorbox{datosbox}{
    colback=azulclaro,
    colframe=azuloscuro,
    fonttitle=\bfseries,
    title=Datos del Problema,
    sharp corners,
    boxrule=1pt
}

\newtcolorbox{solucionbox}{
    colback=white,
    colframe=verdeoscuro,
    fonttitle=\bfseries,
    title=Desarrollo de la Solución,
    sharp corners,
    boxrule=1pt,
    breakable
}

\newtcolorbox{resultadobox}{
    colback=rojoclaro,
    colframe=red!70!black,
    fonttitle=\bfseries,
    title=Resultado Final,
    sharp corners,
    boxrule=2pt
}

% Título y autor
\title{\textbf{Solución del Ejercicio 3.14} \\
\large Movimiento de Proyectiles: Canica y Cavidad}
\author{Física Universitaria - Sears y Zemansky \\ Capítulo 3: Movimiento en Dos o Tres Dimensiones}
\date{\today}

\begin{document}

\maketitle

\section{Enunciado del Problema}

\begin{wrapfigure}[10]{r}{.6\textwidth}  % Ajusta el ancho según necesites
	\centering
	\vspace{-\baselineskip}
	%\begin{center}
\begin{tikzpicture}[scale=1.2]
    % Plataforma
    \draw[fill=brown!30, very thick] (0,0) rectangle (2,3);
    \draw[fill=brown!50] (0,3) rectangle (2,3.2);

    % Suelo
    \draw[fill=gray!20, thick] (-1,-0.2) rectangle (6,0);
    \draw[pattern=north east lines, pattern color=gray] (-1,-0.4) rectangle (6,-0.2);

    % Cavidad
    \draw[fill=white, very thick] (4,-0.2) rectangle (5.5,0);

    % Canica
    \filldraw[red!70!black] (2,3) circle (0.1);
    \draw[-{Latex[length=3mm]},red!70!black, ultra thick] (2,3) -- (3,3) node[midway,above] {$v_0$};

    % Trayectoria parabólica (ecuación física real: y = y0 - (g/(2v0^2))*(x-x0)^2)
    % Para ilustración con v0 ≈ 3.5 m/s (intermedia): coeficiente = 0.40
    \draw[blue!70!black, thick, dashed, -{Latex[length=2mm]}]
        plot[domain=2:4.75, samples=50, smooth] (\x, {3 - 0.40*(\x-2)*(\x-2)});

    % Cotas
    \draw[{Latex[length=2mm]}-{Latex[length=2mm]}, thick] (-0.5,0) -- (-0.5,3) node[midway,left] {2.75 m};
    \draw[{Latex[length=2mm]}-{Latex[length=2mm]}, thick] (2,-0.7) -- (4,-0.7) node[midway,below] {2.00 m};
    \draw[{Latex[length=2mm]}-{Latex[length=2mm]}, thick] (4,-0.9) -- (5.5,-0.9) node[midway,below] {1.50 m};

    % Etiquetas
    \node at (1,1.5) {Plataforma};
    \node at (4.75,-.55) {\small Cavidad};

\end{tikzpicture}
%\end{center}
\vspace{-\baselineskip} % Reduce espacio después
\end{wrapfigure}
Una pequeña canica rueda horizontalmente con una rapidez $v_0$ y cae desde la parte superior de una plataforma de 2.75 m de alto, sin que sufra resistencia del aire. A nivel del piso, a 2.00 m de la base de la plataforma, hay una cavidad de 1.50 m de ancho (figura 3.40).

\textbf{Pregunta:} ¿En qué intervalo de rapideces $v_0$ la canica caerá dentro de la cavidad?

\section{Datos del Problema}

\begin{datosbox}
\begin{itemize}
    \item \textbf{Altura de la plataforma:} $h = 2.75$ m
    \item \textbf{Distancia horizontal a la cavidad:} $x_1 = 2.00$ m (borde cercano)
    \item \textbf{Ancho de la cavidad:} $\Delta x = 1.50$ m
    \item \textbf{Distancia al borde lejano:} $x_2 = 2.00 + 1.50 = 3.50$ m
    \item \textbf{Velocidad inicial vertical:} $v_{0y} = 0$ (movimiento horizontal)
    \item \textbf{Aceleración de la gravedad:} $g = 9.8$ m/s$^2$
    \item \textbf{Resistencia del aire:} despreciable
\end{itemize}
\end{datosbox}

\section{Marco Teórico}

Según la teoría de movimiento de proyectiles (Sección 3.3 del texto), cuando un objeto se mueve como proyectil:

\subsection{Ecuaciones Fundamentales}

\textbf{Movimiento horizontal (componente x):}
\begin{equation}
    x = x_0 + v_{0x} t
\end{equation}

Como la canica parte del borde de la plataforma ($x_0 = 0$) y rueda horizontalmente ($v_{0x} = v_0$):
\begin{equation}
    x = v_0 t
\end{equation}

\textbf{Movimiento vertical (componente y):}
\begin{equation}
    y = y_0 + v_{0y} t - \frac{1}{2}gt^2
\end{equation}

Como la velocidad inicial vertical es cero ($v_{0y} = 0$):
\begin{equation}
    y = y_0 - \frac{1}{2}gt^2
\end{equation}

\section{Desarrollo de la Solución}

\begin{solucionbox}

\subsection*{Paso 1: Calcular el tiempo de caída}

La canica cae desde una altura $y_0 = 2.75$ m hasta el nivel del suelo $y = 0$.

Usando la ecuación (4):
\begin{align*}
    y &= y_0 - \frac{1}{2}gt^2 \\
    0 &= 2.75 - \frac{1}{2}(9.8)t^2 \\
    0 &= 2.75 - 4.9t^2 \\
    4.9t^2 &= 2.75 \\
    t^2 &= \frac{2.75}{4.9} \\
    t^2 &= 0.5612 \\
    t &= \sqrt{0.5612} \\
    t &= 0.749 \text{ s}
\end{align*}

\textbf{Tiempo de caída:} $t = 0.749$ s

\subsection*{Paso 2: Calcular la velocidad mínima}

Para que la canica caiga dentro de la cavidad, debe recorrer horizontalmente \textit{al menos} la distancia hasta el borde cercano de la cavidad.

Condición: $x \geq x_1 = 2.00$ m

Para la velocidad mínima, la canica debe llegar exactamente al borde cercano:
\begin{align*}
    x_1 &= v_{0\text{min}} \cdot t \\
    2.00 &= v_{0\text{min}} \cdot 0.749 \\
    v_{0\text{min}} &= \frac{2.00}{0.749} \\
    v_{0\text{min}} &= 2.67 \text{ m/s}
\end{align*}

\subsection*{Paso 3: Calcular la velocidad máxima}

Para que la canica caiga dentro de la cavidad, no debe superar el borde lejano de la cavidad.

Condición: $x \leq x_2 = 3.50$ m

Para la velocidad máxima, la canica debe llegar exactamente al borde lejano:
\begin{align*}
    x_2 &= v_{0\text{max}} \cdot t \\
    3.50 &= v_{0\text{max}} \cdot 0.749 \\
    v_{0\text{max}} &= \frac{3.50}{0.749} \\
    v_{0\text{max}} &= 4.67 \text{ m/s}
\end{align*}

\end{solucionbox}

\section{Resultado Final}

\begin{resultadobox}
Para que la canica caiga dentro de la cavidad, su rapidez inicial $v_0$ debe estar en el intervalo:

\begin{equation}
    \boxed{2.67 \text{ m/s} \leq v_0 \leq 4.67 \text{ m/s}}
\end{equation}

O expresado de otra forma:
\begin{itemize}
    \item \textbf{Velocidad mínima:} $v_{0\text{min}} = 2.67$ m/s
    \item \textbf{Velocidad máxima:} $v_{0\text{max}} = 4.67$ m/s
    \item \textbf{Intervalo de velocidades:} $v_0 \in [2.67, 4.67]$ m/s
\end{itemize}
\end{resultadobox}

\section{Verificación y Análisis}

\subsection{Verificación con la velocidad mínima}

Si $v_0 = 2.67$ m/s:
\begin{align*}
    x &= v_0 t = 2.67 \times 0.749 = 2.00 \text{ m} \quad \checkmark
\end{align*}

La canica llega exactamente al borde cercano de la cavidad.

\subsection{Verificación con la velocidad máxima}

Si $v_0 = 4.67$ m/s:
\begin{align*}
    x &= v_0 t = 4.67 \times 0.749 = 3.50 \text{ m} \quad \checkmark
\end{align*}

La canica llega exactamente al borde lejano de la cavidad.

\subsection{Análisis físico}

\begin{itemize}
    \item Si $v_0 < 2.67$ m/s: La canica cae \textit{antes} de llegar a la cavidad (en el suelo sólido).

    \item Si $2.67 \leq v_0 \leq 4.67$ m/s: La canica cae \textit{dentro} de la cavidad.

    \item Si $v_0 > 4.67$ m/s: La canica sobrepasa la cavidad y cae \textit{después} de ella (en el suelo sólido).
\end{itemize}

\section{Conceptos Clave Aplicados}

\begin{enumerate}
    \item \textbf{Movimiento de proyectil:} La canica experimenta movimiento bidimensional con aceleración constante (gravedad).

    \item \textbf{Independencia de componentes:} El movimiento horizontal (velocidad constante) y el movimiento vertical (aceleración constante) se analizan por separado.

    \item \textbf{Tiempo de vuelo:} El tiempo que tarda en caer depende únicamente de la altura inicial y la gravedad, no de la velocidad horizontal.

    \item \textbf{Alcance horizontal:} La distancia horizontal recorrida depende directamente de la velocidad inicial horizontal y del tiempo de vuelo.
\end{enumerate}

\section{Ecuaciones Utilizadas - Resumen}

\begin{tcolorbox}[colback=yellow!10!white,colframe=orange!75!black,title=Ecuaciones del Movimiento de Proyectiles]

\textbf{Tiempo de caída (desde altura $h$):}
\begin{equation*}
    t = \sqrt{\frac{2h}{g}}
\end{equation*}

\textbf{Alcance horizontal:}
\begin{equation*}
    x = v_0 t = v_0 \sqrt{\frac{2h}{g}}
\end{equation*}

\textbf{Para este problema específico:}
\begin{align*}
    t &= \sqrt{\frac{2 \times 2.75}{9.8}} = 0.749 \text{ s} \\
    v_{0\text{min}} &= \frac{x_1}{t} = \frac{2.00}{0.749} = 2.67 \text{ m/s} \\
    v_{0\text{max}} &= \frac{x_2}{t} = \frac{3.50}{0.749} = 4.67 \text{ m/s}
\end{align*}

\end{tcolorbox}

\end{document}
