\documentclass[11pt,a4paper]{article}

% Paquetes necesarios
\usepackage[utf8]{inputenc}
\usepackage[T1]{fontenc}
\usepackage[spanish]{babel}
\usepackage[margin=2.5cm]{geometry}
\usepackage{amsmath}
\usepackage{amssymb}
\usepackage{xcolor}
\usepackage{tcolorbox}
\usepackage{graphicx}
\usepackage{tikz}
\usepackage{pgfplots}
\pgfplotsset{compat=1.18}
\usetikzlibrary{arrows.meta,patterns,decorations.pathmorphing,calc}
\usepackage{wrapfig}
\usepackage[export]{adjustbox} % (opcional) claves extra para \includegraphics
\usepackage{xparse}

% Definición de colores
\definecolor{azuloscuro}{RGB}{0,51,102}
\definecolor{azulclaro}{RGB}{230,240,250}
\definecolor{verdeoscuro}{RGB}{0,100,0}
\definecolor{rojoclaro}{RGB}{255,230,230}

% Configuración de cajas
\tcbuselibrary{theorems,skins,breakable}

\newtcolorbox{datosbox}{
    colback=azulclaro,
    colframe=azuloscuro,
    fonttitle=\bfseries,
    title=Datos del Problema,
    sharp corners,
    boxrule=1pt
}

\newtcolorbox{solucionbox}{
    colback=white,
    colframe=verdeoscuro,
    fonttitle=\bfseries,
    title=Desarrollo de la Solución,
    sharp corners,
    boxrule=1pt,
    breakable
}

\newtcolorbox{resultadobox}{
    colback=rojoclaro,
    colframe=red!70!black,
    fonttitle=\bfseries,
    title=Resultado Final,
    sharp corners,
    boxrule=2pt
}

% Título y autor
\title{\textbf{Solución del Ejercicio 3.23} \\
\large Piedra Lanzada desde la Azotea de un Edificio}
\author{Física Universitaria - Sears y Zemansky \\ Capítulo 3: Movimiento en Dos o Tres Dimensiones}
\date{\today}

\begin{document}

\maketitle

\section{Enunciado del Problema}

\begin{wrapfigure}[9]{r}{.7\textwidth}  % Ajusta el ancho según necesites
	\centering
	\vspace{-\baselineskip}
	%\begin{center}
\begin{tikzpicture}[scale=0.72]
    % Edificio
    \draw[fill=gray!30, thick] (0,0) rectangle (1,3);
    \draw[pattern=north east lines, pattern color=gray] (0,0) rectangle (1,-0.2);
    \node[right, font=\scriptsize] at (1,1.5) {Edificio};

    % Azotea
    \draw[fill=brown!30, thick] (0,3) rectangle (1,3.1);

    % Persona en la azotea
    \draw[fill=blue!20, thick] (0.5,3.1) circle (0.08);
    \node[above, font=\scriptsize] at (0.5,3.25) {Lanzador};

    % Vector velocidad inicial
    \draw[-{Latex[length=2.5mm]},blue!70!black, ultra thick] (0.5,3.1) -- (1.3,3.65);
    \node[blue!70!black, above right, font=\scriptsize] at (0.9,3.5) {$v_0=30$ m/s};
    \node[blue!70!black, below, font=\scriptsize] at (1.1,3.3) {$33°$};

    % Ángulo
    \draw[thick] (0.7,3.1) arc (0:33:0.2);

    % Trayectoria parabólica completa hasta el suelo
    % Ecuación: y = 3.1 + 0.649x - 0.0773x²
    % Toca el suelo (y=0) en x = 11.79
    \draw[red!70!black, very thick, -{Latex[length=2.5mm]}]
        plot[domain=0:11.79, samples=150, smooth] (0.5+\x, {3.1 + 0.649*\x - 0.0773*\x*\x});

    % Punto más alto
    \filldraw[orange!80!black] (0.5+4.2,4.463) circle (0.08);
    \node[orange!80!black, above, font=\scriptsize] at (0.5+4.2,4.563) {Altura máxima};
    \node[orange!80!black, below, font=\scriptsize] at (0.5+4.2,4.363) {$h_{max}$};

    % Punto de impacto en el suelo
    \filldraw[green!60!black] (0.5+11.79,0) circle (0.08);
    \node[green!60!black, below, font=\scriptsize] at (0.5+11.79,-0.2) {Impacto};

    % Altura del edificio
    \draw[{Latex[length=2mm]}-{Latex[length=2mm]}, thick] (-0.3,0) -- (-0.3,3);
    \node[left, font=\small] at (-0.3,1.5) {$15.0$ m};

    % Distancia horizontal
    \draw[{Latex[length=2mm]}-{Latex[length=2mm]}, thick] (1,-0.5) -- (12.29,-0.5);
    \node[below, font=\small] at (6.5,-0.5) {$R = ?$};

    % Suelo
    \draw[thick] (-0.5,0) -- (13,0);
    \draw[pattern=north east lines, pattern color=brown] (-0.5,0) rectangle (13,-0.2);

\end{tikzpicture}
%\end{center}
\vspace{-\baselineskip} % Reduce espacio después
\end{wrapfigure}
Un hombre está parado en la azotea de un edificio de 15.0 m y lanza una piedra con velocidad de 30.0 m/s en un ángulo de 33.0° sobre la horizontal. Puede despreciarse la resistencia del aire. Calcule:

\begin{enumerate}
	\item[a)] La altura máxima que alcanza la piedra sobre la azotea
\end{enumerate}

\begin{enumerate}
	\item[b)] La magnitud de la velocidad de la piedra justo antes de golpear el suelo
	\item[c)] La distancia horizontal desde la base del edificio hasta el punto donde la roca golpea el suelo
	\item[d)] Dibuje las gráficas $x$-$t$, $y$-$t$, $v_x$-$t$ y $v_y$-$t$ para el movimiento
\end{enumerate}

\section{Datos del Problema}

\begin{datosbox}
\begin{itemize}
    \item \textbf{Altura del edificio:} $h_0 = 15.0$ m
    \item \textbf{Velocidad inicial:} $v_0 = 30.0$ m/s
    \item \textbf{Ángulo de lanzamiento:} $\alpha_0 = 33.0°$
    \item \textbf{Aceleración gravitacional:} $g = 9.8$ m/s$^2$
    \item \textbf{Posición inicial:} $x_0 = 0$, $y_0 = 15.0$ m (nivel de la azotea)
    \item \textbf{Resistencia del aire:} despreciable
\end{itemize}
\end{datosbox}

\section{Marco Teórico}

Para un proyectil lanzado desde una altura $h_0$ con velocidad inicial $v_0$ a un ángulo $\alpha_0$:

\textbf{Componentes de velocidad inicial:}
\begin{align}
    v_{0x} &= v_0 \cos\alpha_0 \\
    v_{0y} &= v_0 \sin\alpha_0
\end{align}

\textbf{Ecuaciones de posición (con $y_0 = h_0$):}
\begin{align}
    x(t) &= v_{0x}t \\
    y(t) &= h_0 + v_{0y}t - \frac{1}{2}gt^2
\end{align}

\textbf{Ecuaciones de velocidad:}
\begin{align}
    v_x(t) &= v_{0x} \quad \text{(constante)} \\
    v_y(t) &= v_{0y} - gt
\end{align}

\textbf{Altura máxima sobre el punto de lanzamiento:}
\begin{equation}
    h_{max,rel} = \frac{v_{0y}^2}{2g}
\end{equation}

\textbf{Altura máxima sobre el suelo:}
\begin{equation}
    h_{max,abs} = h_0 + h_{max,rel}
\end{equation}

\section{Desarrollo de la Solución}

\begin{solucionbox}

\subsection*{Paso 1: Componentes de la velocidad inicial}

Calculamos las componentes horizontal y vertical de la velocidad inicial:

\begin{align}
    v_{0x} &= v_0 \cos\alpha_0 = (30.0 \text{ m/s}) \cos(33.0°) \\
    v_{0x} &= (30.0 \text{ m/s})(0.8387) = 25.16 \text{ m/s}
\end{align}

\begin{align}
    v_{0y} &= v_0 \sin\alpha_0 = (30.0 \text{ m/s}) \sin(33.0°) \\
    v_{0y} &= (30.0 \text{ m/s})(0.5446) = 16.34 \text{ m/s}
\end{align}

\subsection*{Parte a): Altura máxima sobre la azotea}

La piedra alcanza su altura máxima cuando la velocidad vertical es cero. Usando la ecuación cinemática:

\begin{equation}
    v_y^2 = v_{0y}^2 - 2gh
\end{equation}

En el punto más alto, $v_y = 0$:

\begin{align}
    0 &= v_{0y}^2 - 2gh_{max} \\
    h_{max} &= \frac{v_{0y}^2}{2g} = \frac{(16.34 \text{ m/s})^2}{2(9.8 \text{ m/s}^2)} \\
    h_{max} &= \frac{267.00}{19.6} = 13.62 \text{ m}
\end{align}

La piedra alcanza una altura de \textbf{13.62 metros} sobre la azotea, lo que equivale a $15.0 + 13.62 = 28.62$ m sobre el suelo.

\subsection*{Parte b): Velocidad justo antes de golpear el suelo}

Primero necesitamos encontrar el tiempo en que la piedra llega al suelo. La piedra golpea el suelo cuando $y = 0$. Usando:

\begin{equation}
    y = h_0 + v_{0y}t - \frac{1}{2}gt^2
\end{equation}

Sustitu yendo $y = 0$ y $h_0 = 15.0$ m:

\begin{align}
    0 &= 15.0 + 16.34t - 4.9t^2 \\
    4.9t^2 - 16.34t - 15.0 &= 0
\end{align}

Usando la fórmula cuadrática:

\begin{align}
    t &= \frac{16.34 \pm \sqrt{(16.34)^2 + 4(4.9)(15.0)}}{2(4.9)} \\
    t &= \frac{16.34 \pm \sqrt{267.00 + 294.00}}{9.8} \\
    t &= \frac{16.34 \pm \sqrt{561.00}}{9.8} \\
    t &= \frac{16.34 \pm 23.69}{9.8}
\end{align}

Tomando la raíz positiva:

\begin{equation}
    t = \frac{40.03}{9.8} = 4.085 \text{ s}
\end{equation}

Ahora calculamos las componentes de la velocidad en $t = 4.085$ s:

\textbf{Componente horizontal (constante):}
\begin{equation}
    v_x = v_{0x} = 25.16 \text{ m/s}
\end{equation}

\textbf{Componente vertical:}
\begin{align}
    v_y &= v_{0y} - gt \\
    v_y &= 16.34 - (9.8)(4.085) \\
    v_y &= 16.34 - 40.03 \\
    v_y &= -23.69 \text{ m/s}
\end{align}

El signo negativo indica que la velocidad vertical apunta hacia abajo.

\textbf{Magnitud de la velocidad:}
\begin{align}
    v &= \sqrt{v_x^2 + v_y^2} \\
    v &= \sqrt{(25.16)^2 + (-23.69)^2} \\
    v &= \sqrt{633.02 + 561.21} \\
    v &= \sqrt{1194.23} = 34.56 \text{ m/s}
\end{align}

\textbf{Ángulo con respecto a la horizontal:}
\begin{equation}
    \theta = \arctan\left(\frac{v_y}{v_x}\right) = \arctan\left(\frac{-23.69}{25.16}\right) = -43.3°
\end{equation}

La piedra golpea el suelo con una velocidad de \textbf{34.56 m/s} a un ángulo de \textbf{43.3° bajo la horizontal}.

\subsection*{Parte c): Distancia horizontal}

La distancia horizontal desde la base del edificio hasta el punto de impacto es:

\begin{align}
    R &= v_{0x} \cdot t \\
    R &= (25.16 \text{ m/s})(4.085 \text{ s}) \\
    R &= 102.8 \text{ m}
\end{align}

La piedra cae a \textbf{102.8 metros} de la base del edificio.

\subsection*{Verificación usando conservación de energía}

Podemos verificar la velocidad final usando conservación de energía:

\begin{equation}
    \frac{1}{2}mv_f^2 = \frac{1}{2}mv_0^2 + mgh_0
\end{equation}

\begin{align}
    v_f^2 &= v_0^2 + 2gh_0 \\
    v_f &= \sqrt{(30.0)^2 + 2(9.8)(15.0)} \\
    v_f &= \sqrt{900 + 294} \\
    v_f &= \sqrt{1194} = 34.56 \text{ m/s} \quad \checkmark
\end{align}

\end{solucionbox}

\section{Resultados Finales}

\begin{resultadobox}

\textbf{Parte a) Altura máxima sobre la azotea:}
\begin{equation}
    \boxed{h_{max} = 13.62 \text{ m}}
\end{equation}

La piedra alcanza su altura máxima \textbf{13.62 metros} sobre la azotea (o 28.62 m sobre el suelo).

\vspace{0.3cm}

\textbf{Parte b) Velocidad justo antes de golpear el suelo:}
\begin{equation}
    \boxed{v = 34.56 \text{ m/s a } 43.3° \text{ bajo la horizontal}}
\end{equation}

Componentes: $v_x = 25.16$ m/s (horizontal), $v_y = -23.69$ m/s (vertical)

\vspace{0.3cm}

\textbf{Parte c) Distancia horizontal desde la base del edificio:}
\begin{equation}
    \boxed{R = 102.8 \text{ m}}
\end{equation}

La piedra cae al suelo a \textbf{102.8 metros} de la base del edificio.

\vspace{0.3cm}

\textbf{Información adicional:}
\begin{itemize}
    \item Tiempo de vuelo total: $t = 4.085$ s
    \item Tiempo hasta alcanzar altura máxima: $t_{max} = v_{0y}/g = 1.668$ s
    \item Altura máxima sobre el suelo: $h_{abs} = 28.62$ m
    \item Velocidad inicial: $v_0 = 30.0$ m/s
    \item Velocidad final: $v_f = 34.56$ m/s (mayor que la inicial debido a la ganancia de energía potencial)
\end{itemize}

\end{resultadobox}

\section{Gráficas del Movimiento}

\subsection*{Parte d): Gráficas $x$-$t$, $y$-$t$, $v_x$-$t$ y $v_y$-$t$}

\subsubsection*{Gráfica Posición horizontal vs. tiempo ($x$-$t$)}

\begin{center}
\begin{tikzpicture}
\begin{axis}[
    width=12cm, height=7cm,
    xlabel={Tiempo $t$ (s)},
    ylabel={Posición horizontal $x$ (m)},
    xmin=0, xmax=4.5,
    ymin=0, ymax=110,
    grid=both,
    grid style={line width=.1pt, draw=gray!30},
    major grid style={line width=.2pt,draw=gray!50},
    legend pos=north west
]
    % x(t) = 25.16*t
    \addplot[domain=0:4.085, samples=100, blue, very thick] {25.16*x};
    \addlegendentry{$x(t) = 25.16t$}

    % Punto de impacto
    \addplot[only marks, mark=*, mark size=3pt, red] coordinates {(4.085, 102.8)};
    \node[red, above right] at (axis cs:4.085,102.8) {Impacto};
\end{axis}
\end{tikzpicture}
\end{center}

La posición horizontal aumenta linealmente con el tiempo, ya que la velocidad horizontal es constante.

\subsubsection*{Gráfica Posición vertical vs. tiempo ($y$-$t$)}

\begin{center}
\begin{tikzpicture}
\begin{axis}[
    width=12cm, height=7cm,
    xlabel={Tiempo $t$ (s)},
    ylabel={Posición vertical $y$ (m)},
    xmin=0, xmax=4.5,
    ymin=0, ymax=30,
    grid=both,
    grid style={line width=.1pt, draw=gray!30},
    major grid style={line width=.2pt,draw=gray!50},
    legend pos=north east
]
    % y(t) = 15 + 16.34*t - 4.9*t^2
    \addplot[domain=0:4.085, samples=100, red, very thick] {15 + 16.34*x - 4.9*x^2};
    \addlegendentry{$y(t) = 15 + 16.34t - 4.9t^2$}

    % Altura máxima
    \addplot[only marks, mark=*, mark size=3pt, orange] coordinates {(1.668, 28.62)};
    \node[orange, above] at (axis cs:1.668,28.62) {Altura máxima};

    % Punto de impacto
    \addplot[only marks, mark=*, mark size=3pt, green!60!black] coordinates {(4.085, 0)};
    \node[green!60!black, below right] at (axis cs:4.085,0) {Impacto};

    % Línea de la azotea
    \addplot[domain=0:4.5, dashed, gray, thick] {15};
    \node[gray, right] at (axis cs:4.3,15) {Azotea};
\end{axis}
\end{tikzpicture}
\end{center}

La posición vertical sigue una parábola. La piedra sube hasta alcanzar su altura máxima y luego desciende hasta el suelo.

\subsubsection*{Gráfica Velocidad horizontal vs. tiempo ($v_x$-$t$)}

\begin{center}
\begin{tikzpicture}
\begin{axis}[
    width=12cm, height=6cm,
    xlabel={Tiempo $t$ (s)},
    ylabel={Velocidad horizontal $v_x$ (m/s)},
    xmin=0, xmax=4.5,
    ymin=0, ymax=30,
    grid=both,
    grid style={line width=.1pt, draw=gray!30},
    major grid style={line width=.2pt,draw=gray!50},
    legend pos=south east
]
    % vx(t) = 25.16 (constante)
    \addplot[domain=0:4.085, samples=50, blue, very thick] {25.16};
    \addlegendentry{$v_x(t) = 25.16$ m/s}

    % Punto de impacto
    \addplot[only marks, mark=*, mark size=3pt, red] coordinates {(4.085, 25.16)};
\end{axis}
\end{tikzpicture}
\end{center}

La velocidad horizontal es constante durante todo el movimiento (no hay aceleración horizontal).

\subsubsection*{Gráfica Velocidad vertical vs. tiempo ($v_y$-$t$)}

\begin{center}
\begin{tikzpicture}
\begin{axis}[
    width=12cm, height=7cm,
    xlabel={Tiempo $t$ (s)},
    ylabel={Velocidad vertical $v_y$ (m/s)},
    xmin=0, xmax=4.5,
    ymin=-25, ymax=20,
    grid=both,
    grid style={line width=.1pt, draw=gray!30},
    major grid style={line width=.2pt,draw=gray!50},
    legend pos=north east
]
    % vy(t) = 16.34 - 9.8*t
    \addplot[domain=0:4.085, samples=100, red, very thick] {16.34 - 9.8*x};
    \addlegendentry{$v_y(t) = 16.34 - 9.8t$}

    % Punto donde vy = 0 (altura máxima)
    \addplot[only marks, mark=*, mark size=3pt, orange] coordinates {(1.668, 0)};
    \node[orange, above right] at (axis cs:1.668,0) {$v_y=0$ (altura máx.)};

    % Punto de impacto
    \addplot[only marks, mark=*, mark size=3pt, green!60!black] coordinates {(4.085, -23.69)};
    \node[green!60!black, below right] at (axis cs:4.085,-23.69) {Impacto};

    % Línea vy = 0
    \addplot[domain=0:4.5, dashed, gray, thick] {0};
\end{axis}
\end{tikzpicture}
\end{center}

La velocidad vertical disminuye linealmente con el tiempo debido a la aceleración gravitacional constante. Es positiva (hacia arriba) al inicio, cero en la altura máxima, y negativa (hacia abajo) durante el descenso.

\section{Trayectoria en el plano $xy$}

\begin{center}
\begin{tikzpicture}
\begin{axis}[
    width=13cm, height=8cm,
    xlabel={Posición horizontal $x$ (m)},
    ylabel={Posición vertical $y$ (m)},
    xmin=0, xmax=110,
    ymin=0, ymax=30,
    grid=both,
    grid style={line width=.1pt, draw=gray!30},
    major grid style={line width=.2pt,draw=gray!50},
    legend pos=north east,
    axis equal image=false
]
    % Trayectoria: y = 15 + 0.6494x - 0.0773x²
    % Derivada de x = 25.16t => t = x/25.16
    % y = 15 + 16.34(x/25.16) - 4.9(x/25.16)²
    % y = 15 + 0.6494x - 0.0773x²
    \addplot[domain=0:102.8, samples=100, red, very thick] {15 + 0.6494*x - 0.000773*x^2};
    \addlegendentry{Trayectoria parabólica}

    % Punto inicial (azotea)
    \addplot[only marks, mark=*, mark size=3pt, blue] coordinates {(0, 15)};
    \node[blue, left] at (axis cs:0,15) {Lanzamiento};

    % Altura máxima
    \addplot[only marks, mark=*, mark size=3pt, orange] coordinates {(42.01, 28.62)};
    \node[orange, above] at (axis cs:42.01,28.62) {Altura máxima};

    % Punto de impacto
    \addplot[only marks, mark=*, mark size=3pt, green!60!black] coordinates {(102.8, 0)};
    \node[green!60!black, below right] at (axis cs:102.8,0) {Impacto};

    % Línea del edificio
    \addplot[domain=0:0.01, blue!50, ultra thick] {x};
    \addplot[blue!50, ultra thick] coordinates {(0,0) (0,15)};

    % Línea del suelo
    \addplot[domain=0:110, brown, thick] {0};
\end{axis}
\end{tikzpicture}
\end{center}

\section{Análisis y Conclusión}

Este problema ilustra el movimiento de un proyectil lanzado desde una altura inicial. Las conclusiones principales son:

\begin{enumerate}
    \item \textbf{Altura máxima:} La piedra alcanza 13.62 m sobre la azotea (28.62 m sobre el suelo). Esta altura es menor que si se lanzara desde el suelo con la misma velocidad inicial, pero la altura total sobre el suelo es mayor.

    \item \textbf{Velocidad final mayor que la inicial:} La velocidad al impacto (34.56 m/s) es mayor que la velocidad inicial (30.0 m/s). Esto se debe a que la piedra convierte energía potencial gravitacional en energía cinética durante la caída desde una altura de 15.0 m.

    \item \textbf{Alcance horizontal:} La distancia de 102.8 m es significativa y se debe a dos factores:
    \begin{itemize}
        \item La alta velocidad horizontal constante (25.16 m/s)
        \item El largo tiempo de vuelo (4.085 s), que incluye tanto el tiempo de subida como el de caída desde una altura inicial de 15 m
    \end{itemize}

    \item \textbf{Simetría del movimiento:} A diferencia de un proyectil lanzado desde el suelo que regresa al mismo nivel, en este caso:
    \begin{itemize}
        \item El tiempo de subida (1.668 s) es menor que el tiempo de bajada (2.417 s)
        \item El ángulo de lanzamiento (33°) es diferente del ángulo de impacto (43.3°)
        \item La velocidad vertical de lanzamiento ($+16.34$ m/s) tiene menor magnitud que la de impacto ($-23.69$ m/s)
    \end{itemize}

    \item \textbf{Gráficas del movimiento:}
    \begin{itemize}
        \item $x$-$t$: Línea recta (velocidad horizontal constante)
        \item $y$-$t$: Parábola (aceleración vertical constante)
        \item $v_x$-$t$: Línea horizontal (sin aceleración horizontal)
        \item $v_y$-$t$: Línea recta decreciente (aceleración $-g$)
    \end{itemize}

    \item \textbf{Conservación de energía:} Verificamos que la velocidad final se puede calcular tanto con cinemática como con conservación de energía, obteniendo el mismo resultado (34.56 m/s).

    \item \textbf{Aplicaciones prácticas:} Este tipo de problema es relevante para:
    \begin{itemize}
        \item Lanzamiento de proyectiles desde plataformas elevadas
        \item Diseño de sistemas de extinción de incendios
        \item Balística y trayectorias de proyectiles
        \item Deportes como el lanzamiento desde trampolines
    \end{itemize}
\end{enumerate}

\end{document}
