\documentclass[11pt,a4paper]{article}

% Paquetes necesarios
\usepackage[utf8]{inputenc}
\usepackage[T1]{fontenc}
\usepackage[spanish]{babel}
\usepackage[margin=2.5cm]{geometry}
\usepackage{amsmath}
\usepackage{amssymb}
\usepackage{xcolor}
\usepackage{tcolorbox}
\usepackage{graphicx}
\usepackage{tikz}
\usepackage{pgfplots}
\pgfplotsset{compat=1.18}
\usetikzlibrary{arrows.meta,patterns,decorations.pathmorphing,calc}
\usepackage{wrapfig}
\usepackage[export]{adjustbox} % (opcional) claves extra para \includegraphics
\usepackage{xparse}


% Definición de colores
\definecolor{azuloscuro}{RGB}{0,51,102}
\definecolor{azulclaro}{RGB}{230,240,250}
\definecolor{verdeoscuro}{RGB}{0,100,0}
\definecolor{rojoclaro}{RGB}{255,230,230}

% Configuración de cajas
\tcbuselibrary{theorems,skins,breakable}

\newtcolorbox{datosbox}{
    colback=azulclaro,
    colframe=azuloscuro,
    fonttitle=\bfseries,
    title=Datos del Problema,
    sharp corners,
    boxrule=1pt
}

\newtcolorbox{solucionbox}{
    colback=white,
    colframe=verdeoscuro,
    fonttitle=\bfseries,
    title=Desarrollo de la Solución,
    sharp corners,
    boxrule=1pt,
    breakable
}

\newtcolorbox{resultadobox}{
    colback=rojoclaro,
    colframe=red!70!black,
    fonttitle=\bfseries,
    title=Resultado Final,
    sharp corners,
    boxrule=2pt
}

% Título y autor
\title{\textbf{Solución del Ejercicio 3.17} \\
\large Movimiento de Proyectiles: Lanzamiento a 60°}
\author{Física Universitaria - Sears y Zemansky \\ Capítulo 3: Movimiento en Dos o Tres Dimensiones}
\date{\today}

\begin{document}

\maketitle

\section{Enunciado del Problema}

\begin{wrapfigure}[10]{r}{.7\textwidth}  % Ajusta el ancho según necesites
	\centering
	\vspace{-\baselineskip}
	%\begin{center}
\begin{tikzpicture}[scale=.74]
    % Escala: 1 unidad = 50 m horizontal, 1 unidad = 50 m vertical
    % R = 565.7 m → 11.314 unidades
    % h_max = 245 m → 4.9 unidades
    % x_max = 282.85 m → 5.657 unidades

    % Suelo
    \draw[fill=brown!20, thick] (-0.5,0) rectangle (12,-0.6);
    \draw[pattern=north east lines, pattern color=brown] (-0.5,-0.6) rectangle (12,-0.4);

    % Punto de lanzamiento
    \filldraw[blue!70!black] (0,0) circle (0.1);
    \node[below left] at (0,0.5) {O};

    % Vector velocidad inicial (proporcional pero escalado para visualización)
    \draw[-{Latex[length=3mm]},blue!70!black, ultra thick] (0,0) -- (1.2,2.078);
    \node[blue!70!black, above left] at (0.6,1.3) {$\vec{v}_0 = 80$ m/s};

    % Componentes de velocidad
    \draw[-{Latex[length=2mm]},red, ultra thick, dashed] (0,0) -- (1.2,0);
    \node[red, below] at (1.3,0.1) {$v_{0x}=40$ m/s};
    \draw[-{Latex[length=2mm]},red, very thick, dashed] (1.2,0) -- (1.2,2.078);
    \node[red, right] at (1.2,1) {$v_{0y}=69.3$ m/s};

    % Ángulo
    \draw[thick] (0.5,0) arc (0:60:0.5);
    \node at (0.8,0.25) {$60°$};

    % Trayectoria parabólica con escala consistente
    % Ecuación en metros: y = tan(60°)*x - (g/(2*v0^2*cos^2(60°)))*x^2
    % y = 1.732*x - 0.0030625*x^2
    % Coeficiente: b = 9.8/(2*80^2*0.5^2) = 9.8/3200 = 0.0030625
    % Con escala 1 unidad = 50 m, donde x_graf = x_m/50 y y_graf = y_m/50:
    % y_graf = 1.732*x_graf - 0.0030625*2500*x_graf^2 = 1.732*x_graf - 0.153125*x_graf^2
    \draw[red!70!black, very thick, -{Latex[length=3mm]}]
        plot[domain=0:11.314, samples=100, smooth] (\x, {1.732*\x - 0.153125*\x*\x});

    % Punto de altura máxima (x_max = 5.657, y_max = 4.9)
    \filldraw[violet!80!black] (5.657,4.9) circle (0.1);
    \node[violet!80!black, above right] at (5.657,4.9) {$h_{\text{máx}}=245$ m};

    % Línea vertical de altura máxima (punteada)
    \draw[{Latex[length=2mm]}-{Latex[length=2mm]}, violet!80!black,  thick] (5.657,4.9) -- (5.657,0);

    % Flecha indicando altura máxima
    %\draw[{Latex[length=2mm]}-{Latex[length=2mm]}, thick] (-0.8,0) -- (-0.8,4.9);
    %\node[left, align=center] at (-0.8,2.45) {\small $h_{\text{máx}}$\\\small $245$ m};

    % Punto final
    \filldraw[green!50!black] (11.314,0) circle (0.08);

    % Alcance horizontal
    \draw[{Latex[length=2mm]}-{Latex[length=2mm]}, thick] (0,-0.8) -- (11.314,-0.8);
    \node[below, align=center] at (5.657,-0.8) {$R = 566$ m};

    % Etiquetas de distancia
    \node[below] at (5.657,-1.5) {\small Escala: 1 cm $\approx$ 50 m};

\end{tikzpicture}
%\end{center}
\vspace{-\baselineskip} % Reduce espacio después
\end{wrapfigure}
Se dispara un proyectil desde el nivel del suelo con una velocidad inicial de 80.0 m/s a 60.0° por encima de la horizontal. Calcule:

\begin{enumerate}
	\item[a)] Las componentes $x$ e $y$ de la velocidad inicial del proyectil.
	\item[b)] El tiempo que tarda el proyectil en alcanzar su altura máxima.
\end{enumerate}

\begin{enumerate}
	\item[c)] La altura máxima que alcanza el proyectil.
	\item[d)] El alcance horizontal del proyectil.
	\item[e)] La velocidad y la aceleración del proyectil cuando alcanza su altura máxima.
\end{enumerate}

\section{Datos del Problema}

\begin{datosbox}
\begin{itemize}
    \item \textbf{Velocidad inicial:} $v_0 = 80.0$ m/s
    \item \textbf{Ángulo de lanzamiento:} $\theta = 60.0°$
    \item \textbf{Aceleración gravitacional:} $g = 9.8$ m/s$^2$
    \item \textbf{Altura inicial:} $y_0 = 0$ (nivel del suelo)
    \item \textbf{Resistencia del aire:} despreciable
\end{itemize}
\end{datosbox}

\section{Marco Teórico}

Para un proyectil lanzado con velocidad inicial $v_0$ a un ángulo $\theta$ desde el nivel del suelo:

\textbf{Componentes de velocidad inicial:}
\begin{align}
    v_{0x} &= v_0 \cos\theta \\
    v_{0y} &= v_0 \sin\theta
\end{align}

\textbf{Tiempo de vuelo hasta altura máxima:}
\begin{equation}
    t_{\text{máx}} = \frac{v_{0y}}{g}
\end{equation}

\textbf{Altura máxima:}
\begin{equation}
    h_{\text{máx}} = \frac{v_{0y}^2}{2g}
\end{equation}

\textbf{Alcance horizontal:}
\begin{equation}
    R = \frac{v_0^2 \sin(2\theta)}{g}
\end{equation}

\section{Desarrollo de la Solución}

\begin{solucionbox}

\subsection*{Parte a) Componentes de velocidad inicial}

Calculamos las componentes horizontal y vertical:

\textbf{Componente horizontal:}
\begin{align}
    v_{0x} &= v_0 \cos\theta \\
    v_{0x} &= 80.0 \cos(60.0°) \\
    v_{0x} &= 80.0 \times 0.5 \\
    v_{0x} &= 40.0 \text{ m/s}
\end{align}

\textbf{Componente vertical:}
\begin{align}
    v_{0y} &= v_0 \sin\theta \\
    v_{0y} &= 80.0 \sin(60.0°) \\
    v_{0y} &= 80.0 \times 0.8660 \\
    v_{0y} &= 69.3 \text{ m/s}
\end{align}

\subsection*{Parte b) Tiempo hasta altura máxima}

En el punto más alto, la componente vertical de velocidad es cero. Usando:
\begin{equation}
    v_y = v_{0y} - gt
\end{equation}

En el punto máximo, $v_y = 0$:
\begin{align}
    0 &= v_{0y} - gt_{\text{máx}} \\
    t_{\text{máx}} &= \frac{v_{0y}}{g} \\
    t_{\text{máx}} &= \frac{69.3}{9.8} \\
    t_{\text{máx}} &= 7.07 \text{ s}
\end{align}

\subsection*{Parte c) Altura máxima}

Usando la ecuación de posición vertical:
\begin{equation}
    y = v_{0y}t - \frac{1}{2}gt^2
\end{equation}

En $t = t_{\text{máx}}$:
\begin{align}
    h_{\text{máx}} &= v_{0y}t_{\text{máx}} - \frac{1}{2}g t_{\text{máx}}^2 \\
    h_{\text{máx}} &= 69.3 \times 7.07 - \frac{1}{2} \times 9.8 \times (7.07)^2 \\
    h_{\text{máx}} &= 489.95 - 245.0 \\
    h_{\text{máx}} &= 245 \text{ m}
\end{align}

\textbf{Forma alternativa:}
\begin{equation}
    h_{\text{máx}} = \frac{v_{0y}^2}{2g} = \frac{(69.3)^2}{2 \times 9.8} = \frac{4802.49}{19.6} = 245 \text{ m}
\end{equation}

\subsection*{Parte d) Alcance horizontal}

El tiempo total de vuelo es el doble del tiempo hasta la altura máxima:
\begin{equation}
    t_{\text{total}} = 2t_{\text{máx}} = 2 \times 7.07 = 14.14 \text{ s}
\end{equation}

El alcance horizontal es:
\begin{align}
    R &= v_{0x} \times t_{\text{total}} \\
    R &= 40.0 \times 14.14 \\
    R &= 565.6 \text{ m}
\end{align}

\textbf{Forma alternativa usando la fórmula directa:}
\begin{align}
    R &= \frac{v_0^2 \sin(2\theta)}{g} \\
    R &= \frac{(80.0)^2 \sin(2 \times 60°)}{9.8} \\
    R &= \frac{6400 \times \sin(120°)}{9.8} \\
    R &= \frac{6400 \times 0.8660}{9.8} \\
    R &= \frac{5542.4}{9.8} \\
    R &= 565.6 \text{ m}
\end{align}

\subsection*{Parte e) Velocidad y aceleración en la altura máxima}

\textbf{Velocidad en la altura máxima:}

En el punto más alto, la componente vertical de velocidad es cero:
\begin{equation}
    v_y = 0
\end{equation}

La componente horizontal permanece constante:
\begin{equation}
    v_x = v_{0x} = 40.0 \text{ m/s}
\end{equation}

Por lo tanto, la velocidad en el punto más alto es:
\begin{equation}
    \vec{v}_{\text{máx}} = (40.0 \text{ m/s}, 0)
\end{equation}

La magnitud es:
\begin{equation}
    |\vec{v}_{\text{máx}}| = 40.0 \text{ m/s}
\end{equation}

\textbf{Aceleración en la altura máxima:}

La aceleración es constante durante todo el vuelo y siempre apunta hacia abajo:
\begin{equation}
    \vec{a} = (0, -g) = (0, -9.8 \text{ m/s}^2)
\end{equation}

La magnitud es:
\begin{equation}
    |\vec{a}| = 9.8 \text{ m/s}^2 \text{ (hacia abajo)}
\end{equation}

\end{solucionbox}

\section{Resultados Finales}

\begin{resultadobox}
\textbf{a) Componentes de velocidad inicial:}
\begin{align*}
    \boxed{v_{0x} = 40.0 \text{ m/s}} \\
    \boxed{v_{0y} = 69.3 \text{ m/s}}
\end{align*}

\vspace{0.3cm}

\textbf{b) Tiempo hasta altura máxima:}
\begin{equation*}
    \boxed{t_{\text{máx}} = 7.07 \text{ s}}
\end{equation*}

\vspace{0.3cm}

\textbf{c) Altura máxima:}
\begin{equation*}
    \boxed{h_{\text{máx}} = 245 \text{ m}}
\end{equation*}

\vspace{0.3cm}

\textbf{d) Alcance horizontal:}
\begin{equation*}
    \boxed{R = 566 \text{ m}}
\end{equation*}

\vspace{0.3cm}

\textbf{e) En la altura máxima:}
\begin{itemize}
    \item Velocidad: $\boxed{\vec{v} = 40.0 \text{ m/s (horizontal)}}$
    \item Aceleración: $\boxed{\vec{a} = 9.8 \text{ m/s}^2 \text{ (hacia abajo)}}$
\end{itemize}
\end{resultadobox}

\section{Verificación}

Podemos verificar nuestros resultados usando las relaciones conocidas:

\subsection*{Verificación de altura máxima}

Usando la fórmula alternativa:
\begin{align*}
    h_{\text{máx}} &= \frac{v_{0y}^2}{2g} = \frac{(69.3)^2}{2 \times 9.8} \\
    &= \frac{4802.49}{19.6} = 245 \text{ m} \quad \checkmark
\end{align*}

\subsection*{Verificación del alcance}

Método 1 (cinemática):
\begin{equation*}
    R = v_{0x} \times t_{\text{total}} = 40.0 \times 14.14 = 565.6 \text{ m}
\end{equation*}

Método 2 (fórmula directa):
\begin{equation*}
    R = \frac{v_0^2 \sin(2\theta)}{g} = \frac{6400 \times 0.8660}{9.8} = 565.6 \text{ m} \quad \checkmark
\end{equation*}

\subsection*{Simetría del movimiento}

El tiempo de subida es igual al tiempo de bajada:
\begin{equation*}
    t_{\text{subida}} = t_{\text{bajada}} = 7.07 \text{ s}
\end{equation*}

\begin{equation*}
    t_{\text{total}} = 2 \times t_{\text{máx}} = 14.14 \text{ s} \quad \checkmark
\end{equation*}

\section{Análisis Físico}

\subsection{Características del movimiento}

\begin{enumerate}
    \item \textbf{Movimiento horizontal:} Es uniforme (MRU) con velocidad constante $v_x = 40.0$ m/s.

    \item \textbf{Movimiento vertical:} Es uniformemente acelerado (MRUA) con aceleración $a_y = -g = -9.8$ m/s$^2$.

    \item \textbf{En la altura máxima:}
    \begin{itemize}
        \item La velocidad vertical es cero
        \item La velocidad horizontal permanece constante
        \item La aceleración sigue siendo $g$ hacia abajo
    \end{itemize}

    \item \textbf{Simetría:} El tiempo de subida es igual al tiempo de bajada.

    \item \textbf{Alcance máximo:} Para un ángulo de 60°, el alcance no es máximo. El ángulo óptimo para máximo alcance es 45°.
\end{enumerate}

\subsection{Comparación con otros ángulos}

Si se lanzara el proyectil con la misma velocidad inicial pero a 45°:
\begin{equation*}
    R_{45°} = \frac{v_0^2}{g} = \frac{6400}{9.8} = 653 \text{ m} > 566 \text{ m}
\end{equation*}

El ángulo de 60° favorece la altura máxima sobre el alcance horizontal.

\section{Gráficas del Movimiento}

\subsection{Trayectoria en el plano $xy$}

\begin{center}
\begin{tikzpicture}
\begin{axis}[
    width=12cm,
    height=8cm,
    xlabel={Distancia horizontal $x$ (m)},
    ylabel={Altura $y$ (m)},
    xmin=0, xmax=600,
    ymin=0, ymax=260,
    grid=major,
    grid style={dashed,gray!30},
    legend pos=north east
]

% Trayectoria
\addplot[
    red!70!black,
    very thick,
    domain=0:565.7,
    samples=100,
    smooth
] {1.732*x - 0.001225*x^2};
\addlegendentry{$y(x)$}

% Punto de altura máxima
\addplot[
    mark=*,
    mark size=3pt,
    green!50!black,
    only marks
] coordinates {(282.85, 245)};
\addlegendentry{Altura máxima}

\end{axis}
\end{tikzpicture}
\end{center}

\subsection{Posición horizontal vs. tiempo}

\begin{center}
\begin{tikzpicture}
\begin{axis}[
    width=12cm,
    height=6cm,
    xlabel={Tiempo $t$ (s)},
    ylabel={Posición $x$ (m)},
    xmin=0, xmax=15,
    ymin=0, ymax=600,
    grid=major,
    grid style={dashed,gray!30},
    legend pos=north west
]

% x(t) = v0x * t
\addplot[
    blue!70!black,
    very thick,
    domain=0:14.14,
    samples=50
] {40*x};
\addlegendentry{$x(t) = v_{0x}t$}

\end{axis}
\end{tikzpicture}
\end{center}

\subsection{Posición vertical vs. tiempo}

\begin{center}
\begin{tikzpicture}
\begin{axis}[
    width=12cm,
    height=8cm,
    xlabel={Tiempo $t$ (s)},
    ylabel={Altura $y$ (m)},
    xmin=0, xmax=15,
    ymin=0, ymax=260,
    grid=major,
    grid style={dashed,gray!30},
    legend pos=north east
]

% y(t) = v0y*t - 0.5*g*t^2
\addplot[
    red!70!black,
    very thick,
    domain=0:14.14,
    samples=100
] {69.3*x - 4.9*x^2};
\addlegendentry{$y(t) = v_{0y}t - \frac{1}{2}gt^2$}

% Altura máxima
\addplot[
    mark=*,
    mark size=3pt,
    green!50!black,
    only marks
] coordinates {(7.07, 245)};
\addlegendentry{$h_{\text{máx}}$ en $t = 7.07$ s}

\end{axis}
\end{tikzpicture}
\end{center}

\subsection{Velocidad horizontal vs. tiempo}

\begin{center}
\begin{tikzpicture}
\begin{axis}[
    width=12cm,
    height=6cm,
    xlabel={Tiempo $t$ (s)},
    ylabel={Velocidad $v_x$ (m/s)},
    xmin=0, xmax=15,
    ymin=0, ymax=50,
    grid=major,
    grid style={dashed,gray!30},
    legend pos=south east
]

% vx(t) = constante
\addplot[
    blue!70!black,
    very thick,
    domain=0:14.14,
    samples=10
] {40};
\addlegendentry{$v_x(t) = v_{0x}$ (constante)}

\end{axis}
\end{tikzpicture}
\end{center}

\subsection{Velocidad vertical vs. tiempo}

\begin{center}
\begin{tikzpicture}
\begin{axis}[
    width=12cm,
    height=8cm,
    xlabel={Tiempo $t$ (s)},
    ylabel={Velocidad $v_y$ (m/s)},
    xmin=0, xmax=15,
    ymin=-80, ymax=80,
    grid=major,
    grid style={dashed,gray!30},
    legend pos=north east
]

% vy(t) = v0y - g*t
\addplot[
    red!70!black,
    very thick,
    domain=0:14.14,
    samples=50
] {69.3 - 9.8*x};
\addlegendentry{$v_y(t) = v_{0y} - gt$}

% Punto donde vy = 0
\addplot[
    mark=*,
    mark size=3pt,
    green!50!black,
    only marks
] coordinates {(7.07, 0)};
\addlegendentry{$v_y = 0$ en $t_{\text{máx}}$}

% Línea de referencia y = 0
\addplot[
    dashed,
    black,
    domain=0:14.14
] {0};

\end{axis}
\end{tikzpicture}
\end{center}

\section{Observaciones Importantes}

\begin{enumerate}
    \item La aceleración es constante durante todo el vuelo y vale $9.8$ m/s$^2$ hacia abajo, incluso en la altura máxima.

    \item La velocidad horizontal permanece constante durante todo el vuelo.

    \item La velocidad vertical cambia linealmente con el tiempo, desde $+69.3$ m/s (hacia arriba) hasta $-69.3$ m/s (hacia abajo).

    \item El movimiento es completamente simétrico: lo que sube debe bajar de la misma manera.

    \item En la altura máxima, el proyectil no está "en reposo", solo tiene velocidad horizontal.
\end{enumerate}

\end{document}
