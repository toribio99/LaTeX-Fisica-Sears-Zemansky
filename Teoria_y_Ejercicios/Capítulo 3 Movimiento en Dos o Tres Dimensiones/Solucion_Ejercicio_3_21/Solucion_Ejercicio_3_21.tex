\documentclass[11pt,a4paper]{article}

% Paquetes necesarios
\usepackage[utf8]{inputenc}
\usepackage[T1]{fontenc}
\usepackage[spanish]{babel}
\usepackage[margin=2.5cm]{geometry}
\usepackage{amsmath}
\usepackage{amssymb}
\usepackage{xcolor}
\usepackage{tcolorbox}
\usepackage{graphicx}
\usepackage{tikz}
\usepackage{pgfplots}
\pgfplotsset{compat=1.18}
\usetikzlibrary{arrows.meta,patterns,decorations.pathmorphing,calc}
\usepackage{wrapfig}
\usepackage[export]{adjustbox} % (opcional) claves extra para \includegraphics
\usepackage{xparse}

% Definición de colores
\definecolor{azuloscuro}{RGB}{0,51,102}
\definecolor{azulclaro}{RGB}{230,240,250}
\definecolor{verdeoscuro}{RGB}{0,100,0}
\definecolor{rojoclaro}{RGB}{255,230,230}

% Configuración de cajas
\tcbuselibrary{theorems,skins,breakable}

\newtcolorbox{datosbox}{
    colback=azulclaro,
    colframe=azuloscuro,
    fonttitle=\bfseries,
    title=Datos del Problema,
    sharp corners,
    boxrule=1pt
}

\newtcolorbox{solucionbox}{
    colback=white,
    colframe=verdeoscuro,
    fonttitle=\bfseries,
    title=Desarrollo de la Solución,
    sharp corners,
    boxrule=1pt,
    breakable
}

\newtcolorbox{resultadobox}{
    colback=rojoclaro,
    colframe=red!70!black,
    fonttitle=\bfseries,
    title=Resultado Final,
    sharp corners,
    boxrule=2pt
}

% Título y autor
\title{\textbf{Solución del Ejercicio 3.21} \\
\large Gane el Premio - Moneda en el Platito}
\author{Física Universitaria - Sears y Zemansky \\ Capítulo 3: Movimiento en Dos o Tres Dimensiones}
\date{\today}

\begin{document}

\maketitle

\section{Enunciado del Problema}

\begin{wrapfigure}[10]{r}{.6\textwidth}  % Ajusta el ancho según necesites
	\centering
	\vspace{-\baselineskip}
	%\begin{center}
\begin{tikzpicture}[scale=2]
    % Suelo
    \draw[fill=brown!20, thick] (-0.5,0) rectangle (3,-0.35);
    \draw[pattern=north east lines, pattern color=brown] (-0.5,-0.35) rectangle (3,-0.45);

    % Persona lanzando
    \draw[fill=blue!20, thick] (0,0) circle (0.08);
    \node[below, font=\small] at (0,-0.06) {Lanzador};

    % Plataforma con platito
    \draw[fill=gray!30, thick] (2.1,1.53) rectangle (2.5,1.63);
    \draw[fill=yellow!40, thick] (2.25,1.63) circle (0.08);
    \node[above right, font=\scriptsize] at (2,1.7) {Platito};
    \node[right, font=\scriptsize] at (2.5,1.58) {Repisa};

    % Vector velocidad inicial
    \draw[-{Latex[length=2.5mm]},blue!70!black, ultra thick] (0,0) -- (0.4,0.693);
    \node[blue!70!black, above left, font=\small] at (-0.35,0) {$\vec{v}_0=6.4$ m/s};

    % Componentes de velocidad
    \draw[-{Latex[length=2mm]},red, thick, dashed] (0,0) -- (0.4,0);
    \node[red, below, font=\scriptsize] at (0.7,0.06) {$v_{0x}=3.2$ m/s};

    \draw[-{Latex[length=2mm]},red, thick, dashed] (0.4,0) -- (0.4,0.693);
    \node[red, right, font=\scriptsize] at (0.4,0.35) {$v_{0y}=5.54$ m/s};

    % Ángulo
    \draw[thick] (0.15,0) arc (0:60:0.15);
    \node[font=\scriptsize] at (0.22,0.14) {$60°$};

    % Trayectoria parabólica correcta
    % Ecuación: y = tan(60°)*x - (g/(2*v₀²*cos²(60°)))*x²
    % y = 1.732*x - 0.477*x²
    % Verificación: en x=2.1, y = 1.732(2.1) - 0.477(2.1)² = 3.637 - 2.102 = 1.535 m ✓
    \draw[red!70!black, very thick, -{Latex[length=2.5mm]}]
        plot[domain=0:2.1, samples=100, smooth] (\x, {1.732*\x - 0.477*\x*\x});

    % Punto de impacto en el platito
    \filldraw[orange!80!black] (2.1,1.53) circle (0.06);
    \node[orange!80!black, left, font=\scriptsize] at (3.05,1.4) {$(2.1, 1.53)$ m};

    % Línea punteada vertical
    \draw[orange!80!black, dotted, thick] (2.1,1.53) -- (2.1,0);

    % Altura de la repisa
    \draw[{Latex[length=2mm]}-{Latex[length=2mm]}, thick] (-0.3,0) -- (-0.3,1.53);
    \node[left, font=\small] at (-0.3,1.5) {$h=?$};

    % Distancia horizontal
    \draw[{Latex[length=2mm]}-{Latex[length=2mm]}, thick] (0,-0.6) -- (2.1,-0.6);
    \node[below, font=\small] at (1.05,-0.6) {$x = 2.1$ m};

\end{tikzpicture}
%\end{center}
\vspace{-\baselineskip} % Reduce espacio después
\end{wrapfigure}
Gane el premio. En una feria, se gana una jirafa de peluche lanzando una moneda a un platito, el cual está sobre una repisa más arriba del punto en que la moneda sale de la mano y a una distancia horizontal de 2.1 m desde ese punto. Si lanza la moneda con velocidad de 6.4 m/s, a un ángulo de 60° sobre la horizontal, la moneda caerá en el platito. Ignore la resistencia del aire.

\begin{enumerate}
	\item[a)] ¿A qué altura está la repisa sobre el punto donde se lanza la moneda?
	\item[b)] ¿Qué componente vertical tiene la velocidad de la moneda justo antes de caer en el platito?
\end{enumerate}

\section{Datos del Problema}

\begin{datosbox}
\begin{itemize}
    \item \textbf{Velocidad inicial:} $v_0 = 6.4$ m/s
    \item \textbf{Ángulo de lanzamiento:} $\theta_0 = 60°$
    \item \textbf{Distancia horizontal al platito:} $x = 2.1$ m
    \item \textbf{Aceleración gravitacional:} $g = 9.8$ m/s$^2$
    \item \textbf{Posición inicial:} $x_0 = 0$, $y_0 = 0$ (nivel de lanzamiento)
    \item \textbf{Resistencia del aire:} despreciable
\end{itemize}
\end{datosbox}

\section{Marco Teórico}

Para un proyectil lanzado desde el origen con velocidad inicial $v_0$ a un ángulo $\theta_0$:

\textbf{Componentes de velocidad inicial:}
\begin{align}
    v_{0x} &= v_0 \cos\theta_0 \\
    v_{0y} &= v_0 \sin\theta_0
\end{align}

\textbf{Ecuaciones de posición:}
\begin{align}
    x(t) &= v_{0x}t \\
    y(t) &= v_{0y}t - \frac{1}{2}gt^2
\end{align}

\textbf{Ecuaciones de velocidad:}
\begin{align}
    v_x(t) &= v_{0x} \quad \text{(constante)} \\
    v_y(t) &= v_{0y} - gt
\end{align}

\section{Desarrollo de la Solución}

\begin{solucionbox}

\subsection*{Paso 1: Componentes de la velocidad inicial}

Calculamos las componentes horizontal y vertical de la velocidad inicial:

\begin{align}
    v_{0x} &= v_0 \cos\theta_0 = (6.4 \text{ m/s}) \cos(60°) \\
    v_{0x} &= (6.4 \text{ m/s})(0.5) = 3.2 \text{ m/s}
\end{align}

\begin{align}
    v_{0y} &= v_0 \sin\theta_0 = (6.4 \text{ m/s}) \sin(60°) \\
    v_{0y} &= (6.4 \text{ m/s})(0.866) = 5.542 \text{ m/s}
\end{align}

\subsection*{Paso 2: Tiempo de vuelo hasta el platito}

La moneda recorre una distancia horizontal de $x = 2.1$ m hasta llegar al platito. Como el movimiento horizontal tiene velocidad constante, el tiempo se obtiene de:

\begin{equation}
    x = v_{0x}t
\end{equation}

Despejando $t$:

\begin{align}
    t &= \frac{x}{v_{0x}} = \frac{2.1 \text{ m}}{3.2 \text{ m/s}} \\
    t &= 0.65625 \text{ s}
\end{align}

\subsection*{Parte a): Altura de la repisa}

Usamos la ecuación de posición vertical para calcular la altura $h$ (que es $y$) en el instante en que la moneda llega al platito:

\begin{equation}
    y = v_{0y}t - \frac{1}{2}gt^2
\end{equation}

Sustituyendo los valores:

\begin{align}
    h &= (5.542 \text{ m/s})(0.65625 \text{ s}) - \frac{1}{2}(9.8 \text{ m/s}^2)(0.65625 \text{ s})^2 \\
    h &= 3.637 \text{ m} - \frac{1}{2}(9.8 \text{ m/s}^2)(0.4307 \text{ s}^2) \\
    h &= 3.637 \text{ m} - 2.110 \text{ m} \\
    h &= 1.527 \text{ m} \approx 1.53 \text{ m}
\end{align}

\subsection*{Parte b): Componente vertical de la velocidad}

La componente vertical de la velocidad en el instante en que la moneda llega al platito se calcula con:

\begin{equation}
    v_y = v_{0y} - gt
\end{equation}

Sustituyendo los valores:

\begin{align}
    v_y &= 5.542 \text{ m/s} - (9.8 \text{ m/s}^2)(0.65625 \text{ s}) \\
    v_y &= 5.542 \text{ m/s} - 6.431 \text{ m/s} \\
    v_y &= -0.889 \text{ m/s} \approx -0.89 \text{ m/s}
\end{align}

El signo negativo indica que la componente vertical de la velocidad apunta hacia abajo en el momento en que la moneda cae en el platito.

\subsection*{Verificación de la ecuación de la trayectoria}

La trayectoria parabólica tiene la forma:

\begin{equation}
    y = (\tan\theta_0)x - \frac{g}{2v_0^2\cos^2\theta_0}x^2
\end{equation}

Sustituyendo valores:

\begin{align}
    y &= \tan(60°)x - \frac{9.8}{2(6.4)^2\cos^2(60°)}x^2 \\
    y &= 1.732x - \frac{9.8}{2(40.96)(0.25)}x^2 \\
    y &= 1.732x - \frac{9.8}{20.48}x^2 \\
    y &= 1.732x - 0.478x^2
\end{align}

Verificación en $x = 2.1$ m:
\begin{equation}
    y = 1.732(2.1) - 0.478(2.1)^2 = 3.637 - 2.110 = 1.527 \text{ m} \checkmark
\end{equation}

\end{solucionbox}

\section{Resultados Finales}

\begin{resultadobox}

\textbf{Parte a) Altura de la repisa:}
\begin{equation}
    \boxed{h = 1.53 \text{ m}}
\end{equation}

La repisa está a una altura de \textbf{1.53 metros} sobre el punto desde donde se lanza la moneda.

\vspace{0.3cm}

\textbf{Parte b) Componente vertical de la velocidad:}
\begin{equation}
    \boxed{v_y = -0.89 \text{ m/s}}
\end{equation}

La componente vertical de la velocidad de la moneda justo antes de caer en el platito es \textbf{$-0.89$ m/s} (hacia abajo).

\vspace{0.3cm}

\textbf{Información adicional:}
\begin{itemize}
    \item Tiempo de vuelo: $t = 0.656$ s
    \item Componente horizontal de velocidad (constante): $v_x = 3.2$ m/s
    \item Velocidad total al llegar al platito: $v = \sqrt{v_x^2 + v_y^2} = \sqrt{(3.2)^2 + (-0.89)^2} = 3.32$ m/s
    \item Ángulo de la velocidad respecto a la horizontal: $\alpha = \arctan\left(\frac{v_y}{v_x}\right) = \arctan\left(\frac{-0.89}{3.2}\right) = -15.5°$ (bajo la horizontal)
\end{itemize}

\end{resultadobox}

\section{Análisis y Conclusión}

Este problema ilustra un caso típico de movimiento de proyectiles donde conocemos el punto de impacto y la velocidad inicial. Las conclusiones principales son:

\begin{enumerate}
    \item La moneda sigue una trayectoria parabólica descrita por $y = 1.732x - 0.478x^2$

    \item A pesar de que la moneda se lanza con un ángulo de 60° (muy empinado), cuando llega al platito su trayectoria está descendiendo con un ángulo de solo 15.5° bajo la horizontal.

    \item La componente vertical de la velocidad ha cambiado de $+5.54$ m/s (hacia arriba) a $-0.89$ m/s (hacia abajo), mientras que la componente horizontal se mantiene constante en $3.2$ m/s durante todo el vuelo.

    \item La altura de 1.53 m es razonable para una repisa en un juego de feria, y el tiempo de vuelo de aproximadamente 0.66 segundos hace que el juego sea desafiante pero factible.
\end{enumerate}

\end{document}
