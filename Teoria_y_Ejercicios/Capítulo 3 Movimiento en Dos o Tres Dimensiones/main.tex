% ================================================================================
% DOCUMENTO MAESTRO - CAPÍTULO 3: MOVIMIENTO EN DOS O TRES DIMENSIONES
% Física Universitaria - Sears y Zemansky
% ================================================================================
%
% DESCRIPCIÓN:
%   Este es el archivo principal que compila todos los ejercicios resueltos
%   del Capítulo 3 del libro "Física Universitaria" de Sears y Zemansky.
%
% ESTRUCTURA DEL DOCUMENTO:
%   1. Carga del preámbulo (configuraciones, paquetes, comandos)
%   2. Portada del capítulo
%   3. Tabla de contenidos
%   4. Inclusión de cada ejercicio individual
%
% MODO DE USO:
%   - Para compilar TODO el documento:
%       Comenta o elimina la línea \includeonly
%
%   - Para compilar SOLO algunos ejercicios:
%       Descomenta la línea \includeonly y lista los ejercicios deseados
%       Ejemplo: \includeonly{Solucion_Ejercicio_3_10/Solucion_Ejercicio_3_10,
%                             Solucion_Ejercicio_3_15/Solucion_Ejercicio_3_15}
%
% COMPILACIÓN:
%   Se recomienda compilar DOS veces para actualizar referencias cruzadas
%   y la tabla de contenidos:
%       pdflatex main.tex
%       pdflatex main.tex
%
% AUTOR: Generado automáticamente
% FECHA: \today
% VERSIÓN: 1.0
%
% ================================================================================

% --------------------------------------------------------------------------------
% CLASE DEL DOCUMENTO Y PAQUETES BÁSICOS
% --------------------------------------------------------------------------------
\documentclass[11pt,a4paper]{article}
\usepackage[utf8]{inputenc}
\usepackage[T1]{fontenc}
\usepackage[spanish]{babel}
\usepackage[margin=2.5cm]{geometry}
\usepackage{hyperref}
\usepackage{pdfpages}  % Para incluir PDFs externos

\hypersetup{
    colorlinks=true,
    linkcolor=blue,
    urlcolor=cyan,
    pdftitle={Capítulo 3: Movimiento en Dos o Tres Dimensiones},
    pdfauthor={Física Universitaria - Sears y Zemansky},
}

% --------------------------------------------------------------------------------
% NOTA IMPORTANTE SOBRE LA ESTRUCTURA
% --------------------------------------------------------------------------------
% Los ejercicios individuales son documentos LaTeX completos e independientes.
% Este archivo principal utiliza el paquete 'pdfpages' para incluir los PDFs
% ya compilados de cada ejercicio.
%
% REQUISITO PREVIO:
%   Antes de compilar este documento, asegúrate de que cada ejercicio individual
%   haya sido compilado y exista su archivo PDF correspondiente.
%
% VENTAJAS DE ESTE ENFOQUE:
%   - Cada ejercicio puede compilarse y editarse independientemente
%   - Mayor flexibilidad para modificar ejercicios individuales
%   - Compilación más rápida del documento maestro
%   - Fácil reorganización del orden de ejercicios
%
% PARA COMPILAR SOLO ALGUNOS EJERCICIOS:
%   Simplemente comenta las líneas \includepdf de los ejercicios que no desees incluir

% --------------------------------------------------------------------------------
% INICIO DEL DOCUMENTO
% --------------------------------------------------------------------------------
\begin{document}

% --------------------------------------------------------------------------------
% PORTADA DEL CAPÍTULO
% --------------------------------------------------------------------------------
\begin{titlepage}
    \centering
    \vspace*{2cm}

    % Título principal
    {\Huge\bfseries Física Universitaria\par}
    \vspace{0.5cm}
    {\LARGE Sears y Zemansky\par}
    \vspace{2cm}

    % Título del capítulo
    {\huge\bfseries Capítulo 3\par}
    \vspace{0.5cm}
    {\LARGE Movimiento en Dos o Tres Dimensiones\par}
    \vspace{1cm}
    {\Large Ejercicios Resueltos\par}
    \vspace{2cm}

    % Información adicional
    {\large
    \textbf{Contenido:}\\[0.3cm]
    Ejercicios del 3.9 al 3.27\\[0.2cm]
    \textit{Cinemática Vectorial}\\
    \textit{Movimiento de Proyectiles}\\
    \textit{Movimiento Circular}\\
    \textit{Velocidad Relativa}
    \par}

    \vfill

    % Fecha
    {\large \today\par}
\end{titlepage}

% --------------------------------------------------------------------------------
% ÍNDICE / TABLA DE CONTENIDOS
% --------------------------------------------------------------------------------
% Nota: Como estamos incluyendo PDFs externos completos, no se genera una tabla
% de contenidos automática. Cada ejercicio mantiene su propia estructura interna.

% --------------------------------------------------------------------------------
% INCLUSIÓN DE EJERCICIOS INDIVIDUALES
% --------------------------------------------------------------------------------
% Cada ejercicio se incluye como PDF completo usando el comando \includepdf
%
% SINTAXIS:
%   \includepdf[pages=-]{ruta/archivo.pdf}
%   - pages=-  : incluye todas las páginas del PDF
%   - pages=1-3: incluye solo las páginas 1 a 3
%
% PARA EXCLUIR UN EJERCICIO:
%   Simplemente comenta la línea correspondiente con %
%
% REQUISITO:
%   Cada ejercicio debe haber sido compilado previamente para generar su PDF
% ================================================================================

% Ejercicio 3.9 - Libro cayendo de una mesa
\includepdf[pages=-]{Solucion_Ejercicio_3_9/Solucion_Ejercicio_3_9.pdf}

% Ejercicio 3.10 - Componentes de vectores
\includepdf[pages=-]{Solucion_Ejercicio_3_10/Solucion_Ejercicio_3_10.pdf}

% Ejercicio 3.11 - Velocidad y aceleración vectorial
\includepdf[pages=-]{Solucion_Ejercicio_3_11/Solucion_Ejercicio_3_11.pdf}

% Ejercicio 3.12 - Trayectoria de una partícula
\includepdf[pages=-]{Solucion_Ejercicio_3_12/Solucion_Ejercicio_3_12.pdf}

% Ejercicio 3.13 - Movimiento parabólico básico
\includepdf[pages=-]{Solucion_Ejercicio_3_13/Solucion_Ejercicio_3_13.pdf}

% Ejercicio 3.14 - Alcance de proyectil
\includepdf[pages=-]{Solucion_Ejercicio_3_14/Solucion_Ejercicio_3_14.pdf}

% Ejercicio 3.15 - Altura máxima de proyectil
\includepdf[pages=-]{Solucion_Ejercicio_3_15/Solucion_Ejercicio_3_15.pdf}

% Ejercicio 3.16 - Tiempo de vuelo
\includepdf[pages=-]{Solucion_Ejercicio_3_16/Solucion_Ejercicio_3_16.pdf}

% Ejercicio 3.17 - Proyectil con altura inicial
\includepdf[pages=-]{Solucion_Ejercicio_3_17/Solucion_Ejercicio_3_17.pdf}

% Ejercicio 3.18 - Ángulo óptimo de lanzamiento
\includepdf[pages=-]{Solucion_Ejercicio_3_18/Solucion_Ejercicio_3_18.pdf}

% Ejercicio 3.19 - Velocidad al impacto
\includepdf[pages=-]{Solucion_Ejercicio_3_19/Solucion_Ejercicio_3_19.pdf}

% Ejercicio 3.20 - Proyectil con obstáculo
\includepdf[pages=-]{Solucion_Ejercicio_3_20/Solucion_Ejercicio_3_20.pdf}

% Ejercicio 3.21 - Dos proyectiles simultáneos
\includepdf[pages=-]{Solucion_Ejercicio_3_21/Solucion_Ejercicio_3_21.pdf}

% Ejercicio 3.22 - Cazador y mono
\includepdf[pages=-]{Solucion_Ejercicio_3_22/Solucion_Ejercicio_3_22.pdf}

% Ejercicio 3.23 - Pelota lanzada desde edificio
\includepdf[pages=-]{Solucion_Ejercicio_3_23/Solucion_Ejercicio_3_23.pdf}

% Ejercicio 3.24 - Alcance en terreno inclinado
\includepdf[pages=-]{Solucion_Ejercicio_3_24/Solucion_Ejercicio_3_24.pdf}

% Ejercicio 3.25 - Piedra desde globo en descenso
\includepdf[pages=-]{Solucion_Ejercicio_3_25/Solucion_Ejercicio_3_25.pdf}

% Ejercicio 3.26 - Cañón disparando hacia risco
\includepdf[pages=-]{Solucion_Ejercicio_3_26/Solucion_Ejercicio_3_26.pdf}

% Ejercicio 3.27 - Maleta cayendo de avión
\includepdf[pages=-]{Solucion_Ejercicio_3_27/Solucion_Ejercicio_3_27.pdf}

% --------------------------------------------------------------------------------
% FIN DEL DOCUMENTO
% --------------------------------------------------------------------------------
\end{document}

% ================================================================================
% NOTAS ADICIONALES PARA EL USUARIO
% ================================================================================
%
% ESTRUCTURA DE DIRECTORIOS:
%   Capítulo 3 Movimiento en Dos o Tres Dimensiones/
%   ├── main.tex (este archivo - compila todos los ejercicios en un solo PDF)
%   ├── preambCap_3MovDosTresDim.tex (referencia, no se usa con pdfpages)
%   ├── Solucion_Ejercicio_3_9/
%   │   ├── Solucion_Ejercicio_3_9.tex (fuente LaTeX)
%   │   └── Solucion_Ejercicio_3_9.pdf (PDF compilado)
%   ├── Solucion_Ejercicio_3_10/
%   │   ├── Solucion_Ejercicio_3_10.tex
%   │   └── Solucion_Ejercicio_3_10.pdf
%   └── ... (resto de ejercicios)
%
% PROCESO DE COMPILACIÓN:
%   1. Compilar cada ejercicio individual (genera los PDFs individuales):
%      cd Solucion_Ejercicio_3_9 && pdflatex Solucion_Ejercicio_3_9.tex
%      cd Solucion_Ejercicio_3_10 && pdflatex Solucion_Ejercicio_3_10.tex
%      ... (repetir para cada ejercicio)
%
%   2. Compilar este documento maestro (combina todos los PDFs):
%      pdflatex main.tex
%
% VENTAJAS DE ESTE MÉTODO:
%   - Cada ejercicio es un documento independiente que puede editarse solo
%   - No hay conflictos entre paquetes o configuraciones
%   - Compilación más rápida (solo compilas lo que modificaste)
%   - Fácil inclusión/exclusión de ejercicios (comentando líneas)
%
% PARA INCLUIR SOLO ALGUNOS EJERCICIOS:
%   Comenta con % las líneas \includepdf que no desees incluir
%
% SOLUCIÓN DE PROBLEMAS:
%   - Error "file not found": El PDF del ejercicio no existe. Compílalo primero.
%   - Para modificar un ejercicio: edita su .tex individual y recompílalo
%   - Para reorganizar: cambia el orden de las líneas \includepdf
%
% ================================================================================
