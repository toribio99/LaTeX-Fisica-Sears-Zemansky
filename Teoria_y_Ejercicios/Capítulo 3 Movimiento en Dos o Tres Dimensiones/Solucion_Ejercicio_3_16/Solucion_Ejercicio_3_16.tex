\documentclass[11pt,a4paper]{article}

% Paquetes necesarios
\usepackage[utf8]{inputenc}
\usepackage[T1]{fontenc}
\usepackage[spanish]{babel}
\usepackage[margin=2.5cm]{geometry}
\usepackage{amsmath}
\usepackage{amssymb}
\usepackage{xcolor}
\usepackage{tcolorbox}
\usepackage{graphicx}
\usepackage{tikz}
\usepackage{pgfplots}
\pgfplotsset{compat=1.18}
\usetikzlibrary{arrows.meta,patterns,decorations.pathmorphing,calc}
\usepackage{wrapfig}
\usepackage[export]{adjustbox} % (opcional) claves extra para \includegraphics
\usepackage{xparse}

% Definición de colores
\definecolor{azuloscuro}{RGB}{0,51,102}
\definecolor{azulclaro}{RGB}{230,240,250}
\definecolor{verdeoscuro}{RGB}{0,100,0}
\definecolor{rojoclaro}{RGB}{255,230,230}

% Configuración de cajas
\tcbuselibrary{theorems,skins,breakable}

\newtcolorbox{datosbox}{
    colback=azulclaro,
    colframe=azuloscuro,
    fonttitle=\bfseries,
    title=Datos del Problema,
    sharp corners,
    boxrule=1pt
}

\newtcolorbox{solucionbox}{
    colback=white,
    colframe=verdeoscuro,
    fonttitle=\bfseries,
    title=Desarrollo de la Solución,
    sharp corners,
    boxrule=1pt,
    breakable
}

\newtcolorbox{resultadobox}{
    colback=rojoclaro,
    colframe=red!70!black,
    fonttitle=\bfseries,
    title=Resultado Final,
    sharp corners,
    boxrule=2pt
}

% Título y autor
\title{\textbf{Solución del Ejercicio 3.16} \\
\large Movimiento de Proyectiles: Lanzamiento de Balón}
\author{Física Universitaria - Sears y Zemansky \\ Capítulo 3: Movimiento en Dos o Tres Dimensiones}
\date{\today}

\begin{document}

\maketitle

\section{Enunciado del Problema}

\begin{wrapfigure}[6]{r}{.6\textwidth}  % Ajusta el ancho según necesites
	\centering
	\vspace{-\baselineskip}
	%\begin{center}
\begin{tikzpicture}[scale=0.1]
    % Suelo
    \draw[fill=green!30, thick] (-10,-2) rectangle (70,0);
    \draw[pattern=north east lines, pattern color=green!60] (-10,-2) rectangle (70,-4);

    % Mariscal
    \draw[fill=blue!30, thick] (0,0) circle (3);
    \node[below] at (0,-7.5) {\small Mariscal};

    % Vector velocidad inicial
    \draw[-{Latex[length=4mm]}, red!70!black, ultra thick] (3,0) -- (15,9.6);
    \node[red!70!black, right] at (15,9.6) {$\vec{v}_0$};
    \draw[-{Latex[length=3mm]}, blue!70!black, ultra thick] (3,0) -- (15,0);
    \node[blue!70!black, below] at (13,-1) {$v_{0x} = 20.0$ m/s};
    \draw[-{Latex[length=3mm]}, blue!70!black, ultra thick] (15,0) -- (15,9.6);
    \node[blue!70!black, right] at (15,4.8) {$v_{0y} = 16.0$ m/s};

    % Trayectoria parabólica (ecuación física: y = v0y*t - 0.5*g*t^2, x = v0x*t)
    % En función de x: y = (v0y/v0x)*x - (g/(2*v0x^2))*x^2
    % Coeficientes: a = v0y/v0x = 16/20 = 0.8
    %              b = g/(2*v0x^2) = 9.8/(2*20^2) = 9.8/800 = 0.01225
    \draw[red!70!black, very thick, dashed, -{Latex[length=3mm]}]
        plot[domain=0:65.3, samples=100, smooth] (\x, {0.8*\x - 0.01225*\x*\x});

    % Punto máximo (x_max = 32.66 m, h_max = 13.06 m)
    \filldraw[orange!70!black] (32.66,13.06) circle (1.5);
    \node[orange!70!black, above] at (32.66,14.5) {Altura máxima};
    \draw[{Latex[length=2mm]}-{Latex[length=2mm]}, thick] (-5,0) -- (-5,13.06) node[midway,left] {$h_{\text{max}}$};

    % Punto final
    \filldraw[green!70!black] (65.3,0) circle (1.5);

    % Distancia horizontal
    \draw[{Latex[length=2mm]}-{Latex[length=2mm]}, thick] (0,-8) -- (65.3,-8) node[midway,below] {$R = ?$};

    % Etiquetas de tiempo
    \node[above] at (32.66,7) {$t = t_{\text{h-max}}$};
    \node[below] at (65.3,-7.5) {$t = t_{\text{total}}$};

\end{tikzpicture}
%\end{center}
\vspace{-\baselineskip} % Reduce espacio después
\end{wrapfigure}
Un mariscal de campo novato lanza un balón con una componente de velocidad inicial hacia arriba de 16.0 m/s y una componente de velocidad horizontal de 20.0 m/s. Ignore la resistencia del aire.

\vspace{3mm}

\textbf{Preguntas:}
\begin{enumerate}
	\item[a)] ¿Cuánto tiempo tardará el balón en llegar al punto más alto de la trayectoria?
\end{enumerate}

\begin{enumerate}
	\item[b)] ¿A qué altura está este punto?
	\item[c)] ¿Cuánto tiempo pasa desde que se lanza el balón hasta que vuelve a su nivel original? ¿Qué relación hay entre este tiempo y el calculado en el inciso a)?
	\item[d)] ¿Qué distancia horizontal viaja el balón en este tiempo?
	\item[e)] Dibuje gráficas $x$-$t$, $y$-$t$, $v_x$-$t$ y $v_y$-$t$ para el movimiento.
\end{enumerate}

\section{Datos del Problema}

\begin{datosbox}
\begin{itemize}
    \item \textbf{Componente horizontal de velocidad inicial:} $v_{0x} = 20.0$ m/s
    \item \textbf{Componente vertical de velocidad inicial:} $v_{0y} = 16.0$ m/s
    \item \textbf{Aceleración de la gravedad:} $g = 9.8$ m/s$^2$
    \item \textbf{Resistencia del aire:} despreciable
    \item \textbf{Altura inicial:} $y_0 = 0$ (nivel de lanzamiento)
\end{itemize}
\end{datosbox}

\section{Marco Teórico}

\subsection{Ecuaciones del Movimiento de Proyectiles}

\textbf{Movimiento horizontal:}
\begin{align}
    x &= v_{0x} t \\
    v_x &= v_{0x} = \text{constante}
\end{align}

\textbf{Movimiento vertical:}
\begin{align}
    y &= v_{0y} t - \frac{1}{2}gt^2 \\
    v_y &= v_{0y} - gt
\end{align}

\textbf{En el punto más alto:}
\begin{equation}
    v_y = 0 \quad \Rightarrow \quad t_{\text{max}} = \frac{v_{0y}}{g}
\end{equation}

\section{Desarrollo de la Solución}

\begin{solucionbox}

\subsection*{Parte a) Tiempo para alcanzar el punto más alto}

En el punto más alto de la trayectoria, la componente vertical de la velocidad es cero:
\begin{align*}
    v_y &= v_{0y} - gt_{\text{max}} = 0 \\
    t_{\text{max}} &= \frac{v_{0y}}{g} \\
    t_{\text{max}} &= \frac{16.0}{9.8} \\
    t_{\text{max}} &= 1.633 \text{ s}
\end{align*}

\textbf{Tiempo al punto máximo:} $t_{\text{max}} = 1.63$ s

\subsection*{Parte b) Altura máxima}

Usando la ecuación del movimiento vertical en $t = t_{\text{max}}$:
\begin{align*}
    h_{\text{max}} &= v_{0y} t_{\text{max}} - \frac{1}{2}g t_{\text{max}}^2 \\
    h_{\text{max}} &= 16.0(1.633) - \frac{1}{2}(9.8)(1.633)^2 \\
    h_{\text{max}} &= 26.13 - 13.07 \\
    h_{\text{max}} &= 13.06 \text{ m}
\end{align*}

También podemos usar la fórmula directa:
\begin{align*}
    h_{\text{max}} &= \frac{v_{0y}^2}{2g} \\
    h_{\text{max}} &= \frac{(16.0)^2}{2(9.8)} \\
    h_{\text{max}} &= \frac{256}{19.6} \\
    h_{\text{max}} &= 13.06 \text{ m}
\end{align*}

\textbf{Altura máxima:} $h_{\text{max}} = 13.1$ m

\subsection*{Parte c) Tiempo total de vuelo}

El balón regresa a su nivel original cuando $y = 0$:
\begin{align*}
    y &= v_{0y} t - \frac{1}{2}gt^2 = 0 \\
    t(v_{0y} - \frac{1}{2}gt) &= 0
\end{align*}

Esto da dos soluciones: $t = 0$ (lanzamiento) y:
\begin{align*}
    v_{0y} - \frac{1}{2}gt &= 0 \\
    t_{\text{total}} &= \frac{2v_{0y}}{g} \\
    t_{\text{total}} &= \frac{2(16.0)}{9.8} \\
    t_{\text{total}} &= 3.265 \text{ s}
\end{align*}

\textbf{Tiempo total:} $t_{\text{total}} = 3.27$ s

\textbf{Relación con el inciso a):}
\begin{equation*}
    t_{\text{total}} = 2 \times t_{\text{max}} = 2 \times 1.63 = 3.26 \text{ s}
\end{equation*}

El tiempo total es exactamente el doble del tiempo para alcanzar la altura máxima. Esto se debe a la simetría del movimiento parabólico.

\subsection*{Parte d) Distancia horizontal (alcance)}

El balón mantiene velocidad horizontal constante:
\begin{align*}
    R &= v_{0x} \times t_{\text{total}} \\
    R &= 20.0 \times 3.265 \\
    R &= 65.3 \text{ m}
\end{align*}

\textbf{Alcance horizontal:} $R = 65.3$ m

\end{solucionbox}

\section{Resultados Finales}

\begin{resultadobox}

\textbf{a) Tiempo al punto más alto:}
\begin{equation}
    \boxed{t_{\text{max}} = 1.63 \text{ s}}
\end{equation}

\textbf{b) Altura máxima:}
\begin{equation}
    \boxed{h_{\text{max}} = 13.1 \text{ m}}
\end{equation}

\textbf{c) Tiempo total de vuelo:}
\begin{equation}
    \boxed{t_{\text{total}} = 3.27 \text{ s} = 2 \times t_{\text{max}}}
\end{equation}

\textbf{d) Alcance horizontal:}
\begin{equation}
    \boxed{R = 65.3 \text{ m}}
\end{equation}

\end{resultadobox}

\section{Parte e) Gráficas del Movimiento}

\subsection{Gráfica $x$ vs $t$}

\begin{center}
\begin{tikzpicture}
\begin{axis}[
    width=0.85\textwidth,
    height=7cm,
    xlabel={Tiempo $t$ (s)},
    ylabel={Posición horizontal $x$ (m)},
    xmin=0, xmax=3.5,
    ymin=0, ymax=70,
    grid=major,
    title={Posición horizontal vs tiempo}
]
\addplot[blue, thick, domain=0:3.265] {20*x};
\addplot[red, only marks, mark=*] coordinates {(1.633,32.66) (3.265,65.3)};
\node[above] at (axis cs:1.633,32.66) {$(1.63, 32.7)$};
\node[above right] at (axis cs:3.265,65.3) {$(3.27, 65.3)$};
\end{axis}
\end{tikzpicture}
\end{center}

\subsection{Gráfica $y$ vs $t$}

\begin{center}
\begin{tikzpicture}
\begin{axis}[
    width=0.85\textwidth,
    height=7cm,
    xlabel={Tiempo $t$ (s)},
    ylabel={Altura $y$ (m)},
    xmin=0, xmax=3.5,
    ymin=0, ymax=15,
    grid=major,
    title={Altura vs tiempo}
]
\addplot[blue, thick, domain=0:3.265] {16*x - 4.9*x^2};
\addplot[red, only marks, mark=*] coordinates {(1.633,13.06) (3.265,0)};
\node[above right] at (axis cs:1.633,13.06) {Máximo: $(1.63, 13.1)$};
\node[below] at (axis cs:3.265,0) {$(3.27, 0)$};
\end{axis}
\end{tikzpicture}
\end{center}

\subsection{Gráfica $v_x$ vs $t$}

\begin{center}
\begin{tikzpicture}
\begin{axis}[
    width=0.85\textwidth,
    height=7cm,
    xlabel={Tiempo $t$ (s)},
    ylabel={Velocidad horizontal $v_x$ (m/s)},
    xmin=0, xmax=3.5,
    ymin=0, ymax=25,
    grid=major,
    title={Velocidad horizontal vs tiempo}
]
\addplot[blue, thick, domain=0:3.265] {20};
\node[above] at (axis cs:1.6,20) {$v_x = 20.0$ m/s (constante)};
\end{axis}
\end{tikzpicture}
\end{center}

\subsection{Gráfica $v_y$ vs $t$}

\begin{center}
\begin{tikzpicture}
\begin{axis}[
    width=0.85\textwidth,
    height=7cm,
    xlabel={Tiempo $t$ (s)},
    ylabel={Velocidad vertical $v_y$ (m/s)},
    xmin=0, xmax=3.5,
    ymin=-20, ymax=20,
    grid=major,
    title={Velocidad vertical vs tiempo}
]
\addplot[blue, thick, domain=0:3.265] {16 - 9.8*x};
\addplot[red, only marks, mark=*] coordinates {(1.633,0) (3.265,-16)};
\node[right] at (axis cs:1.633,0) {$v_y = 0$ en el máximo};
\node[below right] at (axis cs:3.265,-16) {$(3.27, -16.0)$};
\draw[dashed] (axis cs:0,0) -- (axis cs:3.5,0);
\end{axis}
\end{tikzpicture}
\end{center}

\section{Verificación}

\subsection{Verificación de simetría}

La velocidad vertical al regresar al nivel original debe ser igual en magnitud pero opuesta en dirección a la velocidad inicial:
\begin{align*}
    v_y(t_{\text{total}}) &= v_{0y} - g t_{\text{total}} \\
    &= 16.0 - 9.8(3.265) \\
    &= 16.0 - 32.0 \\
    &= -16.0 \text{ m/s} \quad \checkmark
\end{align*}

\subsection{Verificación del alcance}

Usando la fórmula del alcance para lanzamiento desde el nivel del suelo:
\begin{align*}
    R &= \frac{2v_{0x}v_{0y}}{g} \\
    R &= \frac{2(20.0)(16.0)}{9.8} \\
    R &= \frac{640}{9.8} \\
    R &= 65.3 \text{ m} \quad \checkmark
\end{align*}

\section{Conceptos Clave}

\begin{enumerate}
    \item \textbf{Simetría del movimiento parabólico:} El tiempo de subida es igual al tiempo de bajada.

    \item \textbf{Velocidad horizontal constante:} No hay aceleración en la dirección horizontal.

    \item \textbf{Altura máxima cuando $v_y = 0$:} En el punto más alto, toda la velocidad es horizontal.

    \item \textbf{Simetría de velocidades:} La velocidad vertical al impacto es igual en magnitud pero opuesta a la velocidad vertical inicial.
\end{enumerate}

\section{Fórmulas Utilizadas}

\begin{tcolorbox}[colback=yellow!10!white,colframe=orange!75!black,title=Resumen de Fórmulas]

\textbf{Tiempo al punto máximo:}
\begin{equation*}
    t_{\text{max}} = \frac{v_{0y}}{g} = \frac{16.0}{9.8} = 1.63 \text{ s}
\end{equation*}

\textbf{Altura máxima:}
\begin{equation*}
    h_{\text{max}} = \frac{v_{0y}^2}{2g} = \frac{(16.0)^2}{2(9.8)} = 13.1 \text{ m}
\end{equation*}

\textbf{Tiempo total:}
\begin{equation*}
    t_{\text{total}} = \frac{2v_{0y}}{g} = 2t_{\text{max}} = 3.27 \text{ s}
\end{equation*}

\textbf{Alcance horizontal:}
\begin{equation*}
    R = v_{0x} t_{\text{total}} = 20.0 \times 3.27 = 65.3 \text{ m}
\end{equation*}

\end{tcolorbox}

\end{document}
