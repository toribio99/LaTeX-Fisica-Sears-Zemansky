\documentclass[11pt,a4paper]{article}

% Paquetes necesarios
\usepackage[utf8]{inputenc}
\usepackage[T1]{fontenc}
\usepackage[spanish]{babel}
\usepackage[margin=2.5cm]{geometry}
\usepackage{amsmath}
\usepackage{amssymb}
\usepackage{xcolor}
\usepackage{tcolorbox}
\usepackage{graphicx}
\usepackage{tikz}
\usepackage{pgfplots}
\pgfplotsset{compat=1.18}
\usetikzlibrary{arrows.meta,patterns,decorations.pathmorphing,calc}
\usepackage{wrapfig}
\usepackage[export]{adjustbox} % (opcional) claves extra para \includegraphics
\usepackage{xparse}

% Definición de colores
\definecolor{azuloscuro}{RGB}{0,51,102}
\definecolor{azulclaro}{RGB}{230,240,250}
\definecolor{verdeoscuro}{RGB}{0,100,0}
\definecolor{rojoclaro}{RGB}{255,230,230}

% Configuración de cajas
\tcbuselibrary{theorems,skins,breakable}

\newtcolorbox{datosbox}{
    colback=azulclaro,
    colframe=azuloscuro,
    fonttitle=\bfseries,
    title=Datos del Problema,
    sharp corners,
    boxrule=1pt
}

\newtcolorbox{solucionbox}{
    colback=white,
    colframe=verdeoscuro,
    fonttitle=\bfseries,
    title=Desarrollo de la Solución,
    sharp corners,
    boxrule=1pt,
    breakable
}

\newtcolorbox{resultadobox}{
    colback=rojoclaro,
    colframe=red!70!black,
    fonttitle=\bfseries,
    title=Resultado Final,
    sharp corners,
    boxrule=2pt
}

% Título y autor
\title{\textbf{Solución del Ejercicio 3.27} \\
\large Maleta que Cae de un Avión}
\author{Física Universitaria - Sears y Zemansky \\ Capítulo 3: Movimiento en Dos o Tres Dimensiones}
\date{\today}

\begin{document}

\maketitle

\section{Enunciado del Problema}

\begin{wrapfigure}[7]{r}{.45\textwidth}  % Ajusta el ancho según necesites
	\centering
	\vspace{-\baselineskip}
	%\begin{center}
\begin{tikzpicture}[scale=0.4]
    % Suelo
    \draw[fill=brown!20, thick] (-1,0) rectangle (12,-0.3);
    \draw[pattern=north east lines, pattern color=brown] (-1,-0.3) rectangle (12,-0.5);

    % Perro
    \draw[fill=brown!60, thick] (0,0) ellipse (0.2 and 0.1);
    \draw[fill=brown!60, thick] (-0.1,0) -- (-0.15,0.15) -- (-0.05,0.15) -- cycle;
    \draw[fill=brown!60, thick] (0.1,0) -- (0.15,0.15) -- (0.05,0.15) -- cycle;
    \node[below, font=\scriptsize] at (0,-0.18) {Perro};

    % Avión (posición inicial)
    \draw[fill=gray!40, thick] (-0.3,7.6) -- (0.5,7.8) -- (0.6,7.7) -- (0.5,7.55) -- cycle;
    \draw[fill=gray!50, thick] (-0.1,7.65) -- (-0.1,7.85) -- (0.15,7.85) -- (0.15,7.65) -- cycle;
    \draw[fill=blue!30, thick] (0.1,7.75) circle (0.06);
    \node[above, font=\scriptsize] at (0,7.95) {Avión};

    % Línea vertical punteada
    \draw[dashed, thick] (0,0) -- (0,7.6);
    \node[left, font=\scriptsize] at (-0.1,3.8) {$114$ m};

    % Vector velocidad del avión
    \draw[-{Latex[length=2.5mm]},red!70!black, ultra thick] (0,7.6) -- (1.5,8.2);
    \node[red!70!black, above, font=\scriptsize] at (0.75,8.3) {$\vec{v}=90$ m/s};
    \node[red!70!black, below, font=\scriptsize] at (0.75,7.9) {$23°$};

    % Trayectoria parabólica de la maleta
    % y = 7.6 + 0.424x - 0.1014x² (ajustado para tocar el suelo en x=11)
    \draw[cyan!70!blue, very thick, -{Latex[length=2.5mm]}]
        plot[domain=0:11, samples=100, smooth] (\x, {7.6 + 0.424*\x - 0.1014*\x*\x});

    % Maleta cayendo en diferentes posiciones
    %\filldraw[orange!80!black] (3,8.2) circle (0.08);
    %\filldraw[orange!80!black] (6,8.1) circle (0.08);
    %\filldraw[orange!80!black] (9,6.5) circle (0.08);

    % Punto de impacto
    \filldraw[green!60!black] (11,0) circle (0.1);
    \node[green!60!black, below, font=\scriptsize] at (11,-0.2) {Impacto};

    % Distancia horizontal
    \draw[{Latex[length=2mm]}-{Latex[length=2mm]}, thick] (0,-1.2) -- (11,-1.2);
    \node[below, font=\small] at (5.5,-1.2) {$d = ?$};

\end{tikzpicture}
%\end{center}
\vspace{-\baselineskip} % Reduce espacio después
\end{wrapfigure}
Un avión vuela con una velocidad de 90.0 m/s a un ángulo de 23.0° arriba de la horizontal. Cuando está 114 m directamente arriba de un perro parado en suelo plano, se cae una maleta del compartimiento de equipaje. ¿A qué distancia del perro caerá la maleta? Ignore la resistencia del aire.

\vspace{6mm}

\section{Datos del Problema}

\begin{datosbox}
\begin{itemize}
    \item \textbf{Velocidad del avión:} $v_{avión} = 90.0$ m/s a $23.0°$ arriba de la horizontal
    \item \textbf{Altura inicial:} $h_0 = 114$ m (directamente arriba del perro)
    \item \textbf{Aceleración gravitacional:} $g = 9.8$ m/s$^2$
    \item \textbf{Velocidad inicial de la maleta:} Igual a la velocidad del avión
    \item \textbf{Resistencia del aire:} despreciable
    \item \textbf{Incógnita:} Distancia horizontal desde el perro hasta donde cae la maleta
\end{itemize}
\end{datosbox}

\section{Marco Teórico}

Cuando un objeto se suelta desde un vehículo en movimiento, hereda la velocidad del vehículo. En este caso, la maleta tiene la misma velocidad inicial que el avión.

\textbf{Componentes de la velocidad inicial de la maleta:}
\begin{align}
    v_{0x} &= v_{avión} \cos\theta \\
    v_{0y} &= v_{avión} \sin\theta
\end{align}

\textbf{Ecuaciones de posición:}
\begin{align}
    x(t) &= v_{0x}t \\
    y(t) &= h_0 + v_{0y}t - \frac{1}{2}gt^2
\end{align}

\section{Desarrollo de la Solución}

\begin{solucionbox}

\subsection*{Paso 1: Componentes de la velocidad inicial}

La maleta hereda la velocidad del avión en el momento de caer:

\begin{align}
    v_{0x} &= v_{avión} \cos(23.0°) = 90.0 \times 0.9205 = 82.85 \text{ m/s} \\
    v_{0y} &= v_{avión} \sin(23.0°) = 90.0 \times 0.3907 = 35.16 \text{ m/s}
\end{align}

\subsection*{Paso 2: Tiempo de caída}

La maleta cae al suelo cuando $y = 0$. Tomando el punto inicial como $y_0 = h_0 = 114$ m:

\begin{equation}
    y = h_0 + v_{0y}t - \frac{1}{2}gt^2
\end{equation}

Cuando $y = 0$:

\begin{align}
    0 &= 114 + 35.16t - 4.9t^2 \\
    4.9t^2 - 35.16t - 114 &= 0
\end{align}

Usando la fórmula cuadrática:

\begin{align}
    t &= \frac{35.16 \pm \sqrt{(35.16)^2 + 4(4.9)(114)}}{2(4.9)} \\
    t &= \frac{35.16 \pm \sqrt{1236.23 + 2234.4}}{9.8} \\
    t &= \frac{35.16 \pm \sqrt{3470.63}}{9.8} \\
    t &= \frac{35.16 \pm 58.91}{9.8}
\end{align}

Tomando la raíz positiva:

\begin{equation}
    t = \frac{94.07}{9.8} = 9.60 \text{ s}
\end{equation}

\subsection*{Paso 3: Distancia horizontal}

La distancia horizontal que recorre la maleta es:

\begin{align}
    d &= v_{0x} \cdot t \\
    d &= 82.85 \times 9.60 \\
    d &= 795.4 \text{ m}
\end{align}

La maleta cae a \textbf{795.4 metros} de distancia del perro.

\subsection*{Verificación y análisis adicional}

\textbf{Velocidad al impactar el suelo:}

Componente horizontal (constante):
\begin{equation}
    v_x = v_{0x} = 82.85 \text{ m/s}
\end{equation}

Componente vertical:
\begin{align}
    v_y &= v_{0y} - gt \\
    v_y &= 35.16 - 9.8(9.60) \\
    v_y &= 35.16 - 94.08 \\
    v_y &= -58.92 \text{ m/s}
\end{align}

Magnitud de la velocidad:
\begin{align}
    v &= \sqrt{v_x^2 + v_y^2} \\
    v &= \sqrt{(82.85)^2 + (-58.92)^2} \\
    v &= \sqrt{6864.12 + 3471.57} \\
    v &= \sqrt{10335.69} = 101.66 \text{ m/s}
\end{align}

\textbf{Altura máxima alcanzada:}

La maleta alcanza su altura máxima cuando $v_y = 0$:

\begin{equation}
    t_{max} = \frac{v_{0y}}{g} = \frac{35.16}{9.8} = 3.59 \text{ s}
\end{equation}

Altura máxima:
\begin{align}
    h_{max} &= h_0 + v_{0y}t_{max} - \frac{1}{2}gt_{max}^2 \\
    &= 114 + 35.16(3.59) - \frac{1}{2}(9.8)(3.59)^2 \\
    &= 114 + 126.22 - 63.11 \\
    &= 177.11 \text{ m}
\end{align}

\end{solucionbox}

\section{Resultados Finales}

\begin{resultadobox}

\textbf{Distancia del perro:}
\begin{equation}
    \boxed{d = 795.4 \text{ m} \approx 795 \text{ m}}
\end{equation}

La maleta cae a \textbf{795 metros} (casi 0.8 km) de distancia del perro.

\vspace{0.3cm}

\textbf{Información adicional:}
\begin{itemize}
    \item Tiempo de caída: $t = 9.60$ s
    \item Velocidad inicial de la maleta: $v_0 = 90.0$ m/s a $23.0°$
    \item Componentes de velocidad inicial: $v_{0x} = 82.85$ m/s, $v_{0y} = 35.16$ m/s
    \item Altura máxima alcanzada: $h_{max} = 177.11$ m (sobre el suelo)
    \item Tiempo para alcanzar altura máxima: $t_{max} = 3.59$ s
    \item Velocidad al impactar: $v = 101.66$ m/s
    \item Componentes de velocidad al impactar: $v_x = 82.85$ m/s, $v_y = -58.92$ m/s
    \item Ángulo de impacto: $\theta = \arctan(-58.92/82.85) = -35.4°$ (bajo la horizontal)
\end{itemize}

\end{resultadobox}

\section{Diagrama de la Trayectoria}

\begin{center}
\begin{tikzpicture}
\begin{axis}[
    width=14cm, height=9cm,
    xlabel={Posición horizontal $x$ (m)},
    ylabel={Altura $y$ (m)},
    xmin=0, xmax=850,
    ymin=0, ymax=190,
    grid=both,
    grid style={line width=.1pt, draw=gray!30},
    major grid style={line width=.2pt,draw=gray!50},
    legend pos=north east
]
    % Trayectoria: y = 114 + 0.424x - 0.000714x²
    % Coeficiente correcto: g/(2v₀x²) = 9.8/(2×82.85²) = 0.000714
    \addplot[domain=0:795.4, samples=100, cyan!70!blue, very thick] {114 + 0.424*x - 0.000714*x^2};

    % Punto inicial (avión)
    \addplot[only marks, mark=*, mark size=3pt, blue] coordinates {(0, 114)};
    \node[blue, left] at (axis cs:0,114) {Avión};

    % Altura máxima
    \addplot[only marks, mark=*, mark size=3pt, orange] coordinates {(297.1, 177.11)};
    \node[orange, above] at (axis cs:297.1,177.11) {Altura máxima};

    % Punto de impacto
    \addplot[only marks, mark=*, mark size=3pt, green!60!black] coordinates {(795.4, 0)};
    \node[green!60!black, below right] at (axis cs:795.4,0) {Impacto};

    % Posición del perro
    \addplot[only marks, mark=*, mark size=3pt, brown] coordinates {(0, 0)};
    \node[brown, below] at (axis cs:0,0) {Perro};

    % Vector velocidad inicial
    \draw[-{Latex[length=3mm]}, red!70!black, ultra thick] (axis cs:0,114) -- (axis cs:100,152);
    \node[red!70!black, above right] at (axis cs:50,140) {$\vec{v}_0$};

\end{axis}
\end{tikzpicture}
\end{center}

\section{Análisis y Conclusión}

Este problema ilustra el movimiento de proyectiles desde una plataforma móvil. Las conclusiones principales son:

\begin{enumerate}
    \item \textbf{Herencia de velocidad:} La maleta hereda completamente la velocidad del avión en el momento de caer. Esto significa que tiene componentes de velocidad tanto horizontal (82.85 m/s) como vertical (35.16 m/s hacia arriba).

    \item \textbf{Larga distancia horizontal:} La maleta cae a 795 m del perro, casi 0.8 km. Esto se debe a:
    \begin{itemize}
        \item La alta velocidad horizontal heredada del avión
        \item El largo tiempo de vuelo (9.6 s)
        \item La velocidad vertical inicial hacia arriba que aumenta el tiempo en el aire
    \end{itemize}

    \item \textbf{Trayectoria ascendente inicial:} A pesar de "caer", la maleta inicialmente sube durante los primeros 3.59 s, alcanzando una altura máxima de 177.11 m (63 m por encima del avión). Esto se debe a la componente vertical positiva de la velocidad inicial.

    \item \textbf{Fases del movimiento:}
    \begin{itemize}
        \item Fase 1 (0 a 3.59 s): Movimiento ascendente, alcanza 177.11 m
        \item Fase 2 (3.59 a 9.60 s): Movimiento descendente hasta el suelo
        \item Tiempo total: 9.60 s
    \end{itemize}

    \item \textbf{Velocidad de impacto:} La maleta golpea el suelo con 101.66 m/s, mayor que la velocidad inicial de 90.0 m/s. Esto se debe a la conversión de energía potencial en cinética durante la caída desde 114 m de altura.

    \item \textbf{Referencia del avión:} Desde la perspectiva de alguien en el avión, la maleta parece caer verticalmente. Pero desde el suelo, sigue una parábola.

    \item \textbf{Consideración de resistencia del aire:} En la realidad, la resistencia del aire reduciría significativamente la distancia horizontal. Una maleta tiene un área frontal grande y caería mucho más cerca del perro que los 795 m calculados.

    \item \textbf{Ecuación de la trayectoria:} La trayectoria parabólica es:
    \begin{equation}
        y = 114 + 0.424x - 0.000714x^2
    \end{equation}
    donde el coeficiente del término cuadrático es $g/(2v_{0x}^2) = 9.8/(2 \times 82.85^2) = 0.000714$.

\end{enumerate}

\textbf{Aplicaciones prácticas:} Este tipo de problema es relevante para:
\begin{itemize}
    \item Suministro aéreo de ayuda humanitaria
    \item Bombardeo aéreo (aplicaciones militares)
    \item Análisis de accidentes aéreos
    \item Entender el movimiento relativo entre sistemas de referencia
\end{itemize}

\end{document}
