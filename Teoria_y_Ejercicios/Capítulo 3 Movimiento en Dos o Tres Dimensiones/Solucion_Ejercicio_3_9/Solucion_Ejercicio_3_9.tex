\documentclass[11pt,a4paper]{article}

% Paquetes necesarios
\usepackage[utf8]{inputenc}
\usepackage[T1]{fontenc}
\usepackage[spanish]{babel}
\usepackage[margin=2.5cm]{geometry}
\usepackage{amsmath}
\usepackage{amssymb}
\usepackage{xcolor}
\usepackage{tcolorbox}
\usepackage{graphicx}
\usepackage{tikz}
\usepackage{pgfplots}
\pgfplotsset{compat=1.18}
\usetikzlibrary{arrows.meta,patterns,decorations.pathmorphing,calc}
\usepackage{wrapfig}
\usepackage[export]{adjustbox} % (opcional) claves extra para \includegraphics
\usepackage{xparse}

% Definición de colores
\definecolor{azuloscuro}{RGB}{0,51,102}
\definecolor{azulclaro}{RGB}{230,240,250}
\definecolor{verdeoscuro}{RGB}{0,100,0}
\definecolor{rojoclaro}{RGB}{255,230,230}

% Configuración de cajas
\tcbuselibrary{theorems,skins,breakable}

\newtcolorbox{datosbox}{
    colback=azulclaro,
    colframe=azuloscuro,
    fonttitle=\bfseries,
    title=Datos del Problema,
    sharp corners,
    boxrule=1pt
}

\newtcolorbox{solucionbox}{
    colback=white,
    colframe=verdeoscuro,
    fonttitle=\bfseries,
    title=Desarrollo de la Solución,
    sharp corners,
    boxrule=1pt,
    breakable
}

\newtcolorbox{resultadobox}{
    colback=rojoclaro,
    colframe=red!70!black,
    fonttitle=\bfseries,
    title=Resultado Final,
    sharp corners,
    boxrule=2pt
}

% Título y autor
\title{\textbf{Solución del Ejercicio 3.9} \\
\large Movimiento de Proyectiles: Libro que Cae de una Mesa}
\author{Física Universitaria - Sears y Zemansky \\ Capítulo 3: Movimiento en Dos o Tres Dimensiones}
\date{\today}

\begin{document}

\maketitle

\section{Enunciado del Problema}

\begin{wrapfigure}[8]{r}{.6\textwidth}  % Ajusta el ancho según necesites
	\centering
	\vspace{-\baselineskip}
	%\begin{center}
\begin{tikzpicture}[scale=1.3]
    % Mesa
    \draw[fill=brown!40, thick] (0,0) rectangle (4,0.3);
    \draw[fill=brown!60, thick] (-0.2,-2) rectangle (0,0);

    % Piso
    \draw[fill=gray!30, thick] (-1,-2) rectangle (6,-2.4);
    \draw[pattern=north east lines, pattern color=gray!50] (-1,-2.6) rectangle (6,-2.4);

    % Libro sobre la mesa
    \filldraw[blue!70, thick] (3.5,0.3) rectangle (4,0.6);
    \node at (3.75,0.45) {\tiny Libro};

    % Flecha de velocidad
    \draw[-{Latex[length=3mm]},red!70!black, ultra thick] (4,0.5) -- (5,0.5)
        node[midway,above] {$v_0 = 1.10$ m/s};

    % Trayectoria parabólica (ecuación física ajustada al diagrama)
    % La parábola debe ir desde (4, 0.45) hasta (4.385, -2)
    % Coeficiente calculado: k = 2.45/0.385^2 = 16.55
    \draw[blue!70!black, thick, dashed, -{Latex[length=2mm]}]
        plot[domain=4:4.385, samples=50, smooth] (\x, {0.45 - 16.55*(\x-4)*(\x-4)});

    % Punto de impacto
    \filldraw[red] (4.385,-2) circle (0.08);

    % Cotas
    \draw[{Latex[length=2mm]}-{Latex[length=2mm]}, thick]
        (4,-2) -- (4,0.3) node[midway,left] {$h = ?$};

    \draw[{Latex[length=2mm]}-{Latex[length=2mm]}, thick]
        (4,-2) -- (4.385,-2) node[midway,below] {$d = ?$};

    % Etiquetas
    \node at (2,0.15) {Mesa};
    \node at (5,-2.2) {Piso};

\end{tikzpicture}
%\end{center}
\vspace{-\baselineskip} % Reduce espacio después
\end{wrapfigure}
Un libro de física que se desliza sobre una mesa horizontal a 1.10 m/s cae al piso en 0.350 s. Ignore la resistencia del aire. Calcule:

\begin{enumerate}
	\item[a)] La altura de la mesa.
	\item[b)] La distancia horizontal del borde de la mesa al punto donde cae el libro.
\end{enumerate}

\begin{enumerate}
	\item[c)] Las componentes horizontal y vertical, y la magnitud y dirección, de la velocidad del libro justo antes de tocar el piso.
	\item[d)] Dibuje gráficas $x$-$t$, $y$-$t$, $v_x$-$t$ y $v_y$-$t$ para el movimiento.
\end{enumerate}

\section{Datos del Problema}

\begin{datosbox}
\begin{itemize}
    \item \textbf{Velocidad inicial horizontal:} $v_0 = 1.10$ m/s
    \item \textbf{Velocidad inicial vertical:} $v_{0y} = 0$ (movimiento horizontal)
    \item \textbf{Tiempo de caída:} $t = 0.350$ s
    \item \textbf{Aceleración de la gravedad:} $g = 9.8$ m/s$^2$
    \item \textbf{Resistencia del aire:} despreciable
\end{itemize}
\end{datosbox}

\section{Marco Teórico}

\subsection{Ecuaciones del Movimiento de Proyectiles}

Para un objeto lanzado horizontalmente:

\textbf{Movimiento horizontal (velocidad constante):}
\begin{equation}
    x = v_0 t
\end{equation}
\begin{equation}
    v_x = v_0 = \text{constante}
\end{equation}

\textbf{Movimiento vertical (caída libre):}
\begin{equation}
    y = y_0 - \frac{1}{2}gt^2
\end{equation}
\begin{equation}
    v_y = -gt
\end{equation}

\section{Desarrollo de la Solución}

\begin{solucionbox}

\subsection*{Parte a) Altura de la mesa}

El libro cae desde el reposo vertical (sin velocidad inicial vertical) y tarda 0.350 s en llegar al suelo.

Usando la ecuación del movimiento vertical:
\begin{align*}
    h &= \frac{1}{2}gt^2 \\
    h &= \frac{1}{2}(9.8)(0.350)^2 \\
    h &= \frac{1}{2}(9.8)(0.1225) \\
    h &= 4.9 \times 0.1225 \\
    h &= 0.600 \text{ m}
\end{align*}

\textbf{Altura de la mesa:} $h = 0.600$ m = 60.0 cm

\subsection*{Parte b) Distancia horizontal}

El libro mantiene su velocidad horizontal constante de 1.10 m/s durante los 0.350 s de caída.

\begin{align*}
    d &= v_0 t \\
    d &= (1.10)(0.350) \\
    d &= 0.385 \text{ m}
\end{align*}

\textbf{Distancia horizontal:} $d = 0.385$ m = 38.5 cm

\subsection*{Parte c) Velocidad justo antes de tocar el piso}

\subsubsection*{Componente horizontal:}
La velocidad horizontal permanece constante:
\begin{equation*}
    v_x = v_0 = 1.10 \text{ m/s}
\end{equation*}

\subsubsection*{Componente vertical:}
\begin{align*}
    v_y &= -gt \\
    v_y &= -(9.8)(0.350) \\
    v_y &= -3.43 \text{ m/s}
\end{align*}

(El signo negativo indica movimiento hacia abajo)

\subsubsection*{Magnitud de la velocidad:}
\begin{align*}
    v &= \sqrt{v_x^2 + v_y^2} \\
    v &= \sqrt{(1.10)^2 + (3.43)^2} \\
    v &= \sqrt{1.21 + 11.76} \\
    v &= \sqrt{12.97} \\
    v &= 3.60 \text{ m/s}
\end{align*}

\subsubsection*{Dirección de la velocidad:}
\begin{align*}
    \theta &= \arctan\left(\frac{|v_y|}{v_x}\right) \\
    \theta &= \arctan\left(\frac{3.43}{1.10}\right) \\
    \theta &= \arctan(3.118) \\
    \theta &= 72.2°
\end{align*}

El libro impacta el piso con un ángulo de 72.2° por debajo de la horizontal.

\end{solucionbox}

\section{Resultados Finales}

\begin{resultadobox}

\subsection*{Parte a) Altura de la mesa}
\begin{equation}
    \boxed{h = 0.600 \text{ m} = 60.0 \text{ cm}}
\end{equation}

\subsection*{Parte b) Distancia horizontal}
\begin{equation}
    \boxed{d = 0.385 \text{ m} = 38.5 \text{ cm}}
\end{equation}

\subsection*{Parte c) Velocidad justo antes del impacto}

\textbf{Componentes:}
\begin{itemize}
    \item Horizontal: $v_x = 1.10$ m/s
    \item Vertical: $v_y = 3.43$ m/s (hacia abajo)
\end{itemize}

\textbf{Magnitud:}
\begin{equation}
    \boxed{v = 3.60 \text{ m/s}}
\end{equation}

\textbf{Dirección:}
\begin{equation}
    \boxed{\theta = 72.2° \text{ por debajo de la horizontal}}
\end{equation}

\end{resultadobox}

\section{Parte d) Gráficas del Movimiento}

\subsection{Gráfica $x$ vs $t$}

La posición horizontal aumenta linealmente con el tiempo:

\begin{center}
\begin{tikzpicture}
\begin{axis}[
    width=0.8\textwidth,
    height=6cm,
    xlabel={Tiempo $t$ (s)},
    ylabel={Posición $x$ (m)},
    xmin=0, xmax=0.4,
    ymin=0, ymax=0.5,
    grid=major,
    title={Posición horizontal vs tiempo}
]
\addplot[blue, thick, domain=0:0.35] {1.10*x};
\addplot[red, only marks, mark=*] coordinates {(0.35,0.385)};
\node[above right] at (axis cs:0.3,0.385) {$(0.350, 0.385)$};
\end{axis}
\end{tikzpicture}
\end{center}

\subsection{Gráfica $y$ vs $t$}

La posición vertical desciende parabólicamente:

\begin{center}
\begin{tikzpicture}
\begin{axis}[
    width=0.8\textwidth,
    height=6cm,
    xlabel={Tiempo $t$ (s)},
    ylabel={Posición $y$ (m)},
    xmin=0, xmax=0.2,
    ymin=-0.1, ymax=0.4,
    grid=major,
    title={Posición vertical vs tiempo (desde el nivel de la mesa)}
]
\addplot[blue, thick, domain=0:0.35] {-4.9*x^2};
\addplot[red, only marks, mark=*] coordinates {(0.35,-0.600)};
\node[below right] at (axis cs:0.35,-0.600) {$(0.350, -0.600)$};
\node[above left] at (axis cs:0,0) {Mesa};
\end{axis}
\end{tikzpicture}
\end{center}

\subsection{Gráfica $v_x$ vs $t$}

La velocidad horizontal permanece constante:

\begin{center}
\begin{tikzpicture}
\begin{axis}[
    width=0.8\textwidth,
    height=6cm,
    xlabel={Tiempo $t$ (s)},
    ylabel={Velocidad $v_x$ (m/s)},
    xmin=0, xmax=0.4,
    ymin=0, ymax=1.5,
    grid=major,
    title={Velocidad horizontal vs tiempo}
]
\addplot[blue, thick, domain=0:0.35] {1.10};
\addplot[red, only marks, mark=*] coordinates {(0.35,1.10)};
\end{axis}
\end{tikzpicture}
\end{center}

\subsection{Gráfica $v_y$ vs $t$}

La velocidad vertical aumenta linealmente (hacia abajo):

\begin{center}
\begin{tikzpicture}
\begin{axis}[
    width=0.8\textwidth,
    height=6cm,
    xlabel={Tiempo $t$ (s)},
    ylabel={Velocidad $v_y$ (m/s)},
    xmin=0, xmax=0.4,
    ymin=-4, ymax=0.5,
    grid=major,
    title={Velocidad vertical vs tiempo}
]
\addplot[blue, thick, domain=0:0.35] {-9.8*x};
\addplot[red, only marks, mark=*] coordinates {(0.35,-3.43)};
\node[below right] at (axis cs:0.35,-3.43) {$(0.350, -3.43)$};
\end{axis}
\end{tikzpicture}
\end{center}

\section{Verificación}

\subsection{Verificación de la altura}

Usando la velocidad vertical final:
\begin{align*}
    v_y^2 &= v_{0y}^2 + 2gh \\
    (-3.43)^2 &= 0 + 2(9.8)h \\
    11.76 &= 19.6h \\
    h &= \frac{11.76}{19.6} = 0.600 \text{ m} \quad \checkmark
\end{align*}

\subsection{Verificación de la velocidad}

Usando energía (considerando solo magnitudes):
\begin{align*}
    \frac{1}{2}mv^2 &= \frac{1}{2}mv_0^2 + mgh \\
    v^2 &= v_0^2 + 2gh \\
    v^2 &= (1.10)^2 + 2(9.8)(0.600) \\
    v^2 &= 1.21 + 11.76 = 12.97 \\
    v &= 3.60 \text{ m/s} \quad \checkmark
\end{align*}

\section{Conceptos Clave}

\begin{enumerate}
    \item \textbf{Lanzamiento horizontal:} El objeto tiene velocidad inicial solo en la dirección horizontal.

    \item \textbf{Caída libre vertical:} El movimiento vertical es independiente del horizontal y sigue las leyes de caída libre.

    \item \textbf{Velocidad constante horizontal:} En ausencia de resistencia del aire, $v_x$ no cambia.

    \item \textbf{Trayectoria parabólica:} La combinación de movimiento horizontal uniforme y vertical acelerado produce una parábola.

    \item \textbf{Tiempo determina todo:} Conocer el tiempo de caída es suficiente para determinar todas las demás cantidades.
\end{enumerate}

\section{Análisis Adicional}

\subsection{Relación entre altura y velocidad vertical}

La velocidad vertical al impactar depende solo de la altura de caída:
\begin{equation*}
    v_y = \sqrt{2gh} = \sqrt{2(9.8)(0.600)} = 3.43 \text{ m/s}
\end{equation*}

\subsection{Ángulo de impacto}

El ángulo de 72.2° indica que el libro impacta el piso con una trayectoria muy inclinada, casi vertical. Esto se debe a que:
\begin{itemize}
    \item La velocidad horizontal es relativamente pequeña (1.10 m/s)
    \item La velocidad vertical al impacto es mucho mayor (3.43 m/s)
    \item La relación es: $v_y/v_x = 3.43/1.10 \approx 3.1$
\end{itemize}

\section{Resumen de Fórmulas}

\begin{tcolorbox}[colback=yellow!10!white,colframe=orange!75!black,title=Fórmulas Clave]

\textbf{Para lanzamiento horizontal desde altura $h$ con velocidad $v_0$:}

\textbf{Tiempo de caída:}
\begin{equation*}
    t = \sqrt{\frac{2h}{g}}
\end{equation*}

\textbf{Altura (si se conoce $t$):}
\begin{equation*}
    h = \frac{1}{2}gt^2
\end{equation*}

\textbf{Alcance horizontal:}
\begin{equation*}
    d = v_0 t = v_0\sqrt{\frac{2h}{g}}
\end{equation*}

\textbf{Velocidad final:}
\begin{align*}
    v_x &= v_0 \\
    v_y &= \sqrt{2gh} \\
    v &= \sqrt{v_0^2 + 2gh}
\end{align*}

\end{tcolorbox}

\end{document}
