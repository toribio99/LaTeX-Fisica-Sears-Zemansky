\documentclass[11pt,a4paper]{article}

% Paquetes necesarios
\usepackage[utf8]{inputenc}
\usepackage[T1]{fontenc}
\usepackage[spanish]{babel}
\usepackage[margin=2.5cm]{geometry}
\usepackage{amsmath}
\usepackage{amssymb}
\usepackage{xcolor}
\usepackage{tcolorbox}
\usepackage{graphicx}
\usepackage{tikz}
\usepackage{pgfplots}
\pgfplotsset{compat=1.18}
\usetikzlibrary{arrows.meta,patterns,decorations.pathmorphing,calc}

% Definición de colores
\definecolor{azuloscuro}{RGB}{0,51,102}
\definecolor{azulclaro}{RGB}{230,240,250}
\definecolor{verdeoscuro}{RGB}{0,100,0}
\definecolor{rojoclaro}{RGB}{255,230,230}

% Configuración de cajas
\tcbuselibrary{theorems,skins,breakable}

\newtcolorbox{datosbox}{
    colback=azulclaro,
    colframe=azuloscuro,
    fonttitle=\bfseries,
    title=Datos del Problema,
    sharp corners,
    boxrule=1pt
}

\newtcolorbox{solucionbox}{
    colback=white,
    colframe=verdeoscuro,
    fonttitle=\bfseries,
    title=Desarrollo de la Solución,
    sharp corners,
    boxrule=1pt,
    breakable
}

\newtcolorbox{resultadobox}{
    colback=rojoclaro,
    colframe=red!70!black,
    fonttitle=\bfseries,
    title=Resultado Final,
    sharp corners,
    boxrule=2pt
}

% Título y autor
\title{\textbf{Solución del Ejercicio 3.18} \\
\large Pistola de Bengalas en Utah y en la Luna}
\author{Física Universitaria - Sears y Zemansky \\ Capítulo 3: Movimiento en Dos o Tres Dimensiones}
\date{\today}

\begin{document}

\maketitle

\section{Enunciado del Problema}

Una pistola de bengalas en el desierto de Utah se dispara a un ángulo de 55° sobre la horizontal con velocidad inicial de 125 m/s. Si se hiciera un experimento similar en la superficie lunar, donde $g = 1.625$ m/s$^2$, ¿en qué factor cambiaría lo siguiente comparado con el resultado en la Tierra?

\begin{enumerate}
    \item[a)] El alcance horizontal.
    \item[b)] El tiempo de vuelo.
    \item[c)] La altura máxima.
\end{enumerate}

\vspace{0.5cm}

\begin{center}
\begin{tikzpicture}[scale=0.72]
    % TIERRA (Utah)
    \node[above] at (4,6) {\Large \textbf{EN LA TIERRA (Utah)}};

    % Suelo Tierra
    \draw[fill=brown!20, thick] (-1,0) rectangle (9,-0.7);
    \draw[pattern=north east lines, pattern color=brown] (-1,-0.7) rectangle (9,-0.9);

    % Punto de lanzamiento
    \filldraw[blue!70!black] (0,0) circle (0.12);
    \node[below left, font=\small] at (0,.7) {O};

    % Vector velocidad inicial y sus componentes
    \draw[-{Latex[length=2.5mm]},blue!70!black, ultra thick] (0,0) -- (1.2,1.72);
    \node[blue!70!black, above, font=\small] at (-0.7,1.2) {$\vec{v}_0=125$ m/s};

    % Componente horizontal
    \draw[-{Latex[length=2mm]},red, thick, dashed] (0,0) -- (1.2,0);
    \node[red, below, font=\scriptsize] at (1.7,0.08) {$v_{0x}=71.7$ m/s};

    % Componente vertical
    \draw[-{Latex[length=2mm]},red, thick, dashed] (1.2,0) -- (1.2,1.72);
    \node[red, right, font=\scriptsize] at (1.2,0.86) {$v_{0y}=102.4$ m/s};

    % Ángulo
    \draw[thick] (0.5,0) arc (0:55:0.5);
    \node[font=\small] at (0.8,0.35) {$55°$};

    % Trayectoria parabólica en Tierra
    % Ecuación: y = 1.428*x - 0.000953*x^2 (en metros)
    % Escala: 1 unidad = 200 m, entonces x_graf = x_m/200, y_graf = y_m/200
    % y_graf = 1.428*x_graf - 0.000953*40000*x_graf^2 = 1.428*x_graf - 38.12*x_graf^2
    % R_T = 1499 m = 7.495 unidades, h_max = 535 m = 2.675 unidades
    \draw[red!70!black, very thick, -{Latex[length=2.5mm]}]
        plot[domain=0:7.495, samples=100, smooth] (\x, {1.428*\x - 0.19*\x*\x});

    % Punto de altura máxima Tierra (x = 3.7475, y = 2.675)
    \filldraw[orange!80!black] (3.7475,2.675) circle (0.1);
    \node[orange!80!black, above right, font=\scriptsize] at (3.7475,2.675) {$h_{\text{máx}}=535$ m};
    \draw[orange!80!black, dotted, thick] (3.7475,2.675) -- (3.7475,0);

    % Alcance con valor
    \draw[{Latex[length=2mm]}-{Latex[length=2mm]}, thick] (0,-1.2) -- (7.495,-1.2);
    \node[below, font=\small] at (3.75,-1.2) {$R_T = 1499$ m $\approx 1.5$ km};

    % Etiqueta gravedad
    \node[below, font=\small] at (4,-2) {$g = 9.8$ m/s$^2$};

    % LUNA
    \node[above] at (15,6) {\Large \textbf{EN LA LUNA}};

    % Suelo Luna
    \draw[fill=gray!30, thick] (10,0) rectangle (20,-0.7);
    \draw[pattern=north east lines, pattern color=gray!50] (10,-0.7) rectangle (20,-0.9);

    % Punto de lanzamiento
    \filldraw[blue!70!black] (11,0) circle (0.12);
    \node[below left, font=\small] at (11,.7) {O};

    % Vector velocidad inicial y sus componentes
    \draw[-{Latex[length=2.5mm]},blue!70!black, ultra thick] (11,0) -- (12.2,1.72);
    \node[blue!70!black, above, font=\small] at (10.2,1.2) {$\vec{v}_0=125$ m/s};

    % Componente horizontal
    \draw[-{Latex[length=2mm]},red, thick, dashed] (11,0) -- (12.2,0);
    \node[red, below, font=\scriptsize] at (13,.08) {$v_{0x}=71.7$ m/s};

    % Componente vertical
    \draw[-{Latex[length=2mm]},red, thick, dashed] (12.2,0) -- (12.2,1.72);
    \node[red, right, font=\scriptsize] at (12.2,0.86) {$v_{0y}=102.4$ m/s};

    % Ángulo
    \draw[thick] (11.5,0) arc (0:55:0.5);
    \node[font=\small] at (11.8,0.35) {$55°$};

    % Trayectoria parabólica en Luna (6.03 veces más larga)
    % R_L = 9036 m = 9.036 unidades (en escala de 1000m/unidad)
    % Usando escala 1 unidad = 1000 m: x_graf = x_m/1000
    % y_graf = 1.428*x_graf - 0.000157*1000000*x_graf^2 = 1.428*x_graf - 157*x_graf^2
    % Mejor escala 1 unidad = 500 m
    % h_max_L = 3226 m, x_max = 4518 m
    % En escala 1:500: R_L = 18.072 unidades, pero eso es muy largo
    % Voy a usar escala diferente para Luna: 1 unidad = 1000 m
    % R_L = 9.036 unidades, h_max = 3.226 unidades
    \draw[red!70!black, very thick, -{Latex[length=2.5mm]}]
        plot[domain=0:9.036, samples=100, smooth] ({11 + \x}, {1.428*\x - 0.1577*\x*\x});

    % Punto de altura máxima Luna (x = 11 + 4.518, y = 3.226)
    \filldraw[orange!80!black] (15.518,3.226) circle (0.1);
    \node[orange!80!black, above, font=\scriptsize, align=center] at (15.518,3.2) {$h_{\text{máx}}=3226$ m\\$\approx 3.2$ km};
    \draw[orange!80!black, dotted, thick] (15.518,3.226) -- (15.518,0);

    % Alcance con valor
    \draw[{Latex[length=2mm]}-{Latex[length=2mm]}, thick] (11,-1.2) -- (20.036,-1.2);
    \node[below, font=\small] at (15.5,-1.2) {$R_L = 9036$ m $\approx 9.0$ km};

    % Etiqueta gravedad
    \node[below, font=\small] at (15,-2) {$g = 1.625$ m/s$^2$};

\end{tikzpicture}
\end{center}

\section{Datos del Problema}

\begin{datosbox}
\begin{itemize}
    \item \textbf{Velocidad inicial:} $v_0 = 125$ m/s (igual en ambos casos)
    \item \textbf{Ángulo de lanzamiento:} $\theta = 55°$ (igual en ambos casos)
    \item \textbf{Gravedad en la Tierra:} $g_T = 9.8$ m/s$^2$
    \item \textbf{Gravedad en la Luna:} $g_L = 1.625$ m/s$^2$
    \item \textbf{Factor:} $\frac{g_T}{g_L} = \frac{9.8}{1.625} = 6.03$
    \item \textbf{Resistencia del aire:} despreciable
\end{itemize}
\end{datosbox}

\section{Marco Teórico}

Para un proyectil lanzado con velocidad inicial $v_0$ a un ángulo $\theta$ desde el nivel del suelo en un lugar con gravedad $g$:

\textbf{Alcance horizontal:}
\begin{equation}
    R = \frac{v_0^2 \sin(2\theta)}{g}
\end{equation}

\textbf{Tiempo de vuelo:}
\begin{equation}
    t_{\text{vuelo}} = \frac{2v_0 \sin\theta}{g}
\end{equation}

\textbf{Altura máxima:}
\begin{equation}
    h_{\text{máx}} = \frac{v_0^2 \sin^2\theta}{2g}
\end{equation}

\textbf{Observación clave:} Todas estas cantidades son inversamente proporcionales a $g$. Por lo tanto:
\begin{equation}
    \text{Si } g_L < g_T \implies R_L > R_T, \quad t_L > t_T, \quad h_L > h_T
\end{equation}

\section{Desarrollo de la Solución}

\begin{solucionbox}

\subsection*{Análisis General}

La clave para resolver este problema es notar que las ecuaciones del movimiento de proyectiles dependen inversamente de $g$:

\begin{equation}
    R \propto \frac{1}{g}, \quad t \propto \frac{1}{g}, \quad h \propto \frac{1}{g}
\end{equation}

Por lo tanto, si la gravedad es menor (como en la Luna), todas estas cantidades aumentan.

El factor de cambio entre Luna y Tierra es:
\begin{equation}
    \text{Factor} = \frac{g_T}{g_L} = \frac{9.8}{1.625} = 6.03
\end{equation}

\subsection*{Parte a) Alcance horizontal}

\textbf{Alcance en la Tierra:}
\begin{align}
    R_T &= \frac{v_0^2 \sin(2\theta)}{g_T} \\
    R_T &= \frac{(125)^2 \sin(2 \times 55°)}{9.8} \\
    R_T &= \frac{15625 \times \sin(110°)}{9.8} \\
    R_T &= \frac{15625 \times 0.9397}{9.8} \\
    R_T &= \frac{14683}{9.8} \\
    R_T &= 1499 \text{ m} \approx 1.50 \text{ km}
\end{align}

\textbf{Alcance en la Luna:}
\begin{align}
    R_L &= \frac{v_0^2 \sin(2\theta)}{g_L} \\
    R_L &= \frac{(125)^2 \sin(110°)}{1.625} \\
    R_L &= \frac{14683}{1.625} \\
    R_L &= 9036 \text{ m} \approx 9.04 \text{ km}
\end{align}

\textbf{Factor de cambio:}
\begin{equation}
    \frac{R_L}{R_T} = \frac{9036}{1499} = \frac{g_T}{g_L} = \frac{9.8}{1.625} = 6.03
\end{equation}

El alcance en la Luna es \textbf{6.03 veces mayor} que en la Tierra.

\subsection*{Parte b) Tiempo de vuelo}

\textbf{Tiempo de vuelo en la Tierra:}

Primero calculamos la componente vertical de velocidad:
\begin{equation}
    v_{0y} = v_0 \sin\theta = 125 \times \sin(55°) = 125 \times 0.8192 = 102.4 \text{ m/s}
\end{equation}

El tiempo de vuelo es:
\begin{align}
    t_T &= \frac{2v_{0y}}{g_T} \\
    t_T &= \frac{2 \times 102.4}{9.8} \\
    t_T &= \frac{204.8}{9.8} \\
    t_T &= 20.9 \text{ s}
\end{align}

\textbf{Tiempo de vuelo en la Luna:}
\begin{align}
    t_L &= \frac{2v_{0y}}{g_L} \\
    t_L &= \frac{2 \times 102.4}{1.625} \\
    t_L &= \frac{204.8}{1.625} \\
    t_L &= 126.0 \text{ s}
\end{align}

\textbf{Factor de cambio:}
\begin{equation}
    \frac{t_L}{t_T} = \frac{126.0}{20.9} = \frac{g_T}{g_L} = \frac{9.8}{1.625} = 6.03
\end{equation}

El tiempo de vuelo en la Luna es \textbf{6.03 veces mayor} que en la Tierra.

\subsection*{Parte c) Altura máxima}

\textbf{Altura máxima en la Tierra:}
\begin{align}
    h_{T,\text{máx}} &= \frac{v_{0y}^2}{2g_T} \\
    h_{T,\text{máx}} &= \frac{(102.4)^2}{2 \times 9.8} \\
    h_{T,\text{máx}} &= \frac{10485.76}{19.6} \\
    h_{T,\text{máx}} &= 535 \text{ m}
\end{align}

\textbf{Altura máxima en la Luna:}
\begin{align}
    h_{L,\text{máx}} &= \frac{v_{0y}^2}{2g_L} \\
    h_{L,\text{máx}} &= \frac{(102.4)^2}{2 \times 1.625} \\
    h_{L,\text{máx}} &= \frac{10485.76}{3.25} \\
    h_{L,\text{máx}} &= 3226 \text{ m} \approx 3.23 \text{ km}
\end{align}

\textbf{Factor de cambio:}
\begin{equation}
    \frac{h_{L,\text{máx}}}{h_{T,\text{máx}}} = \frac{3226}{535} = \frac{g_T}{g_L} = \frac{9.8}{1.625} = 6.03
\end{equation}

La altura máxima en la Luna es \textbf{6.03 veces mayor} que en la Tierra.

\end{solucionbox}

\section{Resultados Finales}

\begin{resultadobox}
\textbf{Factor de cambio entre Luna y Tierra:}
\begin{equation*}
    \boxed{\text{Factor} = \frac{g_T}{g_L} = \frac{9.8}{1.625} = 6.03}
\end{equation*}

\vspace{0.5cm}

\textbf{a) Alcance horizontal:}
\begin{itemize}
    \item Tierra: $R_T = 1499$ m $\approx 1.50$ km
    \item Luna: $R_L = 9036$ m $\approx 9.04$ km
    \item \boxed{\text{Factor: } \frac{R_L}{R_T} = 6.03}
\end{itemize}

\vspace{0.3cm}

\textbf{b) Tiempo de vuelo:}
\begin{itemize}
    \item Tierra: $t_T = 20.9$ s
    \item Luna: $t_L = 126.0$ s
    \item \boxed{\text{Factor: } \frac{t_L}{t_T} = 6.03}
\end{itemize}

\vspace{0.3cm}

\textbf{c) Altura máxima:}
\begin{itemize}
    \item Tierra: $h_{T,\text{máx}} = 535$ m
    \item Luna: $h_{L,\text{máx}} = 3226$ m $\approx 3.23$ km
    \item \boxed{\text{Factor: } \frac{h_{L,\text{máx}}}{h_{T,\text{máx}}} = 6.03}
\end{itemize}
\end{resultadobox}

\section{Verificación}

Podemos verificar que todas las cantidades cambian por el mismo factor:

\subsection*{Relación fundamental}

Como las ecuaciones son:
\begin{align}
    R &= \frac{v_0^2 \sin(2\theta)}{g} \\
    t &= \frac{2v_0 \sin\theta}{g} \\
    h &= \frac{v_0^2 \sin^2\theta}{2g}
\end{align}

Todas son inversamente proporcionales a $g$. Por lo tanto:
\begin{equation}
    \frac{R_L}{R_T} = \frac{t_L}{t_T} = \frac{h_L}{h_T} = \frac{g_T}{g_L} = 6.03 \quad \checkmark
\end{equation}

\subsection*{Verificación numérica}

\textbf{Alcance:}
\begin{equation*}
    \frac{R_L}{R_T} = \frac{9036}{1499} = 6.03 \quad \checkmark
\end{equation*}

\textbf{Tiempo:}
\begin{equation*}
    \frac{t_L}{t_T} = \frac{126.0}{20.9} = 6.03 \quad \checkmark
\end{equation*}

\textbf{Altura:}
\begin{equation*}
    \frac{h_L}{h_T} = \frac{3226}{535} = 6.03 \quad \checkmark
\end{equation*}

\section{Análisis Físico}

\subsection{Interpretación de los resultados}

\begin{enumerate}
    \item \textbf{Menor gravedad $\implies$ Mayor rendimiento:} En la Luna, la gravedad es aproximadamente 6 veces menor que en la Tierra, lo que hace que la bengala:
    \begin{itemize}
        \item Vuele 6 veces más lejos
        \item Permanezca en el aire 6 veces más tiempo
        \item Alcance una altura 6 veces mayor
    \end{itemize}

    \item \textbf{Proporcionalidad inversa:} Todos los parámetros del movimiento de proyectiles son inversamente proporcionales a la gravedad cuando $v_0$ y $\theta$ permanecen constantes.

    \item \textbf{Mismo factor para todo:} El factor $\frac{g_T}{g_L} = 6.03$ se aplica universalmente a alcance, tiempo y altura.
\end{enumerate}

\subsection{Comparación visual}

\begin{center}
\begin{tikzpicture}[scale=0.9]
\begin{axis}[
    width=15cm,
    height=9cm,
    xlabel={Distancia horizontal (km)},
    ylabel={Altura (km)},
    xmin=0, xmax=10,
    ymin=0, ymax=3.5,
    grid=major,
    grid style={dashed,gray!30},
    legend pos=north east
]

% Trayectoria en Tierra (usando x en km)
\addplot[
    red!70!black,
    very thick,
    domain=0:1.5,
    samples=100,
    smooth
] {1.428*x - 0.0348*x^2};
\addlegendentry{Tierra ($g = 9.8$ m/s$^2$)}

% Trayectoria en Luna (usando x en km)
\addplot[
    blue!70!black,
    very thick,
    domain=0:9.04,
    samples=100,
    smooth
] {1.428*x - 0.00577*x^2};
\addlegendentry{Luna ($g = 1.625$ m/s$^2$)}

\end{axis}
\end{tikzpicture}
\end{center}

\subsection{Implicaciones prácticas}

Si se hiciera este experimento en diferentes cuerpos celestes:

\begin{center}
\begin{tabular}{|l|c|c|c|c|}
\hline
\textbf{Cuerpo} & \textbf{$g$ (m/s$^2$)} & \textbf{Alcance (km)} & \textbf{Tiempo (s)} & \textbf{Altura (m)} \\
\hline
Tierra & 9.8 & 1.50 & 20.9 & 535 \\
Luna & 1.625 & 9.04 & 126.0 & 3226 \\
Marte & 3.71 & 3.96 & 55.2 & 1413 \\
Júpiter & 24.8 & 0.59 & 8.3 & 211 \\
\hline
\end{tabular}
\end{center}

\subsection{Curiosidades}

\begin{enumerate}
    \item \textbf{Récord lunar:} En la Luna, la bengala alcanzaría más de 3 km de altura, ¡equivalente a 10 edificios del Empire State apilados!

    \item \textbf{Tiempo de vuelo:} La bengala permanecería en el aire más de 2 minutos en la Luna, comparado con solo 21 segundos en la Tierra.

    \item \textbf{Alcance extraordinario:} Los 9 km de alcance en la Luna son comparables a la distancia de una carrera de 10K.

    \item \textbf{Deportes lunares:} Los astronautas del Apollo experimentaron con el lanzamiento de objetos en la Luna. Un martillo geológico lanzado podría volar cientos de metros.
\end{enumerate}

\section{Fórmula General}

\begin{tcolorbox}[colback=yellow!10!white,colframe=orange!75!black,title=Relación entre Gravedades]

Si el mismo proyectil se lanza con la misma velocidad inicial y ángulo en dos lugares con diferentes gravedades:

\textbf{Factor de cambio:}
\begin{equation}
    \text{Factor} = \frac{g_1}{g_2}
\end{equation}

\textbf{Aplicaciones:}
\begin{align}
    R_2 &= R_1 \times \frac{g_1}{g_2} \\
    t_2 &= t_1 \times \frac{g_1}{g_2} \\
    h_2 &= h_1 \times \frac{g_1}{g_2}
\end{align}

\textbf{Para este problema (Luna vs. Tierra):}
\begin{equation*}
    \text{Factor} = \frac{g_T}{g_L} = \frac{9.8}{1.625} = 6.03
\end{equation*}

\textbf{Resultado:} Todas las cantidades (alcance, tiempo, altura) se multiplican por 6.03 en la Luna.

\end{tcolorbox}

\section{Gráficas Comparativas}

\subsection{Alcance vs. Gravedad}

\begin{center}
\begin{tikzpicture}
\begin{axis}[
    width=12cm,
    height=7cm,
    xlabel={Gravedad $g$ (m/s$^2$)},
    ylabel={Alcance $R$ (km)},
    xmin=0, xmax=12,
    ymin=0, ymax=10,
    grid=major,
    grid style={dashed,gray!30},
    legend pos=north east
]

% R = k/g donde k = v0^2 * sin(2θ)
\addplot[
    red!70!black,
    very thick,
    domain=1:11,
    samples=100,
    smooth
] {14683/(x*1000)};
\addlegendentry{$R = \frac{v_0^2 \sin(2\theta)}{g}$}

% Punto Luna
\addplot[
    mark=*,
    mark size=3pt,
    blue!70!black,
    only marks
] coordinates {(1.625, 9.04)};
\addlegendentry{Luna}

% Punto Tierra
\addplot[
    mark=*,
    mark size=3pt,
    red!70!black,
    only marks
] coordinates {(9.8, 1.50)};
\addlegendentry{Tierra}

\end{axis}
\end{tikzpicture}
\end{center}

\subsection{Tiempo de vuelo vs. Gravedad}

\begin{center}
\begin{tikzpicture}
\begin{axis}[
    width=12cm,
    height=7cm,
    xlabel={Gravedad $g$ (m/s$^2$)},
    ylabel={Tiempo de vuelo $t$ (s)},
    xmin=0, xmax=12,
    ymin=0, ymax=140,
    grid=major,
    grid style={dashed,gray!30},
    legend pos=north east
]

% t = k/g donde k = 2*v0*sin(θ)
\addplot[
    green!50!black,
    very thick,
    domain=1:11,
    samples=100,
    smooth
] {204.8/x};
\addlegendentry{$t = \frac{2v_0\sin\theta}{g}$}

% Punto Luna
\addplot[
    mark=*,
    mark size=3pt,
    blue!70!black,
    only marks
] coordinates {(1.625, 126.0)};
\addlegendentry{Luna}

% Punto Tierra
\addplot[
    mark=*,
    mark size=3pt,
    red!70!black,
    only marks
] coordinates {(9.8, 20.9)};
\addlegendentry{Tierra}

\end{axis}
\end{tikzpicture}
\end{center}

\subsection{Altura máxima vs. Gravedad}

\begin{center}
\begin{tikzpicture}
\begin{axis}[
    width=12cm,
    height=7cm,
    xlabel={Gravedad $g$ (m/s$^2$)},
    ylabel={Altura máxima $h$ (km)},
    xmin=0, xmax=12,
    ymin=0, ymax=3.5,
    grid=major,
    grid style={dashed,gray!30},
    legend pos=north east
]

% h = k/g donde k = v0^2*sin^2(θ)/2
\addplot[
    orange!70!black,
    very thick,
    domain=1:11,
    samples=100,
    smooth
] {10485.76/(2*x*1000)};
\addlegendentry{$h = \frac{v_0^2\sin^2\theta}{2g}$}

% Punto Luna
\addplot[
    mark=*,
    mark size=3pt,
    blue!70!black,
    only marks
] coordinates {(1.625, 3.226)};
\addlegendentry{Luna}

% Punto Tierra
\addplot[
    mark=*,
    mark size=3pt,
    red!70!black,
    only marks
] coordinates {(9.8, 0.535)};
\addlegendentry{Tierra}

\end{axis}
\end{tikzpicture}
\end{center}

\section{Conclusiones}

\begin{enumerate}
    \item Todas las cantidades del movimiento de proyectiles son inversamente proporcionales a la gravedad cuando las condiciones iniciales permanecen constantes.

    \item La relación entre Luna y Tierra es exactamente $\frac{g_T}{g_L} = 6.03$ para alcance, tiempo y altura.

    \item La gravedad lunar, siendo 6 veces menor que la terrestre, permite que los proyectiles viajen 6 veces más lejos, permanezcan 6 veces más tiempo en el aire, y alcancen 6 veces mayor altura.

    \item Este principio se aplica a cualquier par de cuerpos celestes con diferentes gravedades.

    \item Los deportes y actividades que involucren lanzamiento de objetos tendrían características completamente diferentes en la Luna.
\end{enumerate}

\end{document}
