\documentclass[11pt,a4paper]{article}

% Paquetes necesarios
\usepackage[utf8]{inputenc}
\usepackage[T1]{fontenc}
\usepackage[spanish]{babel}
\usepackage[margin=2.5cm]{geometry}
\usepackage{amsmath}
\usepackage{amssymb}
\usepackage{xcolor}
\usepackage{tcolorbox}
\usepackage{graphicx}
\usepackage{tikz}
\usepackage{pgfplots}
\pgfplotsset{compat=1.18}
\usetikzlibrary{arrows.meta,patterns,decorations.pathmorphing,calc}

\usepackage{wrapfig}
\usepackage[export]{adjustbox} % (opcional) claves extra para \includegraphics
\usepackage{xparse}
\NewDocumentCommand{\figuratikz}{ O{l} O{0.5} m m m m}{%
	% #1 = posición (l o r), default: l
	% #2 = ancho como fracción (0.25, 0.5, etc.), default: 0.5
	% #3 = entorno TIKZ
	% #3 = caption
		\begin{wrapfigure}{#1}{#2\textwidth}
		\centering
		\vspace{-\baselineskip} % Ajuste vertical más suave
		{3}
		\caption{#4}
		\vspace{-\baselineskip} % Reduce espacio después
		\end{wrapfigure}
	}

% Definición de colores
\definecolor{azuloscuro}{RGB}{0,51,102}
\definecolor{azulclaro}{RGB}{230,240,250}
\definecolor{verdeoscuro}{RGB}{0,100,0}
\definecolor{rojoclaro}{RGB}{255,230,230}

% Configuración de cajas
\tcbuselibrary{theorems,skins,breakable}

\newtcolorbox{datosbox}{
    colback=azulclaro,
    colframe=azuloscuro,
    fonttitle=\bfseries,
    title=Datos del Problema,
    sharp corners,
    boxrule=1pt
}

\newtcolorbox{solucionbox}{
    colback=white,
    colframe=verdeoscuro,
    fonttitle=\bfseries,
    title=Desarrollo de la Solución,
    sharp corners,
    boxrule=1pt,
    breakable
}

\newtcolorbox{resultadobox}{
    colback=rojoclaro,
    colframe=red!70!black,
    fonttitle=\bfseries,
    title=Resultado Final,
    sharp corners,
    boxrule=2pt
}

% Título y autor
\title{\textbf{Solución del Ejercicio 3.10} \\
\large Movimiento de Proyectiles: Bomba desde Helicóptero}
\author{Física Universitaria - Sears y Zemansky \\ Capítulo 3: Movimiento en Dos o Tres Dimensiones}
\date{\today}

\begin{document}

\maketitle

\section{Enunciado del Problema}


\begin{wrapfigure}[10]{r}{.6\textwidth}  % Ajusta el ancho según necesites
	\centering
	\vspace{-\baselineskip}
	%\begin{center}
\begin{tikzpicture}[scale=0.015]
    % Cielo
    \fill[blue!10] (-50,0) rectangle (550,320);

    % Helicóptero inicial
    \draw[fill=gray!60, thick] (0,300) ellipse (20 and 5);
    \draw[fill=gray!70, thick] (-5,300) rectangle (5,310);
    \draw[thick] (-15,312) -- (15,312);
    \node at (45,270) {\small Helicóptero};

    % Flecha de velocidad del helicóptero
    \draw[-{Latex[length=5mm]},red!70!black, ultra thick] (20,305) -- (60,305)
        node[midway,above] {$v_0 = 60.0$ m/s};

    % Bomba cayendo
    \filldraw[red!70!black] (0,300) circle (3);

    % Trayectoria parabólica de la bomba (ecuación física real: y = y0 - (g/(2v0^2))*x^2)
    % Para v0 = 60.0 m/s, g = 9.8 m/s^2: coeficiente = 0.00136
    \draw[blue!70!black, thick, dashed, -{Latex[length=3mm]}]
        plot[domain=0:470, samples=100, smooth] (\x, {300 - 0.00136*(\x)*(\x)});

    % Helicóptero al final
    \draw[fill=gray!40, thick, dashed] (470,300) ellipse (20 and 5);
    \draw[fill=gray!50, thick, dashed] (465,300) rectangle (475,310);
    \draw[thick, dashed] (455,312) -- (485,312);

    % Línea punteada vertical
    \draw[dotted, thick] (470,0) -- (470,300);

    % Suelo
    \draw[fill=green!40!brown, thick] (-50,0) rectangle (550,-10);
    \draw[pattern=north east lines, pattern color=brown!50] (-50,-10) rectangle (550,-20);

    % Cotas
    \draw[{Latex[length=2mm]}-{Latex[length=2mm]}, thick]
        (-30,0) -- (-30,300) node[midway,right] {300 m};

    \draw[{Latex[length=2mm]}-{Latex[length=2mm]}, thick]
        (0,-40) -- (470,-40) node[midway,below] {$d = ?$};

    % Etiquetas
    \node[blue!70!black] at (235,150) {Trayectoria de la bomba};
    \node at (470,326) {\small Posición final del helicóptero};

\end{tikzpicture}
%\end{center}
\vspace{-\baselineskip} % Reduce espacio después
\end{wrapfigure}
Un helicóptero militar está en una misión de entrenamiento y vuela horizontalmente con una rapidez de 60.0 m/s y accidentalmente suelta una bomba (desactivada, por suerte) a una altitud de 300 m. Puede despreciarse la resistencia del aire.

\begin{enumerate}
	\item[a)] ¿Qué tiempo tarda la bomba en llegar al suelo?
	\item[b)] ¿Qué distancia horizontal viaja mientras cae?
\end{enumerate}

\begin{enumerate}
	\item[c)] Obtenga las componentes horizontal y vertical de su velocidad justo antes de llegar al suelo.
	\item[d)] Dibuje gráficas $x$-$t$, $y$-$t$, $v_x$-$t$ y $v_y$-$t$ para el movimiento de la bomba.
	\item[e)] ¿Dónde está el helicóptero cuando la bomba toca tierra, si la rapidez del helicóptero se mantuvo constante?
\end{enumerate}

\section{Datos del Problema}

\begin{datosbox}
\begin{itemize}
    \item \textbf{Velocidad del helicóptero:} $v_0 = 60.0$ m/s (horizontal)
    \item \textbf{Altitud inicial:} $h = 300$ m
    \item \textbf{Velocidad inicial vertical de la bomba:} $v_{0y} = 0$
    \item \textbf{Aceleración de la gravedad:} $g = 9.8$ m/s$^2$
    \item \textbf{Resistencia del aire:} despreciable
\end{itemize}
\end{datosbox}

\section{Marco Teórico}

\subsection{Ecuaciones del Movimiento de Proyectiles}

Para un objeto lanzado horizontalmente desde una altura $h$:

\textbf{Movimiento horizontal:}
\begin{align}
    x &= v_0 t \\
    v_x &= v_0 = \text{constante}
\end{align}

\textbf{Movimiento vertical:}
\begin{align}
    y &= h - \frac{1}{2}gt^2 \\
    v_y &= -gt
\end{align}

\section{Desarrollo de la Solución}

\begin{solucionbox}

\subsection*{Parte a) Tiempo de caída}

La bomba cae desde una altura de 300 m hasta el suelo.

Usando la ecuación del movimiento vertical:
\begin{align*}
    y &= h - \frac{1}{2}gt^2 \\
    0 &= 300 - \frac{1}{2}(9.8)t^2 \\
    0 &= 300 - 4.9t^2 \\
    4.9t^2 &= 300 \\
    t^2 &= \frac{300}{4.9} \\
    t^2 &= 61.22 \\
    t &= \sqrt{61.22} \\
    t &= 7.82 \text{ s}
\end{align*}

También podemos usar la fórmula directa:
\begin{equation*}
    t = \sqrt{\frac{2h}{g}} = \sqrt{\frac{2(300)}{9.8}} = \sqrt{61.22} = 7.82 \text{ s}
\end{equation*}

\textbf{Tiempo de caída:} $t = 7.82$ s

\subsection*{Parte b) Distancia horizontal}

La bomba mantiene la velocidad horizontal del helicóptero (60.0 m/s) durante toda su caída:

\begin{align*}
    d &= v_0 t \\
    d &= (60.0)(7.82) \\
    d &= 469.2 \text{ m} \\
    d &\approx 469 \text{ m}
\end{align*}

\textbf{Distancia horizontal:} $d = 469$ m

\subsection*{Parte c) Componentes de velocidad al impactar}

\subsubsection*{Componente horizontal:}
\begin{equation*}
    v_x = v_0 = 60.0 \text{ m/s}
\end{equation*}

\subsubsection*{Componente vertical:}
\begin{align*}
    v_y &= -gt \\
    v_y &= -(9.8)(7.82) \\
    v_y &= -76.6 \text{ m/s}
\end{align*}

(El signo negativo indica movimiento hacia abajo)

También podemos calcularla usando:
\begin{align*}
    v_y &= -\sqrt{2gh} \\
    v_y &= -\sqrt{2(9.8)(300)} \\
    v_y &= -\sqrt{5880} \\
    v_y &= -76.7 \text{ m/s}
\end{align*}

\subsubsection*{Magnitud de la velocidad:}
\begin{align*}
    v &= \sqrt{v_x^2 + v_y^2} \\
    v &= \sqrt{(60.0)^2 + (76.6)^2} \\
    v &= \sqrt{3600 + 5868} \\
    v &= \sqrt{9468} \\
    v &= 97.3 \text{ m/s}
\end{align*}

\subsubsection*{Dirección:}
\begin{align*}
    \theta &= \arctan\left(\frac{|v_y|}{v_x}\right) \\
    \theta &= \arctan\left(\frac{76.6}{60.0}\right) \\
    \theta &= \arctan(1.277) \\
    \theta &= 51.9°
\end{align*}

La bomba impacta con un ángulo de 51.9° por debajo de la horizontal.

\subsection*{Parte e) Posición del helicóptero}

Si el helicóptero mantiene su velocidad constante de 60.0 m/s, en 7.82 s habrá recorrido:

\begin{align*}
    d_h &= v_0 t \\
    d_h &= (60.0)(7.82) \\
    d_h &= 469 \text{ m}
\end{align*}

\textbf{El helicóptero está directamente encima del punto de impacto de la bomba}, a 300 m de altura.

Esto se debe a que tanto el helicóptero como la bomba tienen la misma velocidad horizontal, por lo que la bomba siempre está directamente debajo del helicóptero durante la caída.

\end{solucionbox}

\section{Resultados Finales}

\begin{resultadobox}

\subsection*{Parte a) Tiempo de caída}
\begin{equation}
    \boxed{t = 7.82 \text{ s}}
\end{equation}

\subsection*{Parte b) Distancia horizontal}
\begin{equation}
    \boxed{d = 469 \text{ m}}
\end{equation}

\subsection*{Parte c) Velocidad al impactar}

\textbf{Componentes:}
\begin{itemize}
    \item Horizontal: $v_x = 60.0$ m/s
    \item Vertical: $v_y = 76.6$ m/s (hacia abajo)
\end{itemize}

\textbf{Magnitud:}
\begin{equation}
    \boxed{v = 97.3 \text{ m/s}}
\end{equation}

\textbf{Dirección:}
\begin{equation}
    \boxed{\theta = 51.9° \text{ por debajo de la horizontal}}
\end{equation}

\subsection*{Parte e) Posición del helicóptero}

\begin{equation}
    \boxed{\text{Directamente encima del punto de impacto, a 300 m de altura}}
\end{equation}

\end{resultadobox}

\section{Parte d) Gráficas del Movimiento}

\subsection{Gráfica $x$ vs $t$}

\begin{center}
\begin{tikzpicture}
\begin{axis}[
    width=0.85\textwidth,
    height=7cm,
    xlabel={Tiempo $t$ (s)},
    ylabel={Posición horizontal $x$ (m)},
    xmin=0, xmax=8.5,
    ymin=0, ymax=500,
    grid=major,
    title={Posición horizontal vs tiempo}
]
\addplot[blue, thick, domain=0:7.82] {60*x};
\addplot[red, only marks, mark=*] coordinates {(7.82,469.2)};
\node[above right] at (axis cs:7.82,469.2) {$(7.82, 469)$};
\end{axis}
\end{tikzpicture}
\end{center}

\subsection{Gráfica $y$ vs $t$}

\begin{center}
\begin{tikzpicture}
\begin{axis}[
    width=0.85\textwidth,
    height=7cm,
    xlabel={Tiempo $t$ (s)},
    ylabel={Altura $y$ (m)},
    xmin=0, xmax=8.5,
    ymin=0, ymax=350,
    grid=major,
    title={Altura vs tiempo}
]
\addplot[blue, thick, domain=0:7.82] {300 - 4.9*x^2};
\addplot[red, only marks, mark=*] coordinates {(7.82,0)};
\node[below right] at (axis cs:7.82,0) {$(7.82, 0)$};
\node[above left] at (axis cs:0,300) {Altitud inicial: 300 m};
\end{axis}
\end{tikzpicture}
\end{center}

\subsection{Gráfica $v_x$ vs $t$}

\begin{center}
\begin{tikzpicture}
\begin{axis}[
    width=0.85\textwidth,
    height=7cm,
    xlabel={Tiempo $t$ (s)},
    ylabel={Velocidad horizontal $v_x$ (m/s)},
    xmin=0, xmax=8.5,
    ymin=0, ymax=70,
    grid=major,
    title={Velocidad horizontal vs tiempo}
]
\addplot[blue, thick, domain=0:7.82] {60};
\addplot[red, only marks, mark=*] coordinates {(7.82,60)};
\node[above] at (axis cs:4,60) {$v_x = 60.0$ m/s (constante)};
\end{axis}
\end{tikzpicture}
\end{center}

\subsection{Gráfica $v_y$ vs $t$}

\begin{center}
\begin{tikzpicture}
\begin{axis}[
    width=0.85\textwidth,
    height=7cm,
    xlabel={Tiempo $t$ (s)},
    ylabel={Velocidad vertical $v_y$ (m/s)},
    xmin=0, xmax=8.5,
    ymin=-85, ymax=5,
    grid=major,
    title={Velocidad vertical vs tiempo}
]
\addplot[blue, thick, domain=0:7.82] {-9.8*x};
\addplot[red, only marks, mark=*] coordinates {(7.82,-76.6)};
\node[below right] at (axis cs:7.82,-76.6) {$(7.82, -76.6)$};
\end{axis}
\end{tikzpicture}
\end{center}

\section{Verificación}

\subsection{Verificación con energía}

Usando conservación de energía mecánica:
\begin{align*}
    \frac{1}{2}mv_f^2 &= \frac{1}{2}mv_0^2 + mgh \\
    v_f^2 &= v_0^2 + 2gh \\
    v_f^2 &= (60.0)^2 + 2(9.8)(300) \\
    v_f^2 &= 3600 + 5880 \\
    v_f^2 &= 9480 \\
    v_f &= 97.4 \text{ m/s} \quad \checkmark
\end{align*}

\subsection{Verificación de la distancia}

\begin{align*}
    d &= v_0 \sqrt{\frac{2h}{g}} \\
    d &= 60.0 \sqrt{\frac{2(300)}{9.8}} \\
    d &= 60.0 \sqrt{61.22} \\
    d &= 60.0 \times 7.82 \\
    d &= 469 \text{ m} \quad \checkmark
\end{align*}

\section{Conceptos Clave}

\begin{enumerate}
    \item \textbf{Herencia de velocidad:} La bomba hereda la velocidad horizontal del helicóptero en el momento de soltarse.

    \item \textbf{Independencia de movimientos:} El movimiento horizontal (uniforme) es independiente del vertical (acelerado).

    \item \textbf{Posición relativa:} Durante toda la caída, la bomba permanece directamente debajo del helicóptero.

    \item \textbf{Alta velocidad de impacto:} La bomba impacta a 97.3 m/s (≈ 350 km/h), debido a la combinación de las velocidades horizontal y vertical.

    \item \textbf{Gran alcance horizontal:} La altura de 300 m permite un tiempo de caída de casi 8 segundos, resultando en un alcance de casi medio kilómetro.
\end{enumerate}

\section{Análisis Adicional}

\subsection{¿Por qué el helicóptero está encima del impacto?}

Este es un resultado contraintuitivo pero importante. La bomba y el helicóptero tienen la misma velocidad horizontal (60.0 m/s) durante toda la caída. Por lo tanto:

\begin{itemize}
    \item Ambos recorren la misma distancia horizontal en el mismo tiempo
    \item La bomba está siempre directamente debajo del helicóptero
    \item Desde el helicóptero, parecería que la bomba cae verticalmente
\end{itemize}

\subsection{Conversión de unidades}

\textbf{Velocidad del helicóptero:}
\begin{equation*}
    60.0 \text{ m/s} \times 3.6 = 216 \text{ km/h}
\end{equation*}

\textbf{Velocidad de impacto:}
\begin{equation*}
    97.3 \text{ m/s} \times 3.6 = 350 \text{ km/h}
\end{equation*}

\section{Resumen de Fórmulas}

\begin{tcolorbox}[colback=yellow!10!white,colframe=orange!75!black,title=Fórmulas Clave para este Problema]

\textbf{Tiempo de caída desde altura $h$:}
\begin{equation*}
    t = \sqrt{\frac{2h}{g}}
\end{equation*}

\textbf{Alcance horizontal:}
\begin{equation*}
    d = v_0 t = v_0\sqrt{\frac{2h}{g}}
\end{equation*}

\textbf{Velocidad vertical al impactar:}
\begin{equation*}
    v_y = \sqrt{2gh}
\end{equation*}

\textbf{Velocidad total al impactar:}
\begin{equation*}
    v = \sqrt{v_0^2 + 2gh}
\end{equation*}

\textbf{Para este problema:}
\begin{align*}
    h &= 300 \text{ m}, \quad v_0 = 60.0 \text{ m/s} \\
    t &= 7.82 \text{ s}, \quad d = 469 \text{ m} \\
    v &= 97.3 \text{ m/s}, \quad \theta = 51.9°
\end{align*}

\end{tcolorbox}

\end{document}
