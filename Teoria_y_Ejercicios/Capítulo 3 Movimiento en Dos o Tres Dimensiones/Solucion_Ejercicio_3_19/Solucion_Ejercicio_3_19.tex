\documentclass[11pt,a4paper]{article}

% Paquetes necesarios
\usepackage[utf8]{inputenc}
\usepackage[T1]{fontenc}
\usepackage[spanish]{babel}
\usepackage[margin=2.5cm]{geometry}
\usepackage{amsmath}
\usepackage{amssymb}
\usepackage{xcolor}
\usepackage{tcolorbox}
\usepackage{graphicx}
\usepackage{tikz}
\usepackage{pgfplots}
\pgfplotsset{compat=1.18}
\usetikzlibrary{arrows.meta,patterns,decorations.pathmorphing,calc}
\usepackage{wrapfig}
\usepackage[export]{adjustbox} % (opcional) claves extra para \includegraphics
\usepackage{xparse}

% Definición de colores
\definecolor{azuloscuro}{RGB}{0,51,102}
\definecolor{azulclaro}{RGB}{230,240,250}
\definecolor{verdeoscuro}{RGB}{0,100,0}
\definecolor{rojoclaro}{RGB}{255,230,230}

% Configuración de cajas
\tcbuselibrary{theorems,skins,breakable}

\newtcolorbox{datosbox}{
    colback=azulclaro,
    colframe=azuloscuro,
    fonttitle=\bfseries,
    title=Datos del Problema,
    sharp corners,
    boxrule=1pt
}

\newtcolorbox{solucionbox}{
    colback=white,
    colframe=verdeoscuro,
    fonttitle=\bfseries,
    title=Desarrollo de la Solución,
    sharp corners,
    boxrule=1pt,
    breakable
}

\newtcolorbox{resultadobox}{
    colback=rojoclaro,
    colframe=red!70!black,
    fonttitle=\bfseries,
    title=Resultado Final,
    sharp corners,
    boxrule=2pt
}

% Título y autor
\title{\textbf{Solución del Ejercicio 3.19} \\
\large Pelota de Béisbol Bateada}
\author{Física Universitaria - Sears y Zemansky \\ Capítulo 3: Movimiento en Dos o Tres Dimensiones}
\date{\today}

\begin{document}

\maketitle

\section{Enunciado del Problema}

\begin{wrapfigure}[7]{r}{.65\textwidth}  % Ajusta el ancho según necesites
	\centering
	\vspace{-\baselineskip}
	%\begin{center}
\begin{tikzpicture}[scale=.85]
    % Escala: 1 unidad = 10 m
    % R = 88.2 m → 8.82 unidades
    % h_max = 16.5 m → 1.65 unidades en x = 4.41 unidades

    % Suelo (campo de béisbol)
    \draw[fill=green!20, thick] (-0.5,0) rectangle (9.5,-0.6);
    \draw[pattern=north east lines, pattern color=green!50] (-0.5,-0.6) rectangle (9.5,-0.8);

    % Punto de lanzamiento (bateador)
    \filldraw[blue!70!black] (0,0) circle (0.12);
    \node[below left, font=\small] at (0,0.5) {O};

    % Vector velocidad inicial y componentes
    \draw[-{Latex[length=3mm]},blue!70!black, ultra thick] (0,0) -- (1.2,0.9);
    \node[blue!70!black, above left, font=\small] at (0.75,0.6) {$\vec{v}_0=30$ m/s};

    % Componente horizontal
    \draw[-{Latex[length=2mm]},red, thick, dashed] (0,0) -- (1.2,0);
    \node[red, below, font=\scriptsize] at (1.3,0) {$v_{0x}=24$ m/s};

    % Componente vertical
    \draw[-{Latex[length=2mm]},red, thick, dashed] (1.2,0) -- (1.2,0.9);
    \node[red, right, font=\scriptsize] at (1.2,0.45) {$v_{0y}=18$ m/s};

    % Ángulo
    \draw[thick] (0.6,0) arc (0:36.9:0.6);
    \node[font=\small] at (0.95,0.22) {$36.9°$};

    % Trayectoria parabólica correcta
    % Ecuación: y = 0.75*x - 0.0085*x^2 (en metros)
    % Coeficiente: b = g/(2*v0x^2) = 9.8/(2*24^2) = 9.8/1152 = 0.0085
    % Con escala 1:10, donde x_graf = x_m/10 y y_graf = y_m/10:
    % y_m = 0.75*(10*x_graf) - 0.0085*(10*x_graf)^2 = 7.5*x_graf - 0.85*x_graf^2
    % y_graf = y_m/10 = 0.75*x_graf - 0.085*x_graf^2
    \draw[red!70!black, very thick, -{Latex[length=3mm]}]
        plot[domain=0:8.82, samples=100, smooth] (\x, {0.75*\x - 0.085*\x*\x});

    % Punto de altura máxima (x=44.1m=4.41 unid, y=16.5m=1.65 unid)
    \filldraw[green!50!black] (4.41,1.65) circle (0.1);
    \node[green!50!black, above right, font=\small] at (2,1.65) {$h_{\text{máx}}=16.5$ m};
    \draw[green!50!black, dotted, thick] (4.41,1.65) -- (4.41,0);

    % Punto en t=2.00s (x=48m=4.8 unid, y=16.4m=1.64 unid)
    \filldraw[orange!70!black] (4.8,1.64) circle (0.1);
    \node[orange!70!black, above, font=\small] at (5.6,1.65) {$t_{s}=2.00$ s};
    \node[orange!70!black, below, font=\scriptsize] at (5.5,1.2) {$(48, 16.4)$ m};

    % Vector velocidad en t=2s (casi horizontal, ligeramente hacia abajo)
    \draw[-{Latex[length=2.5mm]}, orange!70!black, thick] (4.8,1.64) -- (5.8,1.56);
    \node[orange!70!black, right, font=\scriptsize] at (4.6,1.2) {$\vec{v}(2.00)$};

    % Punto final
    \filldraw[blue!50!black] (8.82,0) circle (0.08);

    % Alcance horizontal con valor correcto
    \draw[{Latex[length=2mm]}-{Latex[length=2mm]}, thick] (0,-0.9) -- (8.82,-0.9);
    \node[below, font=\small] at (4.41,-0.9) {$R = 88.2$ m};

    % Etiqueta inferior
    \node[below, font=\small] at (4.5,-1.5) {Campo de béisbol};

    % Etiqueta "Bateador"
    \node[below, font=\small] at (0,-0.9) {Bateador};

\end{tikzpicture}
%\end{center}
\vspace{-\baselineskip} % Reduce espacio después
\end{wrapfigure}
Se batea una pelota de béisbol de modo que su velocidad inicial es de $v_0 = 30.0$ m/s a un ángulo de $\theta_0 = 36.9°$ sobre la horizontal. Calcule:

\begin{enumerate}
	\item[a)] La posición de la pelota y la magnitud y dirección de su velocidad cuando $t = 2.00$ s.
\end{enumerate}

\begin{enumerate}
	\item[b)] El tiempo que tarda la pelota en llegar al punto más alto de su trayectoria.
	\item[c)] La altura máxima de la pelota.
	\item[d)] El alcance horizontal de la pelota.
	\item[e)] Los componentes de aceleración de la pelota durante su vuelo.
\end{enumerate}

\section{Datos del Problema}

\begin{datosbox}
\begin{itemize}
    \item \textbf{Velocidad inicial:} $v_0 = 30.0$ m/s
    \item \textbf{Ángulo de lanzamiento:} $\theta_0 = 36.9°$
    \item \textbf{Aceleración gravitacional:} $g = 9.8$ m/s$^2$
    \item \textbf{Posición inicial:} $x_0 = 0$, $y_0 = 0$ (nivel del suelo)
    \item \textbf{Resistencia del aire:} despreciable
    \item \textbf{Tiempo específico:} $t = 2.00$ s
\end{itemize}
\end{datosbox}

\section{Marco Teórico}

Para un proyectil lanzado desde el nivel del suelo con velocidad inicial $v_0$ a un ángulo $\theta_0$:

\textbf{Componentes de velocidad inicial:}
\begin{align*}
    v_{0x} &= v_0 \cos\theta_0 \\
    v_{0y} &= v_0 \sin\theta_0
\end{align*}

\textbf{Ecuaciones de posición:}
\begin{align*}
    x(t) &= v_{0x}t \\
    y(t) &= v_{0y}t - \frac{1}{2}gt^2
\end{align*}

\textbf{Ecuaciones de velocidad:}
\begin{align*}
    v_x(t) &= v_{0x} \quad \text{(constante)} \\
    v_y(t) &= v_{0y} - gt
\end{align*}

\textbf{Tiempo hasta altura máxima:}
\begin{equation*}
    t_{\text{máx}} = \frac{v_{0y}}{g}
\end{equation*}

\textbf{Altura máxima:}
\begin{equation*}
    h_{\text{máx}} = \frac{v_{0y}^2}{2g}
\end{equation*}

\textbf{Alcance horizontal:}
\begin{equation*}
    R = \frac{v_0^2 \sin(2\theta_0)}{g}
\end{equation*}

\section{Desarrollo de la Solución}

\begin{solucionbox}

\subsection*{Cálculos preliminares}

\textbf{Componentes de velocidad inicial:}

\begin{align*}
    v_{0x} &= v_0 \cos\theta_0 = 30.0 \times \cos(36.9°) \\
    v_{0x} &= 30.0 \times 0.8 = 24.0 \text{ m/s}
\end{align*}

\begin{align*}
    v_{0y} &= v_0 \sin\theta_0 = 30.0 \times \sin(36.9°) \\
    v_{0y} &= 30.0 \times 0.6 = 18.0 \text{ m/s}
\end{align*}

\subsection*{Parte a) Posición y velocidad en $t = 2.00$ s}

\textbf{Posición en $t = 2.00$ s:}

Posición horizontal:
\begin{align*}
    x(2.00) &= v_{0x}t = 24.0 \times 2.00 \\
    x(2.00) &= 48.0 \text{ m}
\end{align*}

Posición vertical:
\begin{align*}
    y(2.00) &= v_{0y}t - \frac{1}{2}gt^2 \\
    y(2.00) &= 18.0 \times 2.00 - \frac{1}{2} \times 9.8 \times (2.00)^2 \\
    y(2.00) &= 36.0 - 19.6 \\
    y(2.00) &= 16.4 \text{ m}
\end{align*}

\textbf{Posición:} $(x, y) = (48.0 \text{ m}, 16.4 \text{ m})$

\vspace{0.3cm}

\textbf{Velocidad en $t = 2.00$ s:}

Componente horizontal (constante):
\begin{equation}
    v_x(2.00) = v_{0x} = 24.0 \text{ m/s}
\end{equation}

Componente vertical:
\begin{align*}
    v_y(2.00) &= v_{0y} - gt \\
    v_y(2.00) &= 18.0 - 9.8 \times 2.00 \\
    v_y(2.00) &= 18.0 - 19.6 \\
    v_y(2.00) &= -1.6 \text{ m/s}
\end{align*}

El signo negativo indica que la pelota está descendiendo.

\textbf{Magnitud de la velocidad:}
\begin{align*}
    |\vec{v}(2.00)| &= \sqrt{v_x^2 + v_y^2} \\
    |\vec{v}(2.00)| &= \sqrt{(24.0)^2 + (-1.6)^2} \\
    |\vec{v}(2.00)| &= \sqrt{576 + 2.56} \\
    |\vec{v}(2.00)| &= \sqrt{578.56} \\
    |\vec{v}(2.00)| &= 24.1 \text{ m/s}
\end{align*}

\textbf{Dirección de la velocidad:}
\begin{align*}
    \theta &= \arctan\left(\frac{v_y}{v_x}\right) \\
    \theta &= \arctan\left(\frac{-1.6}{24.0}\right) \\
    \theta &= \arctan(-0.0667) \\
    \theta &= -3.8°
\end{align*}

El ángulo negativo significa que la velocidad apunta $3.8°$ debajo de la horizontal.

\subsection*{Parte b) Tiempo hasta la altura máxima}

En la altura máxima, $v_y = 0$:
\begin{align*}
    0 &= v_{0y} - gt_{\text{máx}} \\
    t_{\text{máx}} &= \frac{v_{0y}}{g} \\
    t_{\text{máx}} &= \frac{18.0}{9.8} \\
    t_{\text{máx}} &= 1.84 \text{ s}
\end{align*}

\subsection*{Parte c) Altura máxima}

Usando la ecuación de altura máxima:
\begin{align*}
    h_{\text{máx}} &= \frac{v_{0y}^2}{2g} \\
    h_{\text{máx}} &= \frac{(18.0)^2}{2 \times 9.8} \\
    h_{\text{máx}} &= \frac{324}{19.6} \\
    h_{\text{máx}} &= 16.5 \text{ m}
\end{align*}

\textbf{Verificación usando la ecuación de posición:}
\begin{align*}
    h_{\text{máx}} &= v_{0y}t_{\text{máx}} - \frac{1}{2}g t_{\text{máx}}^2 \\
    h_{\text{máx}} &= 18.0 \times 1.84 - \frac{1}{2} \times 9.8 \times (1.84)^2 \\
    h_{\text{máx}} &= 33.12 - 16.6 \\
    h_{\text{máx}} &= 16.5 \text{ m} \quad \checkmark
\end{align*}

\subsection*{Parte d) Alcance horizontal}

El tiempo total de vuelo es el doble del tiempo hasta la altura máxima:
\begin{equation}
    t_{\text{total}} = 2t_{\text{máx}} = 2 \times 1.84 = 3.68 \text{ s}
\end{equation}

El alcance horizontal es:
\begin{align*}
    R &= v_{0x} \times t_{\text{total}} \\
    R &= 24.0 \times 3.68 \\
    R &= 88.3 \text{ m}
\end{align*}

\textbf{Verificación usando la fórmula directa:}
\begin{align*}
    R &= \frac{v_0^2 \sin(2\theta_0)}{g} \\
    R &= \frac{(30.0)^2 \sin(2 \times 36.9°)}{9.8} \\
    R &= \frac{900 \times \sin(73.8°)}{9.8} \\
    R &= \frac{900 \times 0.9604}{9.8} \\
    R &= \frac{864.4}{9.8} \\
    R &= 88.2 \text{ m} \quad \checkmark
\end{align*}

\subsection*{Parte e) Componentes de aceleración}

Durante todo el vuelo, la única aceleración es la gravedad:

\begin{equation}
    \vec{a} = (0, -g)
\end{equation}

Por lo tanto:
\begin{align*}
    a_x &= 0 \text{ m/s}^2 \\
    a_y &= -9.8 \text{ m/s}^2
\end{align*}

Esta aceleración es constante durante todo el vuelo, incluso en el punto más alto de la trayectoria.

\end{solucionbox}

\section{Resultados Finales}

\begin{resultadobox}

\textbf{a) En $t = 2.00$ s:}
\begin{itemize}
    \item Posición: $\boxed{(x, y) = (48.0 \text{ m}, 16.4 \text{ m})}$
    \item Magnitud de velocidad: $\boxed{|\vec{v}| = 24.1 \text{ m/s}}$
    \item Dirección: $\boxed{\theta = -3.8° \text{ (debajo de la horizontal)}}$
\end{itemize}

\vspace{0.3cm}

\textbf{b) Tiempo hasta altura máxima:}
\begin{equation*}
    \boxed{t_{\text{máx}} = 1.84 \text{ s}}
\end{equation*}

\vspace{0.3cm}

\textbf{c) Altura máxima:}
\begin{equation*}
    \boxed{h_{\text{máx}} = 16.5 \text{ m}}
\end{equation*}

\vspace{0.3cm}

\textbf{d) Alcance horizontal:}
\begin{equation*}
    \boxed{R = 88.2 \text{ m}}
\end{equation*}

\vspace{0.3cm}

\textbf{e) Componentes de aceleración:}
\begin{align*}
    &\boxed{a_x = 0 \text{ m/s}^2} \\
    &\boxed{a_y = -9.8 \text{ m/s}^2}
\end{align*}

\end{resultadobox}

\section{Verificación}

\subsection*{Posición en $t = 2.00$ s}

Verificamos que la posición $(48.0, 16.4)$ m está sobre la trayectoria parabólica:

Ecuación de la trayectoria:
\begin{equation}
    y = \frac{v_{0y}}{v_{0x}}x - \frac{g}{2v_{0x}^2}x^2
\end{equation}

Sustituyendo $x = 48.0$ m:
\begin{align*}
    y &= \frac{18.0}{24.0} \times 48.0 - \frac{9.8}{2 \times (24.0)^2} \times (48.0)^2 \\
    y &= 0.75 \times 48.0 - 0.0085 \times 2304 \\
    y &= 36.0 - 19.6 \\
    y &= 16.4 \text{ m} \quad \checkmark
\end{align*}

\subsection*{Simetría del movimiento}

El tiempo total de vuelo debe ser el doble del tiempo hasta la altura máxima:
\begin{equation}
    t_{\text{total}} = 2t_{\text{máx}} = 2 \times 1.84 = 3.68 \text{ s} \quad \checkmark
\end{equation}

\subsection*{Energía}

Podemos verificar usando conservación de energía. En $t = 2.00$ s, la energía total debe ser igual a la energía inicial:

\textbf{Energía inicial (en el suelo):}
\begin{equation}
    E_0 = \frac{1}{2}mv_0^2 = \frac{1}{2}m(30.0)^2 = 450m \text{ J}
\end{equation}

\textbf{Energía en $t = 2.00$ s:}
\begin{align*}
    E &= \frac{1}{2}m|\vec{v}|^2 + mgy \\
    E &= \frac{1}{2}m(24.1)^2 + m \times 9.8 \times 16.4 \\
    E &= 290.4m + 160.7m \\
    E &= 451m \text{ J} \quad \checkmark
\end{align*}

La pequeña diferencia se debe al redondeo.

\section{Análisis Físico}

\subsection{Interpretación de los resultados}

\begin{enumerate}
    \item \textbf{En $t = 2.00$ s:}
    \begin{itemize}
        \item La pelota ha superado ya el punto más alto (que ocurre en $t = 1.84$ s)
        \item Está descendiendo ($v_y < 0$)
        \item Ha recorrido más de la mitad del alcance total
        \item Su velocidad es casi horizontal (solo $3.8°$ de inclinación)
    \end{itemize}

    \item \textbf{Altura máxima de 16.5 m:}
    \begin{itemize}
        \item Equivalente a un edificio de 5 pisos
        \item Suficiente para pasar sobre la cerca de un campo típico
        \item Alcanzada en menos de 2 segundos
    \end{itemize}

    \item \textbf{Alcance de 88.2 m:}
    \begin{itemize}
        \item Distancia considerable en béisbol
        \item La pelota recorre casi todo el cuadro interior (27.4 m entre bases) más allá del jardín
        \item Un buen batazo que probablemente sería un hit
    \end{itemize}

    \item \textbf{Aceleración constante:}
    \begin{itemize}
        \item La única fuerza es la gravedad
        \item La aceleración no cambia en ningún punto de la trayectoria
        \item Incluso en el punto más alto, $a_y = -9.8$ m/s$^2$
    \end{itemize}
\end{enumerate}

\subsection{Contexto deportivo}

En un campo de béisbol real:

\begin{center}
\begin{tikzpicture}[scale=0.6]
    % Campo de béisbol simplificado
    \draw[thick, fill=brown!30] (0,0) -- (27.4,0) -- (27.4,27.4) -- (0,27.4) -- cycle;
    \node at (13.7,13.7) {Cuadro};

    % Bases
    \filldraw[black] (0,0) circle (0.3) node[below] {Home};
    \filldraw[black] (27.4,0) circle (0.3) node[right] {1B};
    \filldraw[black] (27.4,27.4) circle (0.3) node[above] {2B};
    \filldraw[black] (0,27.4) circle (0.3) node[left] {3B};

    % Alcance de la pelota
    \draw[-{Latex[length=3mm]}, red, very thick] (0,0) -- (88.2,0);
    \node[below, red] at (88.2,0) {88.2 m};

    % Jardín
    \draw[thick, green!50!black] (0,27.4) -- (88.2,27.4);
    \node at (44,35) {Jardín exterior};
\end{tikzpicture}
\end{center}

La pelota aterriza bien en el jardín exterior, ¡probablemente un doble o triple!

\section{Gráficas del Movimiento}

\subsection{Trayectoria en el plano $xy$}

\begin{center}
\begin{tikzpicture}
\begin{axis}[
    width=13cm,
    height=8cm,
    xlabel={Distancia horizontal $x$ (m)},
    ylabel={Altura $y$ (m)},
    xmin=0, xmax=95,
    ymin=0, ymax=18,
    grid=major,
    grid style={dashed,gray!30},
    legend pos=north east
]

% Trayectoria (y = 0.75x - 0.0204x^2)
\addplot[
    red!70!black,
    very thick,
    domain=0:88.2,
    samples=100,
    smooth
] {0.75*x - 0.0204*x^2};
\addlegendentry{Trayectoria}

% Punto en t=2.00s
\addplot[
    mark=*,
    mark size=4pt,
    orange!70!black,
    only marks
] coordinates {(48.0, 16.4)};
\addlegendentry{$t=2.00$ s}

% Punto de altura máxima
\addplot[
    mark=*,
    mark size=4pt,
    green!50!black,
    only marks
] coordinates {(44.1, 16.5)};
\addlegendentry{Altura máxima}

\end{axis}
\end{tikzpicture}
\end{center}

\subsection{Posición horizontal vs. tiempo}

\begin{center}
\begin{tikzpicture}
\begin{axis}[
    width=12cm,
    height=7cm,
    xlabel={Tiempo $t$ (s)},
    ylabel={Posición $x$ (m)},
    xmin=0, xmax=4,
    ymin=0, ymax=95,
    grid=major,
    grid style={dashed,gray!30},
    legend pos=north west
]

% x(t) = 24t
\addplot[
    blue!70!black,
    very thick,
    domain=0:3.68,
    samples=50
] {24*x};
\addlegendentry{$x(t) = 24.0t$}

% Punto en t=2.00s
\addplot[
    mark=*,
    mark size=4pt,
    orange!70!black,
    only marks
] coordinates {(2.0, 48.0)};
\addlegendentry{$t=2.00$ s}

\end{axis}
\end{tikzpicture}
\end{center}

\subsection{Posición vertical vs. tiempo}

\begin{center}
\begin{tikzpicture}
\begin{axis}[
    width=12cm,
    height=8cm,
    xlabel={Tiempo $t$ (s)},
    ylabel={Altura $y$ (m)},
    xmin=0, xmax=4,
    ymin=0, ymax=18,
    grid=major,
    grid style={dashed,gray!30},
    legend pos=north east
]

% y(t) = 18t - 4.9t^2
\addplot[
    red!70!black,
    very thick,
    domain=0:3.68,
    samples=100
] {18*x - 4.9*x^2};
\addlegendentry{$y(t) = 18.0t - 4.9t^2$}

% Punto en t=2.00s
\addplot[
    mark=*,
    mark size=4pt,
    orange!70!black,
    only marks
] coordinates {(2.0, 16.4)};
\addlegendentry{$t=2.00$ s}

% Punto de altura máxima
\addplot[
    mark=*,
    mark size=4pt,
    green!50!black,
    only marks
] coordinates {(1.84, 16.5)};
\addlegendentry{$t_{\text{máx}}=1.84$ s}

\end{axis}
\end{tikzpicture}
\end{center}

\subsection{Velocidad vertical vs. tiempo}

\begin{center}
\begin{tikzpicture}
\begin{axis}[
    width=12cm,
    height=8cm,
    xlabel={Tiempo $t$ (s)},
    ylabel={Velocidad $v_y$ (m/s)},
    xmin=0, xmax=4,
    ymin=-20, ymax=20,
    grid=major,
    grid style={dashed,gray!30},
    legend pos=north east
]

% vy(t) = 18 - 9.8t
\addplot[
    red!70!black,
    very thick,
    domain=0:3.68,
    samples=50
] {18 - 9.8*x};
\addlegendentry{$v_y(t) = 18.0 - 9.8t$}

% Punto en t=2.00s
\addplot[
    mark=*,
    mark size=4pt,
    orange!70!black,
    only marks
] coordinates {(2.0, -1.6)};
\addlegendentry{$t=2.00$ s}

% Punto donde vy=0
\addplot[
    mark=*,
    mark size=4pt,
    green!50!black,
    only marks
] coordinates {(1.84, 0)};
\addlegendentry{$v_y=0$ en $t_{\text{máx}}$}

% Línea y=0
\addplot[
    dashed,
    black,
    domain=0:3.68
] {0};

\end{axis}
\end{tikzpicture}
\end{center}

\section{Observaciones Importantes}

\begin{enumerate}
    \item \textbf{Movimiento simétrico:} El tiempo de subida es igual al tiempo de bajada.

    \item \textbf{Velocidad horizontal constante:} $v_x = 24.0$ m/s durante todo el vuelo.

    \item \textbf{Velocidad en $t = 2.00$ s:} La pelota está descendiendo pero todavía tiene una velocidad considerable de 24.1 m/s.

    \item \textbf{Ángulo de 36.9°:} Este es un ángulo típico en béisbol, cercano al ángulo óptimo de 45° para máximo alcance pero más bajo para evitar que la pelota sea atrapada fácilmente.

    \item \textbf{Aceleración:} Es constante e independiente del movimiento horizontal. Solo afecta la componente vertical.

    \item \textbf{Comparación con 45°:} Si se bateara con el mismo $v_0$ pero a 45°, el alcance sería máximo: $R_{45°} = v_0^2/g = 900/9.8 = 91.8$ m, solo 3.6 m más lejos.
\end{enumerate}

\end{document}
