\documentclass[11pt,a4paper]{article}

% Paquetes necesarios
\usepackage[utf8]{inputenc}
\usepackage[T1]{fontenc}
\usepackage[spanish]{babel}
\usepackage[margin=2.5cm]{geometry}
\usepackage{amsmath}
\usepackage{amssymb}
\usepackage{xcolor}
\usepackage{tcolorbox}
\usepackage{graphicx}
\usepackage{tikz}
\usepackage{pgfplots}
\pgfplotsset{compat=1.18}
\usetikzlibrary{arrows.meta,patterns,decorations.pathmorphing,calc}

% Definición de colores
\definecolor{azuloscuro}{RGB}{0,51,102}
\definecolor{azulclaro}{RGB}{230,240,250}
\definecolor{verdeoscuro}{RGB}{0,100,0}
\definecolor{rojoclaro}{RGB}{255,230,230}
\definecolor{naranja}{RGB}{255,140,0}

% Configuración de cajas
\tcbuselibrary{theorems,skins,breakable}

\newtcolorbox{datosbox}{
    colback=azulclaro,
    colframe=azuloscuro,
    fonttitle=\bfseries,
    title=Datos del Problema,
    sharp corners,
    boxrule=1pt
}

\newtcolorbox{solucionbox}{
    colback=white,
    colframe=verdeoscuro,
    fonttitle=\bfseries,
    title=Desarrollo de la Solución,
    sharp corners,
    boxrule=1pt,
    breakable
}

\newtcolorbox{resultadobox}{
    colback=rojoclaro,
    colframe=red!70!black,
    fonttitle=\bfseries,
    title=Resultado Final,
    sharp corners,
    boxrule=2pt
}

% Título y autor
\title{\textbf{Solución del Ejercicio 3.22} \\
\large El Dardo y el Mono - Análisis de Tres Casos}
\author{Física Universitaria - Sears y Zemansky \\ Capítulo 3: Movimiento en Dos o Tres Dimensiones}
\date{\today}

\begin{document}

\maketitle

\section{Enunciado del Problema}

Suponga que el ángulo inicial $\alpha_0$ de la figura 3.26 es de 42.0° y la distancia $d$ es de 3.00 m. ¿Dónde se encontrarán el dardo y el mono, si la rapidez inicial del dardo es:

\begin{enumerate}
    \item[a)] 12.0 m/s?
    \item[b)] 8.0 m/s?
    \item[c)] ¿Qué sucederá si la rapidez inicial del dardo es de 4.0 m/s? Dibuje la trayectoria en cada caso.
\end{enumerate}

\section{Datos del Problema}

\begin{datosbox}
\begin{itemize}
    \item \textbf{Ángulo de lanzamiento del dardo:} $\alpha_0 = 42.0°$
    \item \textbf{Distancia horizontal al mono:} $d = 3.00$ m
    \item \textbf{Altura inicial del mono:} $h_0 = d \tan\alpha_0 = 3.00 \times \tan(42.0°) = 2.70$ m
    \item \textbf{Aceleración gravitacional:} $g = 9.8$ m/s$^2$
    \item \textbf{Velocidades iniciales a analizar:}
    \begin{itemize}
        \item Caso a): $v_0 = 12.0$ m/s
        \item Caso b): $v_0 = 8.0$ m/s
        \item Caso c): $v_0 = 4.0$ m/s
    \end{itemize}
    \item \textbf{Condición:} El mono se suelta en el instante en que se dispara el dardo
    \item \textbf{Resistencia del aire:} despreciable
\end{itemize}
\end{datosbox}

\section{Marco Teórico}

\subsection*{El problema del dardo y el mono}

Este es un problema clásico de movimiento de proyectiles. Un mono está colgado de una rama a una distancia horizontal $d$ y altura $h_0 = d\tan\alpha_0$. En el instante en que se dispara el dardo (apuntando directamente al mono), el mono se suelta y cae libremente.

\textbf{Principio fundamental:} El dardo siempre golpeará al mono, independientemente de la velocidad inicial, porque ambos caen la misma distancia vertical bajo la acción de la gravedad.

\textbf{Ecuaciones del dardo (proyectil):}
\begin{align}
    x_d(t) &= v_0 \cos\alpha_0 \cdot t \\
    y_d(t) &= v_0 \sin\alpha_0 \cdot t - \frac{1}{2}gt^2
\end{align}

\textbf{Ecuaciones del mono (caída libre):}
\begin{align}
    x_m &= d \quad \text{(constante)} \\
    y_m(t) &= h_0 - \frac{1}{2}gt^2 = d\tan\alpha_0 - \frac{1}{2}gt^2
\end{align}

\textbf{Encuentro:} Ocurre cuando $x_d = x_m = d$, lo que sucede en el tiempo:
\begin{equation}
    t_{enc} = \frac{d}{v_0 \cos\alpha_0}
\end{equation}

\section{Desarrollo de la Solución}

\begin{solucionbox}

\subsection*{Cálculos previos}

Calculamos las componentes de la velocidad inicial para cada caso:

\begin{align}
    v_{0x} &= v_0 \cos(42.0°) = v_0 \times 0.7431 \\
    v_{0y} &= v_0 \sin(42.0°) = v_0 \times 0.6691
\end{align}

Altura inicial del mono:
\begin{equation}
    h_0 = d \tan(42.0°) = 3.00 \times 0.9004 = 2.70 \text{ m}
\end{equation}

\subsection*{Caso a): $v_0 = 12.0$ m/s}

\textbf{Componentes de velocidad inicial:}
\begin{align}
    v_{0x} &= 12.0 \times 0.7431 = 8.917 \text{ m/s} \\
    v_{0y} &= 12.0 \times 0.6691 = 8.029 \text{ m/s}
\end{align}

\textbf{Tiempo de encuentro:}

El dardo alcanza la posición horizontal $x = d = 3.00$ m en el tiempo:
\begin{equation}
    t_{enc} = \frac{d}{v_{0x}} = \frac{3.00}{8.917} = 0.3365 \text{ s}
\end{equation}

\textbf{Altura del encuentro:}

Para el dardo:
\begin{align}
    y_d &= v_{0y}t - \frac{1}{2}gt^2 \\
    &= 8.029(0.3365) - \frac{1}{2}(9.8)(0.3365)^2 \\
    &= 2.702 - 0.555 \\
    &= 2.15 \text{ m}
\end{align}

Para el mono (verificación):
\begin{align}
    y_m &= h_0 - \frac{1}{2}gt^2 \\
    &= 2.70 - \frac{1}{2}(9.8)(0.3365)^2 \\
    &= 2.70 - 0.555 \\
    &= 2.15 \text{ m} \quad \checkmark
\end{align}

\textbf{Conclusión:} El dardo golpea al mono en la posición $(3.00, 2.15)$ m, a una altura de 2.15 m sobre el suelo, después de 0.337 s.

\subsection*{Caso b): $v_0 = 8.0$ m/s}

\textbf{Componentes de velocidad inicial:}
\begin{align}
    v_{0x} &= 8.0 \times 0.7431 = 5.945 \text{ m/s} \\
    v_{0y} &= 8.0 \times 0.6691 = 5.353 \text{ m/s}
\end{align}

\textbf{Tiempo de encuentro:}
\begin{equation}
    t_{enc} = \frac{3.00}{5.945} = 0.5046 \text{ s}
\end{equation}

\textbf{Altura del encuentro:}

Para el dardo:
\begin{align}
    y_d &= 5.353(0.5046) - \frac{1}{2}(9.8)(0.5046)^2 \\
    &= 2.701 - 1.248 \\
    &= 1.45 \text{ m}
\end{align}

Para el mono (verificación):
\begin{align}
    y_m &= 2.70 - \frac{1}{2}(9.8)(0.5046)^2 \\
    &= 2.70 - 1.248 \\
    &= 1.45 \text{ m} \quad \checkmark
\end{align}

\textbf{Conclusión:} El dardo golpea al mono en la posición $(3.00, 1.45)$ m, a una altura de 1.45 m sobre el suelo, después de 0.505 s.

\subsection*{Caso c): $v_0 = 4.0$ m/s}

\textbf{Componentes de velocidad inicial:}
\begin{align}
    v_{0x} &= 4.0 \times 0.7431 = 2.972 \text{ m/s} \\
    v_{0y} &= 4.0 \times 0.6691 = 2.676 \text{ m/s}
\end{align}

\textbf{Tiempo para alcanzar $x = 3.00$ m:}
\begin{equation}
    t = \frac{3.00}{2.972} = 1.009 \text{ s}
\end{equation}

\textbf{Altura del dardo en ese instante:}
\begin{align}
    y_d &= 2.676(1.009) - \frac{1}{2}(9.8)(1.009)^2 \\
    &= 2.700 - 4.986 \\
    &= -2.29 \text{ m}
\end{align}

\textbf{¡El dardo está por debajo del suelo!} Esto significa que el dardo cae al suelo antes de alcanzar la posición horizontal del mono.

\textbf{¿Cuándo cae el dardo al suelo?} Cuando $y_d = 0$:
\begin{equation}
    0 = v_{0y}t - \frac{1}{2}gt^2 = t\left(v_{0y} - \frac{1}{2}gt\right)
\end{equation}

Las soluciones son $t = 0$ (lanzamiento) y:
\begin{equation}
    t = \frac{2v_{0y}}{g} = \frac{2(2.676)}{9.8} = 0.546 \text{ s}
\end{equation}

\textbf{Posición horizontal donde cae:}
\begin{equation}
    x = v_{0x}t = 2.972(0.546) = 1.62 \text{ m}
\end{equation}

\textbf{¿Dónde está el mono cuando el dardo cae al suelo?}

En $t = 0.546$ s:
\begin{align}
    x_m &= 3.00 \text{ m} \\
    y_m &= 2.70 - \frac{1}{2}(9.8)(0.546)^2 = 2.70 - 1.46 = 1.24 \text{ m}
\end{align}

\textbf{Conclusión:} Con $v_0 = 4.0$ m/s, el dardo no tiene suficiente velocidad inicial. Cae al suelo en $x = 1.62$ m, mientras el mono todavía está cayendo y se encuentra a 3.00 m horizontalmente y 1.24 m verticalmente. \textbf{El dardo no alcanza al mono.}

\subsection*{Verificación: ¿Cuándo cae el mono al suelo?}

El mono cae al suelo cuando $y_m = 0$:
\begin{equation}
    0 = h_0 - \frac{1}{2}gt^2 \quad \Rightarrow \quad t = \sqrt{\frac{2h_0}{g}} = \sqrt{\frac{2(2.70)}{9.8}} = 0.742 \text{ s}
\end{equation}

El mono cae al suelo después de 0.742 s, mientras que el dardo con $v_0 = 4.0$ m/s cae después de 0.546 s.

\end{solucionbox}

\section{Diagramas de las Trayectorias}

\subsection*{Caso a): $v_0 = 12.0$ m/s (Encuentro a 2.15 m de altura)}

\begin{center}
\begin{tikzpicture}[scale=1.8]
    % Suelo
    \draw[fill=brown!20, thick] (-0.5,0) rectangle (4,-.3);
    \draw[pattern=north east lines, pattern color=brown] (-0.5,-0.3) rectangle (4,-0.45);

    % Lanzador
    \draw[fill=blue!20, thick] (0,0) circle (0.08);
    \node[below, font=\small] at (0,0) {Lanzador};

    % Posición inicial del mono
    \draw[fill=yellow!40, thick] (3,2.7) circle (0.1);
    \node[right, font=\scriptsize] at (3.15,2.7) {Mono (inicial)};

    % Línea punteada desde lanzador al mono
    \draw[dashed, gray] (0,0) -- (3,2.7);
    \node[above, font=\scriptsize, rotate=42] at (1.5,1.35) {$\alpha_0=42°$};

    % Vector velocidad inicial
    \draw[-{Latex[length=2.5mm]},blue!70!black, ultra thick] (0,0) -- (0.7,0.528);
    \node[blue!70!black, above right, font=\scriptsize] at (0.7,0.4) {$v_0=12$ m/s};

    % Trayectoria del dardo - Ecuación: y = 0.9004x - 0.0616x²
    % La trayectoria completa hasta el suelo (x = 14.62 m donde y = 0)
    \draw[red!70!black, very thick]
        plot[domain=0:3, samples=100, smooth] (\x, {0.9004*\x - 0.0616*\x*\x});
    \draw[red!70!black, very thick, dashed, -{Latex[length=2.5mm]}]
        plot[domain=3:4, samples=50, smooth] (\x, {0.9004*\x - 0.0616*\x*\x});

    % Caída del mono (línea vertical)
    \draw[orange!80!black, very thick, dashed] (3,2.7) -- (3,2.15);

    % Punto de encuentro
    \filldraw[green!60!black] (3,2.15) circle (0.08);
    \node[green!60!black, right, font=\small] at (3.15,2.15) {Encuentro};
    \node[green!60!black, right, font=\scriptsize] at (3.15,1.95) {$(3.00, 2.15)$ m};
    \node[green!60!black, right, font=\scriptsize] at (3.15,1.75) {$t=0.337$ s};

    % Distancia horizontal
    \draw[{Latex[length=2mm]}-{Latex[length=2mm]}, thick] (0,-0.6) -- (3,-0.6);
    \node[below, font=\small] at (1.5,-0.6) {$d = 3.00$ m};

    % Altura inicial del mono
    \draw[{Latex[length=2mm]}-{Latex[length=2mm]}, thick] (-0.3,0) -- (-0.3,2.7);
    \node[left, font=\scriptsize] at (-0.3,1.35) {$h_0=2.70$ m};
\end{tikzpicture}
\end{center}

\subsection*{Caso b): $v_0 = 8.0$ m/s (Encuentro a 1.45 m de altura)}

\begin{center}
\begin{tikzpicture}[scale=1.8]
    % Suelo
    \draw[fill=brown!20, thick] (-0.5,0) rectangle (4,-0.3);
    \draw[pattern=north east lines, pattern color=brown] (-0.5,-0.3) rectangle (4,-0.45);

    % Lanzador
    \draw[fill=blue!20, thick] (0,0) circle (0.08);
    \node[below, font=\small] at (0,0) {Lanzador};

    % Posición inicial del mono
    \draw[fill=yellow!40, thick] (3,2.7) circle (0.1);
    \node[right, font=\scriptsize] at (3.15,2.7) {Mono (inicial)};

    % Línea punteada desde lanzador al mono
    \draw[dashed, gray] (0,0) -- (3,2.7);

    % Vector velocidad inicial
    \draw[-{Latex[length=2.5mm]},blue!70!black, ultra thick] (0,0) -- (0.5,0.358);
    \node[blue!70!black, above right, font=\scriptsize] at (0.5,0.2) {$v_0=8$ m/s};

    % Trayectoria del dardo - Ecuación: y = 0.9004x - 0.1386x²
    % La trayectoria completa hasta el suelo (x = 6.50 m donde y = 0)
    \draw[red!70!black, very thick]
        plot[domain=0:3, samples=100, smooth] (\x, {0.9004*\x - 0.1386*\x*\x});
    \draw[red!70!black, very thick, dashed, -{Latex[length=2.5mm]}]
        plot[domain=3:4, samples=50, smooth] (\x, {0.9004*\x - 0.1386*\x*\x});

    % Caída del mono
    \draw[orange!80!black, very thick, dashed] (3,2.7) -- (3,1.45);

    % Punto de encuentro
    \filldraw[green!60!black] (3,1.45) circle (0.08);
    \node[green!60!black, right, font=\small] at (3.15,1.45) {Encuentro};
    \node[green!60!black, right, font=\scriptsize] at (3.15,1.25) {$(3.00, 1.45)$ m};
    \node[green!60!black, right, font=\scriptsize] at (3.15,1.05) {$t=0.505$ s};

    % Distancia horizontal
    \draw[{Latex[length=2mm]}-{Latex[length=2mm]}, thick] (0,-0.6) -- (3,-0.6);
    \node[below, font=\small] at (1.5,-0.6) {$d = 3.00$ m};
\end{tikzpicture}
\end{center}

\subsection*{Caso c): $v_0 = 4.0$ m/s (El dardo no alcanza al mono)}

\begin{center}
\begin{tikzpicture}[scale=1.8]
    % Suelo
    \draw[fill=brown!20, thick] (-0.5,0) rectangle (4,-0.5);
    \draw[pattern=north east lines, pattern color=brown] (-0.5,-0.5) rectangle (4,-0.65);

    % Lanzador
    \draw[fill=blue!20, thick] (0,0) circle (0.08);
    \node[below, font=\small] at (0,-0.1) {Lanzador};

    % Posición inicial del mono
    \draw[fill=yellow!40, thick] (3,2.7) circle (0.1);
    \node[right, font=\scriptsize] at (3.15,2.7) {Mono (inicial)};

    % Línea punteada desde lanzador al mono
    \draw[dashed, gray] (0,0) -- (3,2.7);

    % Vector velocidad inicial
    \draw[-{Latex[length=2.5mm]},blue!70!black, ultra thick] (0,0) -- (0.3,0.215);
    \node[blue!70!black, above right, font=\scriptsize] at (0.2,0) {$v_0=4$ m/s};

    % Trayectoria del dardo - Ecuación: y = 0.9004x - 0.5545x²
    % El dardo cae al suelo en x = 1.62 m
    \draw[red!70!black, very thick, -{Latex[length=2.5mm]}]
        plot[domain=0:1.62, samples=100, smooth] (\x, {0.9004*\x - 0.5545*\x*\x});

    % Punto donde cae el dardo
    \filldraw[red!70!black] (1.62,0) circle (0.08);
    \node[red!70!black, below, font=\small] at (0.65,-0.7) {Dardo cae};
    \node[red!70!black, below, font=\scriptsize] at (1.62,-0.7) {$x=1.62$ m};
    \node[red!70!black, below, font=\scriptsize] at (2.62,-0.7) {$t=0.546$ s};

    % Posición del mono cuando el dardo cae
    \draw[orange!80!black, very thick, dashed] (3,2.7) -- (3,1.24);
    \filldraw[orange!80!black] (3,1.24) circle (0.08);
    \node[orange!80!black, right, font=\small] at (3.15,1.24) {Mono en $t=0.546$ s};
    \node[orange!80!black, right, font=\scriptsize] at (3.15,1.04) {$(3.00, 1.24)$ m};

    % Línea mostrando la separación
    \draw[purple, thick, dotted] (1.62,0) -- (3,1.24);
    \node[purple, above, font=\scriptsize] at (2.3,0.62) {NO se encuentran};

    % Distancia horizontal
    \draw[{Latex[length=2mm]}-{Latex[length=2mm]}, thick] (0,-0.95) -- (3,-0.95);
    \node[below, font=\small] at (1.5,-0.95) {$d = 3.00$ m};
\end{tikzpicture}
\end{center}

\section{Resultados Finales}

\begin{resultadobox}

\subsection*{Caso a): $v_0 = 12.0$ m/s}
\begin{equation}
    \boxed{\text{Encuentro en: } (x, y) = (3.00, 2.15) \text{ m en } t = 0.337 \text{ s}}
\end{equation}

El dardo golpea al mono a una altura de \textbf{2.15 metros} sobre el suelo. El mono ha caído $2.70 - 2.15 = 0.55$ m desde su posición inicial.

\subsection*{Caso b): $v_0 = 8.0$ m/s}
\begin{equation}
    \boxed{\text{Encuentro en: } (x, y) = (3.00, 1.45) \text{ m en } t = 0.505 \text{ s}}
\end{equation}

El dardo golpea al mono a una altura de \textbf{1.45 metros} sobre el suelo. El mono ha caído $2.70 - 1.45 = 1.25$ m desde su posición inicial.

\subsection*{Caso c): $v_0 = 4.0$ m/s}
\begin{equation}
    \boxed{\text{NO HAY ENCUENTRO - El dardo cae al suelo antes}}
\end{equation}

\textbf{Dardo:} Cae al suelo en $x = 1.62$ m después de $t = 0.546$ s

\textbf{Mono:} En ese instante está en $(3.00, 1.24)$ m, todavía cayendo

\textbf{Distancia de separación:} El mono está 1.38 m más lejos horizontalmente cuando el dardo cae al suelo.

\vspace{0.3cm}

\textbf{Velocidad mínima necesaria:}

Para que el dardo alcance la posición horizontal $x = d = 3.00$ m antes de caer al suelo ($y = 0$), necesitamos:

\begin{equation}
    v_{0,\text{min}} = \sqrt{\frac{gd^2}{2d\sin\alpha_0\cos\alpha_0}} = \sqrt{\frac{9.8(3.00)^2}{2(3.00)\sin(42°)\cos(42°)}} = 5.74 \text{ m/s}
\end{equation}

Como $4.0 < 5.74$ m/s, el dardo no tiene suficiente velocidad inicial.

\end{resultadobox}

\section{Análisis y Conclusión}

Este problema ilustra varios aspectos importantes del movimiento de proyectiles y el problema clásico del "dardo y el mono":

\begin{enumerate}
    \item \textbf{Principio del dardo y el mono:} Cuando la velocidad inicial es suficiente (casos a y b), el dardo \textit{siempre} golpea al mono, sin importar la rapidez inicial. Esto se debe a que ambos objetos caen la misma distancia vertical bajo la acción de la gravedad en el mismo tiempo.

    \item \textbf{Efecto de la velocidad inicial:}
    \begin{itemize}
        \item A mayor velocidad, el encuentro ocurre más rápido y a mayor altura
        \item Caso a ($v_0 = 12$ m/s): Encuentro a 2.15 m (79.6\% de la altura inicial)
        \item Caso b ($v_0 = 8$ m/s): Encuentro a 1.45 m (53.7\% de la altura inicial)
        \item Caso c ($v_0 = 4$ m/s): No hay encuentro
    \end{itemize}

    \item \textbf{Velocidad mínima requerida:} Existe una velocidad mínima ($v_{0,\text{min}} = 5.74$ m/s) por debajo de la cual el dardo cae al suelo antes de alcanzar la posición horizontal del mono. Esta velocidad mínima depende de:
    \begin{itemize}
        \item La distancia horizontal $d$
        \item El ángulo de lanzamiento $\alpha_0$
        \item La aceleración gravitacional $g$
    \end{itemize}

    \item \textbf{Verificación matemática:} En los casos a) y b), verificamos que $y_d = y_m$ en el tiempo de encuentro, confirmando que el dardo golpea al mono en el mismo instante en que ambos están a la misma altura.

    \item \textbf{Trayectorias parabólicas:} Las tres trayectorias son parabólicas, pero difieren en:
    \begin{itemize}
        \item El alcance horizontal (mayor con mayor $v_0$)
        \item La altura máxima (mayor con mayor $v_0$)
        \item El tiempo de vuelo (mayor con menor $v_0$)
    \end{itemize}

    \item \textbf{Aplicación práctica:} Este problema modela situaciones reales como el disparo de proyectiles a objetivos en movimiento, donde es necesario "apuntar directamente" al objetivo (no por debajo ni por encima), siempre que la velocidad sea suficiente.
\end{enumerate}

\end{document}
