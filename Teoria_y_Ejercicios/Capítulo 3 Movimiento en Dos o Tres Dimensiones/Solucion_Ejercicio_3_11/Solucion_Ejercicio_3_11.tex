\documentclass[11pt,a4paper]{article}

% Paquetes necesarios
\usepackage[utf8]{inputenc}
\usepackage[T1]{fontenc}
\usepackage[spanish]{babel}
\usepackage[margin=2.5cm]{geometry}
\usepackage{amsmath}
\usepackage{amssymb}
\usepackage{xcolor}
\usepackage{tcolorbox}
\usepackage{graphicx}
\usepackage{tikz}
\usetikzlibrary{arrows.meta,patterns,decorations.pathmorphing,calc}
\usepackage{wrapfig}

% Definición de colores
\definecolor{azuloscuro}{RGB}{0,51,102}
\definecolor{azulclaro}{RGB}{230,240,250}
\definecolor{verdeoscuro}{RGB}{0,100,0}
\definecolor{rojoclaro}{RGB}{255,230,230}

% Configuración de cajas
\tcbuselibrary{theorems,skins,breakable}

\newtcolorbox{datosbox}{
    colback=azulclaro,
    colframe=azuloscuro,
    fonttitle=\bfseries,
    title=Datos del Problema,
    sharp corners,
    boxrule=1pt
}

\newtcolorbox{solucionbox}{
    colback=white,
    colframe=verdeoscuro,
    fonttitle=\bfseries,
    title=Desarrollo de la Solución,
    sharp corners,
    boxrule=1pt,
    breakable
}

\newtcolorbox{resultadobox}{
    colback=rojoclaro,
    colframe=red!70!black,
    fonttitle=\bfseries,
    title=Resultado Final,
    sharp corners,
    boxrule=2pt
}

% Título y autor
\title{\textbf{Solución del Ejercicio 3.11} \\
\large Movimiento de Proyectiles: Los Grillos Chirpy y Milada}
\author{Física Universitaria - Sears y Zemansky \\ Capítulo 3: Movimiento en Dos o Tres Dimensiones}
\date{\today}

\begin{document}

\maketitle

\section{Enunciado del Problema}

\begin{wrapfigure}[10]{r}{.6\textwidth}  % Ajusta el ancho según necesites
\centering
\vspace{-\baselineskip}
%\begin{center}
\begin{tikzpicture}[scale=1.1]
    % Acantilado
    \draw[fill=brown!50, thick] (0,-4) rectangle (1,0);
    \draw[brown!70, thick] (1,-4) -- (1,0);

    % Chirpy (caída vertical)
    \filldraw[green!60!black] (0.3,0) circle (0.1);
    \node[left] at (0.3,0.2) {\small Chirpy};
    \draw[green!60!black, thick, dashed, -{Latex[length=2mm]}] (0.3,0) -- (0.3,-4);
    \node[green!60!black, right] at (0.3,-2) {Caída vertical};

    % Milada (salto horizontal)
    \filldraw[red!70!black] (0.7,0) circle (0.1);
    \node[right] at (0.4,-0.3) {\small Milada};
    \draw[-{Latex[length=2mm]},red!70!black, ultra thick] (0.8,0) -- (1.3,0)
        node[midway,above] {\tiny $v_0$};
    % Trayectoria parabólica (ecuación física real: y = y0 - (g/(2v0^2))*x^2)
    % Para v0 = 0.950 m/s, g = 9.8 m/s^2, con escala vertical ajustada: coeficiente = 0.362
    \draw[red!70!black, thick, dashed, -{Latex[length=2mm]}]
        plot[domain=0.7:4, samples=50, smooth] (\x, {0 - 0.362*(\x-0.7)*(\x-0.7)});
    \node[red!70!black] at (4.3,-1.5) {Trayectoria parabólica};

    % Suelo
    \draw[fill=green!40!brown, thick] (-0.5,-4) rectangle (6,-4.2);
    \draw[pattern=north east lines, pattern color=brown!50] (-0.5,-4.5) rectangle (6,-4.2);

    % Puntos de impacto
    \filldraw[green!60!black] (0.3,-4) circle (0.08);
    \node[green!60!black, below] at (0.3,-4.2) {\tiny Chirpy};

    \filldraw[red!70!black] (4,-4) circle (0.08);
    \node[red!70!black, below] at (4,-4.2) {\tiny Milada};

    % Cotas
    \draw[{Latex[length=2mm]}-{Latex[length=2mm]}, thick]
        (-0.3,-4) -- (-0.3,0) node[midway,left] {$h = ?$};

    \draw[{Latex[length=2mm]}-{Latex[length=2mm]}, thick]
        (1,-4.7) -- (4,-4.7) node[midway,below] {$d = ?$};

    % Etiquetas
    \node at (0.5,-2.5) {Acantilado};

\end{tikzpicture}
%\end{center}
\vspace{-\baselineskip} % Reduce espacio después
\end{wrapfigure}
Dos grillos, Chirpy y Milada, saltan desde lo alto de un acantilado vertical. Chirpy simplemente se deja caer y llega al suelo en 3.50 s; en tanto que Milada salta horizontalmente con una rapidez inicial de 95.0 cm/s. ¿A qué distancia de la base del acantilado tocará Milada el suelo?

\vspace{10mm}

\section{Datos del Problema}

\begin{datosbox}
\begin{itemize}
    \item \textbf{Tiempo de caída de Chirpy:} $t = 3.50$ s (caída libre vertical)
    \item \textbf{Velocidad inicial de Milada:} $v_0 = 95.0$ cm/s $= 0.950$ m/s (horizontal)
    \item \textbf{Velocidad inicial vertical de Milada:} $v_{0y} = 0$
    \item \textbf{Aceleración de la gravedad:} $g = 9.8$ m/s$^2$
    \item \textbf{Resistencia del aire:} despreciable
\end{itemize}
\end{datosbox}

\section{Análisis del Problema}

\subsection{Información clave}

El problema nos da información indirecta sobre la altura del acantilado a través del tiempo de caída de Chirpy. Ambos grillos:

\begin{itemize}
    \item Parten desde la misma altura (lo alto del acantilado)
    \item Caen la misma distancia vertical
    \item Tienen el mismo tiempo de caída (el movimiento horizontal no afecta el tiempo de caída)
\end{itemize}

\subsection{Estrategia}

\begin{enumerate}
    \item Usar el tiempo de caída de Chirpy para determinar la altura del acantilado
    \item Usar la altura para confirmar que Milada también tarda 3.50 s en caer
    \item Calcular la distancia horizontal recorrida por Milada
\end{enumerate}

\section{Marco Teórico}

\subsection{Caída Libre Vertical (Chirpy)}

Para un objeto que cae desde el reposo:
\begin{equation}
    h = \frac{1}{2}gt^2
\end{equation}

\subsection{Movimiento de Proyectil (Milada)}

Para un objeto lanzado horizontalmente:

\textbf{Horizontal:}
\begin{equation}
    x = v_0 t
\end{equation}

\textbf{Vertical:}
\begin{equation}
    y = h - \frac{1}{2}gt^2
\end{equation}

\section{Desarrollo de la Solución}

\begin{solucionbox}

\subsection*{Paso 1: Determinar la altura del acantilado}

Chirpy cae verticalmente desde el reposo y tarda 3.50 s en llegar al suelo.

Usando la ecuación de caída libre:
\begin{align*}
    h &= \frac{1}{2}gt^2 \\
    h &= \frac{1}{2}(9.8)(3.50)^2 \\
    h &= \frac{1}{2}(9.8)(12.25) \\
    h &= 4.9 \times 12.25 \\
    h &= 60.025 \text{ m} \\
    h &\approx 60.0 \text{ m}
\end{align*}

\textbf{Altura del acantilado:} $h = 60.0$ m

\subsection*{Paso 2: Confirmar el tiempo de caída de Milada}

Aunque Milada salta horizontalmente, el tiempo de caída vertical es el mismo que Chirpy porque:

\begin{itemize}
    \item La velocidad horizontal no afecta el movimiento vertical
    \item Ambos caen desde la misma altura
    \item Ambos tienen velocidad inicial vertical cero
\end{itemize}

Por lo tanto, Milada también tarda:
\begin{equation*}
    t = 3.50 \text{ s}
\end{equation*}

Verificación:
\begin{align*}
    t &= \sqrt{\frac{2h}{g}} = \sqrt{\frac{2(60.0)}{9.8}} \\
    t &= \sqrt{\frac{120}{9.8}} = \sqrt{12.24} \\
    t &= 3.50 \text{ s} \quad \checkmark
\end{align*}

\subsection*{Paso 3: Calcular la distancia horizontal de Milada}

Milada mantiene su velocidad horizontal constante de 0.950 m/s durante los 3.50 s de caída:

\begin{align*}
    d &= v_0 t \\
    d &= (0.950)(3.50) \\
    d &= 3.325 \text{ m} \\
    d &\approx 3.33 \text{ m}
\end{align*}

\textbf{Distancia horizontal:} $d = 3.33$ m = 333 cm

\end{solucionbox}

\section{Resultado Final}

\begin{resultadobox}

Milada tocará el suelo a una distancia de:

\begin{equation}
    \boxed{d = 3.33 \text{ m} = 333 \text{ cm}}
\end{equation}

desde la base del acantilado.

\vspace{0.5cm}

\textbf{Información adicional calculada:}
\begin{itemize}
    \item \textbf{Altura del acantilado:} $h = 60.0$ m
    \item \textbf{Tiempo de caída de ambos grillos:} $t = 3.50$ s
    \item \textbf{Distancia de Chirpy desde la base:} $d_C = 0$ m (caída vertical)
    \item \textbf{Distancia de Milada desde la base:} $d_M = 3.33$ m
\end{itemize}

\end{resultadobox}

\section{Verificación}

\subsection{Verificación de la altura}

Usando la altura calculada para verificar el tiempo:
\begin{align*}
    t &= \sqrt{\frac{2h}{g}} = \sqrt{\frac{2(60.0)}{9.8}} \\
    t &= \sqrt{12.24} = 3.50 \text{ s} \quad \checkmark
\end{align*}

\subsection{Verificación de la distancia}

\begin{align*}
    d &= v_0 t = v_0 \sqrt{\frac{2h}{g}} \\
    d &= 0.950 \sqrt{\frac{2(60.0)}{9.8}} \\
    d &= 0.950 \times 3.50 \\
    d &= 3.33 \text{ m} \quad \checkmark
\end{align*}

\section{Análisis Adicional}

\subsection{Velocidades al impactar}

\textbf{Chirpy (caída vertical):}
\begin{align*}
    v_x &= 0 \text{ m/s} \\
    v_y &= gt = 9.8 \times 3.50 = 34.3 \text{ m/s} \\
    v_{Chirpy} &= 34.3 \text{ m/s} \text{ (vertical)}
\end{align*}

\textbf{Milada (lanzamiento horizontal):}
\begin{align*}
    v_x &= 0.950 \text{ m/s} \\
    v_y &= gt = 9.8 \times 3.50 = 34.3 \text{ m/s} \\
    v_{Milada} &= \sqrt{(0.950)^2 + (34.3)^2} \\
    v_{Milada} &= \sqrt{0.90 + 1176} \\
    v_{Milada} &= \sqrt{1177} \\
    v_{Milada} &= 34.3 \text{ m/s}
\end{align*}

\textbf{Conclusión:} Ambos grillos impactan con prácticamente la misma rapidez (34.3 m/s), ya que la componente horizontal de Milada (0.950 m/s) es despreciable comparada con la componente vertical (34.3 m/s).

\subsection{Ángulo de impacto de Milada}

\begin{align*}
    \theta &= \arctan\left(\frac{v_y}{v_x}\right) \\
    \theta &= \arctan\left(\frac{34.3}{0.950}\right) \\
    \theta &= \arctan(36.1) \\
    \theta &= 88.4°
\end{align*}

Milada impacta casi verticalmente, con solo 1.6° de desviación de la vertical.

\subsection{Comparación de trayectorias}

  \begin{wrapfigure}{r}{6cm}  % Ajusta el ancho según necesites
	\centering
	\vspace{\baselineskip}
%\begin{center}
\begin{tikzpicture}[scale=0.08]
    % Acantilado
    \draw[fill=brown!40, very thick] (0,-60) rectangle (3,0);

    % Chirpy
    \filldraw[green!60!black] (1,0) circle (0.5);
    \draw[green!60!black, very thick, dashed, -{Latex[length=3mm]}] (1,0) -- (1,-60);
    \node[green!60!black] at (1,-40) [left] {Chirpy};

    % Milada
    \filldraw[red!70!black] (2,0) circle (0.5);
    % Trayectoria parabólica (ecuación física real: y = y0 - (g/(2v0^2))*x^2)
    % Para v0 = 0.950 m/s, g = 9.8 m/s^2: coeficiente = 5.43
    \draw[red!70!black, very thick, dashed, -{Latex[length=3mm]}]
        plot[domain=2:5.33, samples=50, smooth] (\x, {0 - 5.43*(\x-2)*(\x-2)});
    \node[red!70!black] at (3.5,-30) [right] {Milada};

    % Suelo
    \draw[fill=green!40!brown, thick] (-1,-60) rectangle (10,-62);

    % Distancias
    \draw[{Latex[length=2mm]}-{Latex[length=2mm]}, thick]
        (3,-65) -- (5.33,-65) node[midway,below] {3.33 m};

    % Altura
    \draw[{Latex[length=2mm]}-{Latex[length=2mm]}, thick]
        (-16,-60) -- (-16,0) node[midway,left] {60.0 m};

\end{tikzpicture}
%\end{center}
      \caption{Trayectorias de Chirpy y Milada}
      \vspace{-\baselineskip}
\end{wrapfigure}


\section{Conceptos Clave}

\begin{enumerate}
    \item \textbf{Independencia de movimientos:} La velocidad horizontal no afecta el tiempo de caída vertical.

    \item \textbf{Mismo tiempo de caída:} Objetos que caen desde la misma altura tardan el mismo tiempo en caer, independientemente de su velocidad horizontal inicial.

    \item \textbf{Información indirecta:} El tiempo de caída de Chirpy nos permite calcular la altura del acantilado.

    \item \textbf{Velocidad pequeña:} La velocidad horizontal de Milada (95 cm/s) es muy pequeña comparada con la velocidad vertical final (34.3 m/s), por lo que su trayectoria es casi vertical.

    \item \textbf{Distancia proporcional:} La distancia horizontal es directamente proporcional a la velocidad horizontal inicial y al tiempo de caída.
\end{enumerate}

\section{Resumen de Fórmulas}

\begin{tcolorbox}[colback=yellow!10!white,colframe=orange!75!black,title=Fórmulas Utilizadas]

\textbf{Altura desde tiempo de caída libre:}
\begin{equation*}
    h = \frac{1}{2}gt^2
\end{equation*}

\textbf{Tiempo de caída desde altura conocida:}
\begin{equation*}
    t = \sqrt{\frac{2h}{g}}
\end{equation*}

\textbf{Alcance horizontal (lanzamiento horizontal):}
\begin{equation*}
    d = v_0 t = v_0 \sqrt{\frac{2h}{g}}
\end{equation*}

\textbf{Para este problema:}
\begin{align*}
    t &= 3.50 \text{ s} \\
    h &= 60.0 \text{ m} \\
    v_0 &= 0.950 \text{ m/s} \\
    d &= 3.33 \text{ m}
\end{align*}

\end{tcolorbox}

\section{Interpretación Biológica}

Este problema ilustra una realidad interesante:

\begin{itemize}
    \item \textbf{Estrategia de Chirpy:} Caída vertical directa, llega a la base del acantilado.

    \item \textbf{Estrategia de Milada:} Salto horizontal, aterriza 3.33 m más lejos, potencialmente evitando obstáculos o depredadores en la base del acantilado.

    \item \textbf{Velocidad de impacto similar:} Ambos impactan con la misma rapidez (≈34 m/s), por lo que el riesgo es similar.

    \item \textbf{Ventaja de Milada:} El pequeño impulso horizontal le permite alejarse de la base, lo que puede ser ventajoso para escapar.
\end{itemize}

\end{document}
