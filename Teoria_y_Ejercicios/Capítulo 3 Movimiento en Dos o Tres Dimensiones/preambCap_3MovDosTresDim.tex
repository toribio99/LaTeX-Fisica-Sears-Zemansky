% ================================================================================
% PREÁMBULO DEL CAPÍTULO 3: MOVIMIENTO EN DOS O TRES DIMENSIONES
% Física Universitaria - Sears y Zemansky
% ================================================================================
%
% Este archivo contiene todas las configuraciones, paquetes y definiciones
% necesarias para compilar los ejercicios del Capítulo 3.
%
% AUTOR: Generado automáticamente
% FECHA: \today
% VERSIÓN: 1.0
%
% MODO DE USO:
%   Este archivo debe ser invocado desde main.tex usando:
%   % ================================================================================
% PREÁMBULO DEL CAPÍTULO 3: MOVIMIENTO EN DOS O TRES DIMENSIONES
% Física Universitaria - Sears y Zemansky
% ================================================================================
%
% Este archivo contiene todas las configuraciones, paquetes y definiciones
% necesarias para compilar los ejercicios del Capítulo 3.
%
% AUTOR: Generado automáticamente
% FECHA: \today
% VERSIÓN: 1.0
%
% MODO DE USO:
%   Este archivo debe ser invocado desde main.tex usando:
%   % ================================================================================
% PREÁMBULO DEL CAPÍTULO 3: MOVIMIENTO EN DOS O TRES DIMENSIONES
% Física Universitaria - Sears y Zemansky
% ================================================================================
%
% Este archivo contiene todas las configuraciones, paquetes y definiciones
% necesarias para compilar los ejercicios del Capítulo 3.
%
% AUTOR: Generado automáticamente
% FECHA: \today
% VERSIÓN: 1.0
%
% MODO DE USO:
%   Este archivo debe ser invocado desde main.tex usando:
%   % ================================================================================
% PREÁMBULO DEL CAPÍTULO 3: MOVIMIENTO EN DOS O TRES DIMENSIONES
% Física Universitaria - Sears y Zemansky
% ================================================================================
%
% Este archivo contiene todas las configuraciones, paquetes y definiciones
% necesarias para compilar los ejercicios del Capítulo 3.
%
% AUTOR: Generado automáticamente
% FECHA: \today
% VERSIÓN: 1.0
%
% MODO DE USO:
%   Este archivo debe ser invocado desde main.tex usando:
%   \input{preambCap_3MovDosTresDim}
%
% ================================================================================

% --------------------------------------------------------------------------------
% CLASE DEL DOCUMENTO
% --------------------------------------------------------------------------------
% Utilizamos la clase 'article' con tamaño de fuente 11pt y papel A4
\documentclass[11pt,a4paper]{article}

% --------------------------------------------------------------------------------
% CODIFICACIÓN Y FUENTES
% --------------------------------------------------------------------------------
% Configura la codificación de entrada como UTF-8 para soportar caracteres especiales
\usepackage[utf8]{inputenc}

% Configura la codificación de fuentes como T1 para mejor renderizado
\usepackage[T1]{fontenc}

% --------------------------------------------------------------------------------
% IDIOMA Y LOCALIZACIÓN
% --------------------------------------------------------------------------------
% Configura el idioma del documento como español con babel
\usepackage[spanish]{babel}

% --------------------------------------------------------------------------------
% GEOMETRÍA DE LA PÁGINA
% --------------------------------------------------------------------------------
% Configura los márgenes del documento
% Márgenes de 2.5 cm en todos los lados para mejor aprovechamiento del espacio
\usepackage[margin=2.5cm]{geometry}

% --------------------------------------------------------------------------------
% PAQUETES MATEMÁTICOS
% --------------------------------------------------------------------------------
% amsmath: Proporciona entornos y comandos mejorados para matemáticas
\usepackage{amsmath}

% amssymb: Proporciona símbolos matemáticos adicionales
\usepackage{amssymb}

% --------------------------------------------------------------------------------
% COLORES
% --------------------------------------------------------------------------------
% xcolor: Permite usar colores en el documento
\usepackage{xcolor}

% Definición de colores personalizados para el documento
\definecolor{azuloscuro}{RGB}{0,51,102}      % Color para marcos de datos
\definecolor{azulclaro}{RGB}{230,240,250}    % Color de fondo para datos
\definecolor{verdeoscuro}{RGB}{0,100,0}      % Color para marcos de soluciones
\definecolor{rojoclaro}{RGB}{255,230,230}    % Color de fondo para resultados

% --------------------------------------------------------------------------------
% CAJAS Y ENTORNOS PERSONALIZADOS
% --------------------------------------------------------------------------------
% tcolorbox: Permite crear cajas coloreadas y personalizadas
\usepackage{tcolorbox}
\tcbuselibrary{theorems,skins,breakable}

% Caja para presentar los datos del problema
\newtcolorbox{datosbox}{
    colback=azulclaro,           % Color de fondo
    colframe=azuloscuro,         % Color del marco
    fonttitle=\bfseries,         % Título en negrita
    title=Datos del Problema,    % Título de la caja
    sharp corners,               % Esquinas rectas
    boxrule=1pt                  % Grosor del marco
}

% Caja para el desarrollo de la solución
\newtcolorbox{solucionbox}{
    colback=white,               % Fondo blanco
    colframe=verdeoscuro,        % Marco verde oscuro
    fonttitle=\bfseries,         % Título en negrita
    title=Desarrollo de la Solución,
    sharp corners,               % Esquinas rectas
    boxrule=1pt,                 % Grosor del marco
    breakable                    % Permite dividir en páginas
}

% Caja para presentar el resultado final
\newtcolorbox{resultadobox}{
    colback=rojoclaro,           % Fondo rojo claro
    colframe=red!70!black,       % Marco rojo oscuro
    fonttitle=\bfseries,         % Título en negrita
    title=Resultado Final,       % Título de la caja
    sharp corners,               % Esquinas rectas
    boxrule=2pt                  % Marco más grueso para destacar
}

% --------------------------------------------------------------------------------
% GRÁFICOS Y FIGURAS
% --------------------------------------------------------------------------------
% graphicx: Permite incluir imágenes
\usepackage{graphicx}

% tikz: Sistema poderoso para crear gráficos vectoriales
\usepackage{tikz}

% pgfplots: Para crear gráficas y plots científicos
\usepackage{pgfplots}
\pgfplotsset{compat=1.18}      % Versión de compatibilidad

% Librerías adicionales de TikZ
\usetikzlibrary{arrows.meta,patterns,decorations.pathmorphing,calc}

% --------------------------------------------------------------------------------
% REFERENCIAS CRUZADAS Y ENLACES
% --------------------------------------------------------------------------------
% hyperref: Crea enlaces internos y externos en el documento
\usepackage{hyperref}
\hypersetup{
    colorlinks=true,             % Enlaces coloreados en lugar de cajas
    linkcolor=blue,              % Color de enlaces internos
    filecolor=magenta,           % Color de enlaces a archivos
    urlcolor=cyan,               % Color de URLs
    citecolor=green,             % Color de citas
    bookmarks=true,              % Crea marcadores en el PDF
    bookmarksnumbered=true,      % Numera los marcadores
    pdftitle={Capítulo 3: Movimiento en Dos o Tres Dimensiones},
    pdfauthor={Física Universitaria - Sears y Zemansky},
    pdfsubject={Ejercicios Resueltos},
    pdfkeywords={Física, Movimiento, Cinemática, Proyectiles}
}

% --------------------------------------------------------------------------------
% FORMATO DE TÍTULOS Y SECCIONES
% --------------------------------------------------------------------------------
% titlesec: Personaliza el formato de títulos
\usepackage{titlesec}

% Formato para capítulos de ejercicios
\titleformat{\section}
    {\normalfont\Large\bfseries\color{azuloscuro}}
    {Ejercicio \thesection:}
    {1em}
    {}

% --------------------------------------------------------------------------------
% ENCABEZADOS Y PIES DE PÁGINA
% --------------------------------------------------------------------------------
% fancyhdr: Personaliza encabezados y pies de página
\usepackage{fancyhdr}
\pagestyle{fancy}
\fancyhf{}                       % Limpia todos los encabezados y pies
\fancyhead[L]{\leftmark}         % Título de sección a la izquierda
\fancyhead[R]{Capítulo 3}        % "Capítulo 3" a la derecha
\fancyfoot[C]{\thepage}          % Número de página centrado
\renewcommand{\headrulewidth}{0.4pt}  % Línea del encabezado
\renewcommand{\footrulewidth}{0.4pt}  % Línea del pie de página

% --------------------------------------------------------------------------------
% COMANDOS PERSONALIZADOS
% --------------------------------------------------------------------------------
% Comando para vector con flecha
\newcommand{\vect}[1]{\vec{#1}}

% Comando para valor absoluto
\newcommand{\abs}[1]{\left|#1\right|}

% Comando para destacar resultados importantes
\newcommand{\importante}[1]{\textbf{\color{red!70!black}#1}}

% --------------------------------------------------------------------------------
% CONFIGURACIÓN DE LISTAS
% --------------------------------------------------------------------------------
% enumitem: Personaliza listas
\usepackage{enumitem}
\setlist[itemize]{leftmargin=*}
\setlist[enumerate]{leftmargin=*}

% --------------------------------------------------------------------------------
% FIN DEL PREÁMBULO
% --------------------------------------------------------------------------------

%
% ================================================================================

% --------------------------------------------------------------------------------
% CLASE DEL DOCUMENTO
% --------------------------------------------------------------------------------
% Utilizamos la clase 'article' con tamaño de fuente 11pt y papel A4
\documentclass[11pt,a4paper]{article}

% --------------------------------------------------------------------------------
% CODIFICACIÓN Y FUENTES
% --------------------------------------------------------------------------------
% Configura la codificación de entrada como UTF-8 para soportar caracteres especiales
\usepackage[utf8]{inputenc}

% Configura la codificación de fuentes como T1 para mejor renderizado
\usepackage[T1]{fontenc}

% --------------------------------------------------------------------------------
% IDIOMA Y LOCALIZACIÓN
% --------------------------------------------------------------------------------
% Configura el idioma del documento como español con babel
\usepackage[spanish]{babel}

% --------------------------------------------------------------------------------
% GEOMETRÍA DE LA PÁGINA
% --------------------------------------------------------------------------------
% Configura los márgenes del documento
% Márgenes de 2.5 cm en todos los lados para mejor aprovechamiento del espacio
\usepackage[margin=2.5cm]{geometry}

% --------------------------------------------------------------------------------
% PAQUETES MATEMÁTICOS
% --------------------------------------------------------------------------------
% amsmath: Proporciona entornos y comandos mejorados para matemáticas
\usepackage{amsmath}

% amssymb: Proporciona símbolos matemáticos adicionales
\usepackage{amssymb}

% --------------------------------------------------------------------------------
% COLORES
% --------------------------------------------------------------------------------
% xcolor: Permite usar colores en el documento
\usepackage{xcolor}

% Definición de colores personalizados para el documento
\definecolor{azuloscuro}{RGB}{0,51,102}      % Color para marcos de datos
\definecolor{azulclaro}{RGB}{230,240,250}    % Color de fondo para datos
\definecolor{verdeoscuro}{RGB}{0,100,0}      % Color para marcos de soluciones
\definecolor{rojoclaro}{RGB}{255,230,230}    % Color de fondo para resultados

% --------------------------------------------------------------------------------
% CAJAS Y ENTORNOS PERSONALIZADOS
% --------------------------------------------------------------------------------
% tcolorbox: Permite crear cajas coloreadas y personalizadas
\usepackage{tcolorbox}
\tcbuselibrary{theorems,skins,breakable}

% Caja para presentar los datos del problema
\newtcolorbox{datosbox}{
    colback=azulclaro,           % Color de fondo
    colframe=azuloscuro,         % Color del marco
    fonttitle=\bfseries,         % Título en negrita
    title=Datos del Problema,    % Título de la caja
    sharp corners,               % Esquinas rectas
    boxrule=1pt                  % Grosor del marco
}

% Caja para el desarrollo de la solución
\newtcolorbox{solucionbox}{
    colback=white,               % Fondo blanco
    colframe=verdeoscuro,        % Marco verde oscuro
    fonttitle=\bfseries,         % Título en negrita
    title=Desarrollo de la Solución,
    sharp corners,               % Esquinas rectas
    boxrule=1pt,                 % Grosor del marco
    breakable                    % Permite dividir en páginas
}

% Caja para presentar el resultado final
\newtcolorbox{resultadobox}{
    colback=rojoclaro,           % Fondo rojo claro
    colframe=red!70!black,       % Marco rojo oscuro
    fonttitle=\bfseries,         % Título en negrita
    title=Resultado Final,       % Título de la caja
    sharp corners,               % Esquinas rectas
    boxrule=2pt                  % Marco más grueso para destacar
}

% --------------------------------------------------------------------------------
% GRÁFICOS Y FIGURAS
% --------------------------------------------------------------------------------
% graphicx: Permite incluir imágenes
\usepackage{graphicx}

% tikz: Sistema poderoso para crear gráficos vectoriales
\usepackage{tikz}

% pgfplots: Para crear gráficas y plots científicos
\usepackage{pgfplots}
\pgfplotsset{compat=1.18}      % Versión de compatibilidad

% Librerías adicionales de TikZ
\usetikzlibrary{arrows.meta,patterns,decorations.pathmorphing,calc}

% --------------------------------------------------------------------------------
% REFERENCIAS CRUZADAS Y ENLACES
% --------------------------------------------------------------------------------
% hyperref: Crea enlaces internos y externos en el documento
\usepackage{hyperref}
\hypersetup{
    colorlinks=true,             % Enlaces coloreados en lugar de cajas
    linkcolor=blue,              % Color de enlaces internos
    filecolor=magenta,           % Color de enlaces a archivos
    urlcolor=cyan,               % Color de URLs
    citecolor=green,             % Color de citas
    bookmarks=true,              % Crea marcadores en el PDF
    bookmarksnumbered=true,      % Numera los marcadores
    pdftitle={Capítulo 3: Movimiento en Dos o Tres Dimensiones},
    pdfauthor={Física Universitaria - Sears y Zemansky},
    pdfsubject={Ejercicios Resueltos},
    pdfkeywords={Física, Movimiento, Cinemática, Proyectiles}
}

% --------------------------------------------------------------------------------
% FORMATO DE TÍTULOS Y SECCIONES
% --------------------------------------------------------------------------------
% titlesec: Personaliza el formato de títulos
\usepackage{titlesec}

% Formato para capítulos de ejercicios
\titleformat{\section}
    {\normalfont\Large\bfseries\color{azuloscuro}}
    {Ejercicio \thesection:}
    {1em}
    {}

% --------------------------------------------------------------------------------
% ENCABEZADOS Y PIES DE PÁGINA
% --------------------------------------------------------------------------------
% fancyhdr: Personaliza encabezados y pies de página
\usepackage{fancyhdr}
\pagestyle{fancy}
\fancyhf{}                       % Limpia todos los encabezados y pies
\fancyhead[L]{\leftmark}         % Título de sección a la izquierda
\fancyhead[R]{Capítulo 3}        % "Capítulo 3" a la derecha
\fancyfoot[C]{\thepage}          % Número de página centrado
\renewcommand{\headrulewidth}{0.4pt}  % Línea del encabezado
\renewcommand{\footrulewidth}{0.4pt}  % Línea del pie de página

% --------------------------------------------------------------------------------
% COMANDOS PERSONALIZADOS
% --------------------------------------------------------------------------------
% Comando para vector con flecha
\newcommand{\vect}[1]{\vec{#1}}

% Comando para valor absoluto
\newcommand{\abs}[1]{\left|#1\right|}

% Comando para destacar resultados importantes
\newcommand{\importante}[1]{\textbf{\color{red!70!black}#1}}

% --------------------------------------------------------------------------------
% CONFIGURACIÓN DE LISTAS
% --------------------------------------------------------------------------------
% enumitem: Personaliza listas
\usepackage{enumitem}
\setlist[itemize]{leftmargin=*}
\setlist[enumerate]{leftmargin=*}

% --------------------------------------------------------------------------------
% FIN DEL PREÁMBULO
% --------------------------------------------------------------------------------

%
% ================================================================================

% --------------------------------------------------------------------------------
% CLASE DEL DOCUMENTO
% --------------------------------------------------------------------------------
% Utilizamos la clase 'article' con tamaño de fuente 11pt y papel A4
\documentclass[11pt,a4paper]{article}

% --------------------------------------------------------------------------------
% CODIFICACIÓN Y FUENTES
% --------------------------------------------------------------------------------
% Configura la codificación de entrada como UTF-8 para soportar caracteres especiales
\usepackage[utf8]{inputenc}

% Configura la codificación de fuentes como T1 para mejor renderizado
\usepackage[T1]{fontenc}

% --------------------------------------------------------------------------------
% IDIOMA Y LOCALIZACIÓN
% --------------------------------------------------------------------------------
% Configura el idioma del documento como español con babel
\usepackage[spanish]{babel}

% --------------------------------------------------------------------------------
% GEOMETRÍA DE LA PÁGINA
% --------------------------------------------------------------------------------
% Configura los márgenes del documento
% Márgenes de 2.5 cm en todos los lados para mejor aprovechamiento del espacio
\usepackage[margin=2.5cm]{geometry}

% --------------------------------------------------------------------------------
% PAQUETES MATEMÁTICOS
% --------------------------------------------------------------------------------
% amsmath: Proporciona entornos y comandos mejorados para matemáticas
\usepackage{amsmath}

% amssymb: Proporciona símbolos matemáticos adicionales
\usepackage{amssymb}

% --------------------------------------------------------------------------------
% COLORES
% --------------------------------------------------------------------------------
% xcolor: Permite usar colores en el documento
\usepackage{xcolor}

% Definición de colores personalizados para el documento
\definecolor{azuloscuro}{RGB}{0,51,102}      % Color para marcos de datos
\definecolor{azulclaro}{RGB}{230,240,250}    % Color de fondo para datos
\definecolor{verdeoscuro}{RGB}{0,100,0}      % Color para marcos de soluciones
\definecolor{rojoclaro}{RGB}{255,230,230}    % Color de fondo para resultados

% --------------------------------------------------------------------------------
% CAJAS Y ENTORNOS PERSONALIZADOS
% --------------------------------------------------------------------------------
% tcolorbox: Permite crear cajas coloreadas y personalizadas
\usepackage{tcolorbox}
\tcbuselibrary{theorems,skins,breakable}

% Caja para presentar los datos del problema
\newtcolorbox{datosbox}{
    colback=azulclaro,           % Color de fondo
    colframe=azuloscuro,         % Color del marco
    fonttitle=\bfseries,         % Título en negrita
    title=Datos del Problema,    % Título de la caja
    sharp corners,               % Esquinas rectas
    boxrule=1pt                  % Grosor del marco
}

% Caja para el desarrollo de la solución
\newtcolorbox{solucionbox}{
    colback=white,               % Fondo blanco
    colframe=verdeoscuro,        % Marco verde oscuro
    fonttitle=\bfseries,         % Título en negrita
    title=Desarrollo de la Solución,
    sharp corners,               % Esquinas rectas
    boxrule=1pt,                 % Grosor del marco
    breakable                    % Permite dividir en páginas
}

% Caja para presentar el resultado final
\newtcolorbox{resultadobox}{
    colback=rojoclaro,           % Fondo rojo claro
    colframe=red!70!black,       % Marco rojo oscuro
    fonttitle=\bfseries,         % Título en negrita
    title=Resultado Final,       % Título de la caja
    sharp corners,               % Esquinas rectas
    boxrule=2pt                  % Marco más grueso para destacar
}

% --------------------------------------------------------------------------------
% GRÁFICOS Y FIGURAS
% --------------------------------------------------------------------------------
% graphicx: Permite incluir imágenes
\usepackage{graphicx}

% tikz: Sistema poderoso para crear gráficos vectoriales
\usepackage{tikz}

% pgfplots: Para crear gráficas y plots científicos
\usepackage{pgfplots}
\pgfplotsset{compat=1.18}      % Versión de compatibilidad

% Librerías adicionales de TikZ
\usetikzlibrary{arrows.meta,patterns,decorations.pathmorphing,calc}

% --------------------------------------------------------------------------------
% REFERENCIAS CRUZADAS Y ENLACES
% --------------------------------------------------------------------------------
% hyperref: Crea enlaces internos y externos en el documento
\usepackage{hyperref}
\hypersetup{
    colorlinks=true,             % Enlaces coloreados en lugar de cajas
    linkcolor=blue,              % Color de enlaces internos
    filecolor=magenta,           % Color de enlaces a archivos
    urlcolor=cyan,               % Color de URLs
    citecolor=green,             % Color de citas
    bookmarks=true,              % Crea marcadores en el PDF
    bookmarksnumbered=true,      % Numera los marcadores
    pdftitle={Capítulo 3: Movimiento en Dos o Tres Dimensiones},
    pdfauthor={Física Universitaria - Sears y Zemansky},
    pdfsubject={Ejercicios Resueltos},
    pdfkeywords={Física, Movimiento, Cinemática, Proyectiles}
}

% --------------------------------------------------------------------------------
% FORMATO DE TÍTULOS Y SECCIONES
% --------------------------------------------------------------------------------
% titlesec: Personaliza el formato de títulos
\usepackage{titlesec}

% Formato para capítulos de ejercicios
\titleformat{\section}
    {\normalfont\Large\bfseries\color{azuloscuro}}
    {Ejercicio \thesection:}
    {1em}
    {}

% --------------------------------------------------------------------------------
% ENCABEZADOS Y PIES DE PÁGINA
% --------------------------------------------------------------------------------
% fancyhdr: Personaliza encabezados y pies de página
\usepackage{fancyhdr}
\pagestyle{fancy}
\fancyhf{}                       % Limpia todos los encabezados y pies
\fancyhead[L]{\leftmark}         % Título de sección a la izquierda
\fancyhead[R]{Capítulo 3}        % "Capítulo 3" a la derecha
\fancyfoot[C]{\thepage}          % Número de página centrado
\renewcommand{\headrulewidth}{0.4pt}  % Línea del encabezado
\renewcommand{\footrulewidth}{0.4pt}  % Línea del pie de página

% --------------------------------------------------------------------------------
% COMANDOS PERSONALIZADOS
% --------------------------------------------------------------------------------
% Comando para vector con flecha
\newcommand{\vect}[1]{\vec{#1}}

% Comando para valor absoluto
\newcommand{\abs}[1]{\left|#1\right|}

% Comando para destacar resultados importantes
\newcommand{\importante}[1]{\textbf{\color{red!70!black}#1}}

% --------------------------------------------------------------------------------
% CONFIGURACIÓN DE LISTAS
% --------------------------------------------------------------------------------
% enumitem: Personaliza listas
\usepackage{enumitem}
\setlist[itemize]{leftmargin=*}
\setlist[enumerate]{leftmargin=*}

% --------------------------------------------------------------------------------
% FIN DEL PREÁMBULO
% --------------------------------------------------------------------------------

%
% ================================================================================

% --------------------------------------------------------------------------------
% CLASE DEL DOCUMENTO
% --------------------------------------------------------------------------------
% Utilizamos la clase 'article' con tamaño de fuente 11pt y papel A4
\documentclass[11pt,a4paper]{article}

% --------------------------------------------------------------------------------
% CODIFICACIÓN Y FUENTES
% --------------------------------------------------------------------------------
% Configura la codificación de entrada como UTF-8 para soportar caracteres especiales
\usepackage[utf8]{inputenc}

% Configura la codificación de fuentes como T1 para mejor renderizado
\usepackage[T1]{fontenc}

% --------------------------------------------------------------------------------
% IDIOMA Y LOCALIZACIÓN
% --------------------------------------------------------------------------------
% Configura el idioma del documento como español con babel
\usepackage[spanish]{babel}

% --------------------------------------------------------------------------------
% GEOMETRÍA DE LA PÁGINA
% --------------------------------------------------------------------------------
% Configura los márgenes del documento
% Márgenes de 2.5 cm en todos los lados para mejor aprovechamiento del espacio
\usepackage[margin=2.5cm]{geometry}

% --------------------------------------------------------------------------------
% PAQUETES MATEMÁTICOS
% --------------------------------------------------------------------------------
% amsmath: Proporciona entornos y comandos mejorados para matemáticas
\usepackage{amsmath}

% amssymb: Proporciona símbolos matemáticos adicionales
\usepackage{amssymb}

% --------------------------------------------------------------------------------
% COLORES
% --------------------------------------------------------------------------------
% xcolor: Permite usar colores en el documento
\usepackage{xcolor}

% Definición de colores personalizados para el documento
\definecolor{azuloscuro}{RGB}{0,51,102}      % Color para marcos de datos
\definecolor{azulclaro}{RGB}{230,240,250}    % Color de fondo para datos
\definecolor{verdeoscuro}{RGB}{0,100,0}      % Color para marcos de soluciones
\definecolor{rojoclaro}{RGB}{255,230,230}    % Color de fondo para resultados

% --------------------------------------------------------------------------------
% CAJAS Y ENTORNOS PERSONALIZADOS
% --------------------------------------------------------------------------------
% tcolorbox: Permite crear cajas coloreadas y personalizadas
\usepackage{tcolorbox}
\tcbuselibrary{theorems,skins,breakable}

% Caja para presentar los datos del problema
\newtcolorbox{datosbox}{
    colback=azulclaro,           % Color de fondo
    colframe=azuloscuro,         % Color del marco
    fonttitle=\bfseries,         % Título en negrita
    title=Datos del Problema,    % Título de la caja
    sharp corners,               % Esquinas rectas
    boxrule=1pt                  % Grosor del marco
}

% Caja para el desarrollo de la solución
\newtcolorbox{solucionbox}{
    colback=white,               % Fondo blanco
    colframe=verdeoscuro,        % Marco verde oscuro
    fonttitle=\bfseries,         % Título en negrita
    title=Desarrollo de la Solución,
    sharp corners,               % Esquinas rectas
    boxrule=1pt,                 % Grosor del marco
    breakable                    % Permite dividir en páginas
}

% Caja para presentar el resultado final
\newtcolorbox{resultadobox}{
    colback=rojoclaro,           % Fondo rojo claro
    colframe=red!70!black,       % Marco rojo oscuro
    fonttitle=\bfseries,         % Título en negrita
    title=Resultado Final,       % Título de la caja
    sharp corners,               % Esquinas rectas
    boxrule=2pt                  % Marco más grueso para destacar
}

% --------------------------------------------------------------------------------
% GRÁFICOS Y FIGURAS
% --------------------------------------------------------------------------------
% graphicx: Permite incluir imágenes
\usepackage{graphicx}

% tikz: Sistema poderoso para crear gráficos vectoriales
\usepackage{tikz}

% pgfplots: Para crear gráficas y plots científicos
\usepackage{pgfplots}
\pgfplotsset{compat=1.18}      % Versión de compatibilidad

% Librerías adicionales de TikZ
\usetikzlibrary{arrows.meta,patterns,decorations.pathmorphing,calc}

% --------------------------------------------------------------------------------
% REFERENCIAS CRUZADAS Y ENLACES
% --------------------------------------------------------------------------------
% hyperref: Crea enlaces internos y externos en el documento
\usepackage{hyperref}
\hypersetup{
    colorlinks=true,             % Enlaces coloreados en lugar de cajas
    linkcolor=blue,              % Color de enlaces internos
    filecolor=magenta,           % Color de enlaces a archivos
    urlcolor=cyan,               % Color de URLs
    citecolor=green,             % Color de citas
    bookmarks=true,              % Crea marcadores en el PDF
    bookmarksnumbered=true,      % Numera los marcadores
    pdftitle={Capítulo 3: Movimiento en Dos o Tres Dimensiones},
    pdfauthor={Física Universitaria - Sears y Zemansky},
    pdfsubject={Ejercicios Resueltos},
    pdfkeywords={Física, Movimiento, Cinemática, Proyectiles}
}

% --------------------------------------------------------------------------------
% FORMATO DE TÍTULOS Y SECCIONES
% --------------------------------------------------------------------------------
% titlesec: Personaliza el formato de títulos
\usepackage{titlesec}

% Formato para capítulos de ejercicios
\titleformat{\section}
    {\normalfont\Large\bfseries\color{azuloscuro}}
    {Ejercicio \thesection:}
    {1em}
    {}

% --------------------------------------------------------------------------------
% ENCABEZADOS Y PIES DE PÁGINA
% --------------------------------------------------------------------------------
% fancyhdr: Personaliza encabezados y pies de página
\usepackage{fancyhdr}
\pagestyle{fancy}
\fancyhf{}                       % Limpia todos los encabezados y pies
\fancyhead[L]{\leftmark}         % Título de sección a la izquierda
\fancyhead[R]{Capítulo 3}        % "Capítulo 3" a la derecha
\fancyfoot[C]{\thepage}          % Número de página centrado
\renewcommand{\headrulewidth}{0.4pt}  % Línea del encabezado
\renewcommand{\footrulewidth}{0.4pt}  % Línea del pie de página

% --------------------------------------------------------------------------------
% COMANDOS PERSONALIZADOS
% --------------------------------------------------------------------------------
% Comando para vector con flecha
\newcommand{\vect}[1]{\vec{#1}}

% Comando para valor absoluto
\newcommand{\abs}[1]{\left|#1\right|}

% Comando para destacar resultados importantes
\newcommand{\importante}[1]{\textbf{\color{red!70!black}#1}}

% --------------------------------------------------------------------------------
% CONFIGURACIÓN DE LISTAS
% --------------------------------------------------------------------------------
% enumitem: Personaliza listas
\usepackage{enumitem}
\setlist[itemize]{leftmargin=*}
\setlist[enumerate]{leftmargin=*}

% --------------------------------------------------------------------------------
% FIN DEL PREÁMBULO
% --------------------------------------------------------------------------------
