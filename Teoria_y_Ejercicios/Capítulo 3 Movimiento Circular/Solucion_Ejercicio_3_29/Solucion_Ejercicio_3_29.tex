\documentclass[11pt,a4paper]{article}

% Paquetes necesarios
\usepackage[utf8]{inputenc}
\usepackage[T1]{fontenc}
\usepackage[spanish]{babel}
\usepackage[margin=2.5cm]{geometry}
\usepackage{amsmath}
\usepackage{amssymb}
\usepackage{xcolor}
\usepackage{tcolorbox}
\usepackage{graphicx}
\usepackage{tikz}
\usepackage{pgfplots}
\pgfplotsset{compat=1.18}
\usetikzlibrary{arrows.meta,patterns,decorations.pathmorphing,calc}
\usepackage{wrapfig}
\usepackage[export]{adjustbox} % (opcional) claves extra para \includegraphics
\usepackage{xparse}

% Definición de colores
\definecolor{azuloscuro}{RGB}{0,51,102}
\definecolor{azulclaro}{RGB}{230,240,250}
\definecolor{verdeoscuro}{RGB}{0,100,0}
\definecolor{rojoclaro}{RGB}{255,230,230}

% Configuración de cajas
\tcbuselibrary{theorems,skins,breakable}

\newtcolorbox{datosbox}{
    colback=azulclaro,
    colframe=azuloscuro,
    fonttitle=\bfseries,
    title=Datos del Problema,
    sharp corners,
    boxrule=1pt
}

\newtcolorbox{solucionbox}{
    colback=white,
    colframe=verdeoscuro,
    fonttitle=\bfseries,
    title=Desarrollo de la Solución,
    sharp corners,
    boxrule=1pt,
    breakable
}

\newtcolorbox{resultadobox}{
    colback=rojoclaro,
    colframe=red!70!black,
    fonttitle=\bfseries,
    title=Resultado Final,
    sharp corners,
    boxrule=2pt
}

% Título y autor
\title{\textbf{Solución del Ejercicio 3.29} \\
\large Movimiento Circular: Rotación de la Tierra}
\author{Física Universitaria - Sears y Zemansky \\ Capítulo 3: Movimiento en Dos o Tres Dimensiones}
\date{\today}

\begin{document}

\maketitle

\section{Enunciado del Problema}

\begin{wrapfigure}[10]{r}{.45\textwidth}  % Ajusta el ancho según necesites
	\centering
	\vspace{-\baselineskip}
	%\begin{center}
\begin{tikzpicture}[scale=.85]
    % Tierra (vista desde el polo)
    \shade[ball color=blue!40!cyan!60, opacity=0.8] (0,0) circle (2.5);

    % Eje de rotación (indicado)
    \draw[dashed, thick] (0,-2.8) -- (0,2.8);
    \node[below] at (0,-2.8) {Polo Sur};
    \node[above] at (0,2.8) {Polo Norte};

    % Ecuador
    \draw[red!70!black, very thick] (-2.5,0) arc (180:360:2.5 and 0.5);
    \draw[red!70!black, very thick, dashed] (-2.5,0) arc (180:0:2.5 and 0.5);
    \node[red!70!black, right] at (2.8,0) {Ecuador};

    % Radio de la Tierra
    \draw[-{Latex[length=2mm]}, blue!70!black, thick] (0,0) -- (2.5,0);
    \node[blue!70!black, above] at (0.65,0) {$R$};

    % Objeto en el ecuador
    \filldraw[orange!80!black] (2.5,0) circle (0.12);
    \node[orange!80!black, right] at (2.7,0.3) {Objeto};

    % Vector de velocidad tangencial
    \draw[-{Latex[length=3mm]}, green!60!black, ultra thick] (2.5,0) -- (2.5,1.2);
    \node[green!60!black, right] at (2.5,0.8) {$\vec{v}$};

    % Vector de aceleración centrípeta
    \draw[-{Latex[length=3mm]}, red!70!black, ultra thick] (2.5,0) -- (1.3,0);
    \node[white, above] at (1.9,-0.1) {$\vec{a}_{\text{rad}}$};

    % Flecha de rotación
    \draw[-{Latex[length=3mm]}, purple!70!black, very thick]
        (0.8,2.2) arc (60:120:1.5);
    \node[purple!70!black] at (0.25,2.68) {$\omega$};

    % Etiqueta
    \node[below, align=center] at (0,-3.8) {\textbf{Vista desde el polo} \\ (rotación alrededor del eje)};

\end{tikzpicture}
%\end{center}
\vspace{-\baselineskip} % Reduce espacio después
\end{wrapfigure}
La Tierra tiene 6380 km de radio y gira una vez sobre su eje en 24 h. \\[.6mm]

a) ?`Qué aceleración radial tiene un objeto en el ecuador? Dé su respuesta en m/s$^2$ y como fracción de $g$. \\[.6mm]

b) Si $a_{\text{rad}}$ en el ecuador fuera mayor que $g$, los objetos saldrían volando hacia el espacio. (Veremos por qué en el capítulo 5.) ?`Cuál tendría que ser el periodo de rotación para que esto sucediera? \\[.6mm]

\vspace{3mm}

\section{Datos del Problema}

\begin{datosbox}
\begin{itemize}
    \item \textbf{Radio de la Tierra:} $R = 6380$ km $= 6.38 \times 10^6$ m
    \item \textbf{Periodo de rotación:} $T = 24$ h $= 86\,400$ s
    \item \textbf{Aceleración gravitacional:} $g = 9.8$ m/s$^2$
    \item \textbf{Localización:} Ecuador (distancia máxima del eje de rotación)
\end{itemize}
\end{datosbox}

\section{Marco Teórico}

\subsection{Aceleración Centrípeta}

Un objeto en el ecuador de la Tierra está en movimiento circular uniforme. La aceleración centrípeta en términos del radio y el periodo es:

\begin{equation*}
    a_{\text{rad}} = \frac{4\pi^2 R}{T^2}
\end{equation*}

Esta aceleración está dirigida hacia el centro de la Tierra.

\subsection{Concepto Importante}

La aceleración centrípeta es la aceleración necesaria para mantener un objeto en movimiento circular. En el ecuador:
\begin{itemize}
    \item La gravedad proporciona esta aceleración centrípeta
    \item Si la Tierra girara más rápido, se necesitaría mayor aceleración centrípeta
    \item Si $a_{\text{rad}} > g$, la gravedad sería insuficiente para mantener los objetos en la superficie
\end{itemize}

\section{Desarrollo de la Solución}

\begin{solucionbox}

\subsection*{Parte a) Aceleración radial en el ecuador}

\subsubsection*{Paso 1: Aplicar la fórmula}

Sustituimos los valores en la ecuación de aceleración centrípeta:

\begin{align*}
    a_{\text{rad}} &= \frac{4\pi^2 R}{T^2} \\
    &= \frac{4\pi^2 (6.38 \times 10^6 \text{ m})}{(86\,400 \text{ s})^2}
\end{align*}

\subsubsection*{Paso 2: Calcular el numerador}

\begin{align*}
    \text{Numerador} &= 4\pi^2 \times 6.38 \times 10^6 \\
    &= 4 \times 9.8696 \times 6.38 \times 10^6 \\
    &= 251.9 \times 10^6 \text{ m} \\
    &= 2.519 \times 10^8 \text{ m}
\end{align*}

\subsubsection*{Paso 3: Calcular el denominador}

\begin{align*}
    T^2 &= (86\,400 \text{ s})^2 \\
    &= 7.466 \times 10^9 \text{ s}^2
\end{align*}

\subsubsection*{Paso 4: Calcular la aceleración}

\begin{align*}
    a_{\text{rad}} &= \frac{2.519 \times 10^8}{7.466 \times 10^9} \\
    &= 0.0337 \text{ m/s}^2 \\
    &\approx 0.034 \text{ m/s}^2
\end{align*}

\subsubsection*{Paso 5: Expresar como fracción de $g$}

\begin{align*}
    \frac{a_{\text{rad}}}{g} &= \frac{0.0337}{9.8} \\
    &= 0.00344 \\
    &\approx \frac{1}{291}
\end{align*}

Por lo tanto:
\begin{equation*}
    a_{\text{rad}} \approx 0.0034\,g \quad \text{o} \quad \frac{g}{291}
\end{equation*}

\textbf{Interpretación:} La aceleración centrípeta en el ecuador es aproximadamente $\frac{1}{300}$ de la aceleración gravitacional. Esto es muy pequeño, por eso no notamos el efecto de la rotación terrestre en nuestra vida diaria.

\subsection*{Parte b) Periodo para que $a_{\text{rad}} = g$}

\subsubsection*{Paso 1: Plantear la condición}

Para que los objetos "salgan volando", necesitamos:
\begin{equation*}
    a_{\text{rad}} = g
\end{equation*}

Sustituyendo en la ecuación de aceleración centrípeta:
\begin{equation*}
    \frac{4\pi^2 R}{T^2} = g
\end{equation*}

\subsubsection*{Paso 2: Despejar el periodo $T$}

\begin{align*}
    T^2 &= \frac{4\pi^2 R}{g} \\
    T &= \sqrt{\frac{4\pi^2 R}{g}} \\
    T &= 2\pi\sqrt{\frac{R}{g}}
\end{align*}

\subsubsection*{Paso 3: Sustituir valores}

\begin{align*}
    T &= 2\pi\sqrt{\frac{6.38 \times 10^6 \text{ m}}{9.8 \text{ m/s}^2}} \\
    &= 2\pi\sqrt{6.51 \times 10^5 \text{ s}^2} \\
    &= 2\pi \times 807 \text{ s} \\
    &= 5069 \text{ s}
\end{align*}

\subsubsection*{Paso 4: Convertir a horas}

\begin{equation*}
    T = \frac{5069 \text{ s}}{3600 \text{ s/h}} = 1.41 \text{ h} \approx 1.4 \text{ h}
\end{equation*}

O en minutos:
\begin{equation*}
    T = \frac{5069 \text{ s}}{60 \text{ s/min}} = 84.5 \text{ min} \approx 85 \text{ min}
\end{equation*}

\textbf{Interpretación:} Si la Tierra completara una rotación en aproximadamente 1.4 horas (84.5 minutos) en lugar de 24 horas, la aceleración centrípeta en el ecuador sería igual a $g$, y los objetos no estarían "pegados" a la superficie por la gravedad.

\subsection*{Verificación}

Verifiquemos que con $T = 5069$ s, la aceleración es igual a $g$:

\begin{align*}
    a_{\text{rad}} &= \frac{4\pi^2 (6.38 \times 10^6)}{(5069)^2} \\
    &= \frac{2.519 \times 10^8}{2.569 \times 10^7} \\
    &= 9.8 \text{ m/s}^2 = g \quad \checkmark
\end{align*}

\end{solucionbox}

\section{Resultados Finales}

\begin{resultadobox}

\subsection*{Parte a) Aceleración radial en el ecuador}

\textbf{En m/s$^2$:}
\begin{equation*}
    \boxed{a_{\text{rad}} = 0.034 \text{ m/s}^2}
\end{equation*}

\textbf{Como fracción de $g$:}
\begin{equation*}
    \boxed{a_{\text{rad}} = 0.0034\,g \approx \frac{g}{291}}
\end{equation*}

\subsection*{Parte b) Periodo crítico}

Para que $a_{\text{rad}} = g$ (objetos saldrían volando):

\begin{equation*}
    \boxed{T = 5069 \text{ s} = 84.5 \text{ min} = 1.41 \text{ h}}
\end{equation*}

\textbf{Comparación:} El periodo actual de 24 h tendría que reducirse a aproximadamente 1.4 h, es decir, \textbf{la Tierra tendría que girar 17 veces más rápido}.

\end{resultadobox}

\section{Análisis Adicional}

\subsection{Comparación de periodos}

\begin{center}
\begin{tikzpicture}
\begin{axis}[
    width=0.9\textwidth,
    height=7cm,
    xlabel={Periodo de rotación (horas)},
    ylabel={Aceleración radial / $g$},
    xmin=0, xmax=25,
    ymin=0, ymax=1.2,
    grid=major,
    legend pos=north east,
    title={Aceleración centrípeta en el ecuador vs periodo de rotación}
]

% Curva arad/g vs T
\addplot[blue, thick, domain=1.4:24, samples=100] {(24/x)^2 * 0.0034};

% Línea horizontal en arad/g = 1
\addplot[red, thick, dashed] coordinates {(0,1) (25,1)};

% Punto actual (T=24h)
\addplot[green!60!black, only marks, mark=*, mark size=3pt] coordinates {(24, 0.0034)};

% Punto crítico (T=1.41h)
\addplot[orange, only marks, mark=*, mark size=3pt] coordinates {(1.41, 1.0)};

\legend{$a_{\text{rad}}/g$, $a_{\text{rad}} = g$ (crítico), Actual (24 h), Crítico (1.41 h)}

\end{axis}
\end{tikzpicture}
\end{center}

\subsection{Efectos de la rotación terrestre}

Aunque la aceleración centrípeta es pequeña ($\sim 0.3\%$ de $g$), tiene efectos observables:

\begin{enumerate}
    \item \textbf{Forma de la Tierra:} La Tierra no es perfectamente esférica; está achatada en los polos debido a la rotación

    \item \textbf{Variación de $g$:} El valor efectivo de $g$ es ligeramente menor en el ecuador que en los polos

    \item \textbf{Lanzamiento de cohetes:} Se prefiere lanzar desde el ecuador para aprovechar la velocidad tangencial adicional

    \item \textbf{Huracanes y tornados:} La rotación terrestre afecta la dirección de los vientos (efecto Coriolis)
\end{enumerate}

\subsection{?`Qué pasaría si la Tierra girara más rápido?}

\begin{center}
\begin{tabular}{|c|c|c|}
\hline
\textbf{Periodo (h)} & \textbf{$a_{\text{rad}}/g$} & \textbf{Efecto} \\
\hline
24 (actual) & 0.0034 & Normal \\
\hline
12 & 0.0136 & Peso aparente 1.4\% menor \\
\hline
6 & 0.054 & Peso aparente 5.4\% menor \\
\hline
3 & 0.217 & Peso aparente 22\% menor \\
\hline
1.5 & 0.867 & Peso aparente 87\% menor \\
\hline
1.41 (crítico) & 1.0 & Objetos flotan \\
\hline
$<$ 1.41 & $>$ 1.0 & Objetos salen volando \\
\hline
\end{tabular}
\end{center}

\section{Conceptos Clave}

\begin{enumerate}
    \item La aceleración centrípeta en el ecuador es muy pequeña comparada con $g$ (menos del 0.4\%)

    \item La aceleración centrípeta es inversamente proporcional al cuadrado del periodo: $a_{\text{rad}} \propto 1/T^2$

    \item Para que los objetos "salgan volando", la Tierra tendría que girar aproximadamente 17 veces más rápido

    \item El periodo crítico es $T = 2\pi\sqrt{R/g}$, una fórmula que relaciona el tamaño del planeta con la gravedad

    \item Aunque pequeña, la rotación tiene efectos importantes en la forma de la Tierra y en fenómenos meteorológicos
\end{enumerate}

\section{Curiosidad: Otros cuerpos celestes}

\begin{center}
\begin{tabular}{|l|c|c|c|}
\hline
\textbf{Cuerpo} & \textbf{Radio (km)} & \textbf{Periodo (h)} & \textbf{$a_{\text{rad}}/g_{\text{superficie}}$} \\
\hline
Tierra & 6\,380 & 24 & 0.0034 \\
\hline
Marte & 3\,396 & 24.6 & 0.0018 \\
\hline
Júpiter & 71\,492 & 9.9 & 0.089 \\
\hline
Saturno & 60\,268 & 10.7 & 0.14 \\
\hline
\end{tabular}
\end{center}

Júpiter y Saturno, al ser gigantes gaseosos con rotaciones rápidas, tienen aceleraciones centrípetas mucho mayores en proporción a su gravedad superficial.

\end{document}
