\documentclass[11pt,a4paper]{article}

% Paquetes necesarios
\usepackage[utf8]{inputenc}
\usepackage[T1]{fontenc}
\usepackage[spanish]{babel}
\usepackage[margin=2.5cm]{geometry}
\usepackage{amsmath}
\usepackage{amssymb}
\usepackage{xcolor}
\usepackage{tcolorbox}
\usepackage{graphicx}
\usepackage{tikz}
\usepackage{pgfplots}
\pgfplotsset{compat=1.18}
\usetikzlibrary{arrows.meta,patterns,decorations.pathmorphing,calc}
\usepackage{wrapfig}
\usepackage[export]{adjustbox} % (opcional) claves extra para \includegraphics
\usepackage{xparse}

% Definición de colores
\definecolor{azuloscuro}{RGB}{0,51,102}
\definecolor{azulclaro}{RGB}{230,240,250}
\definecolor{verdeoscuro}{RGB}{0,100,0}
\definecolor{rojoclaro}{RGB}{255,230,230}

% Configuración de cajas
\tcbuselibrary{theorems,skins,breakable}

\newtcolorbox{datosbox}{
    colback=azulclaro,
    colframe=azuloscuro,
    fonttitle=\bfseries,
    title=Datos del Problema,
    sharp corners,
    boxrule=1pt
}

\newtcolorbox{solucionbox}{
    colback=white,
    colframe=verdeoscuro,
    fonttitle=\bfseries,
    title=Desarrollo de la Solución,
    sharp corners,
    boxrule=1pt,
    breakable
}

\newtcolorbox{resultadobox}{
    colback=rojoclaro,
    colframe=red!70!black,
    fonttitle=\bfseries,
    title=Resultado Final,
    sharp corners,
    boxrule=2pt
}

% Título y autor
\title{\textbf{Solución del Ejercicio 3.32} \\
\large Movimiento Circular: Órbitas Planetarias}
\author{Física Universitaria - Sears y Zemansky \\ Capítulo 3: Movimiento en Dos o Tres Dimensiones}
\date{\today}

\begin{document}

\maketitle

\section{Enunciado del Problema}

\begin{wrapfigure}[13]{r}{.5\textwidth}  % Ajusta el ancho según necesites
	\centering
	\vspace{-\baselineskip}
	%\begin{center}
\begin{tikzpicture}[scale=.85]
    % Sol
    \shade[ball color=yellow!80!orange] (0,0) circle (0.6);
    \node[below] at (0,-0.8) {\textbf{Sol}};

    % Órbita de Mercurio
    \draw[blue!50, thick, dashed] (0,0) circle (2);
    \node[blue!50] at (0,2.45) {\small Mercurio};
    \filldraw[blue!60] (0,2) circle (0.12);

    % Órbita de la Tierra
    \draw[green!60!black, thick] (0,0) circle (4);
    \node[green!60!black] at (2,2.8) {\small Tierra};
    \filldraw[green!60!black] (4,0) circle (0.15);

    % Vectores para la Tierra
    \draw[-{Latex[length=2.5mm]}, purple!70!black, ultra thick]
        (4,0) -- (4,1.2);
    \node[purple!70!black, right] at (4,0.8) {$\vec{v}_T$};

    \draw[-{Latex[length=2.5mm]}, red!70!black, ultra thick]
        (4,0) -- (3,0);
    \node[red!70!black, above] at (3.5,0.1) {$\vec{a}_T$};

    % Radio de la Tierra
    \draw[{Latex[length=2mm]}-,green!60!black, thick] (0,0) -- (4,0);
    \node[green!60!black, below] at (2,-0.1) {$R_T$};

    % Radio de Mercurio
    \draw[{Latex[length=2mm]}-,blue!50, thick] (0,0) -- (0,2);
    \node[blue!50, left] at (-0.1,1) {$R_M$};

    % Flecha de órbita
    \draw[-{Latex[length=3mm]}, red!70!black, very thick]
        (3.5,-2.6) arc (-37:10:4);

    \node[below] at (0,-4.2) {\textbf{Sistema Solar (no a escala)}};

\end{tikzpicture}
%\end{center}
\vspace{-\baselineskip} % Reduce espacio después
\end{wrapfigure}
El radio de la órbita terrestre alrededor del Sol (suponiendo que fuera circular) es de $1.50 \times 10^8$ km, y la Tierra la recorre en 365 días. \\[.1mm]

a) Calcule la magnitud de la velocidad orbital de la Tierra en m/s. \\[.1mm]

b) Calcule la aceleración radial de la Tierra hacia el Sol en m/s$^2$. \\[.1mm]

c) Repita los incisos a) y b) para el movimiento del planeta Mercurio (radio orbital $= 5.79 \times 10^7$ km, periodo orbital $= 88.0$ días). \\[.1mm]

\section{Datos del Problema}

\begin{datosbox}

\textbf{Tierra:}
\begin{itemize}
    \item \textbf{Radio orbital:} $R_T = 1.50 \times 10^8$ km $= 1.50 \times 10^{11}$ m
    \item \textbf{Periodo orbital:} $T_T = 365$ días $= 3.156 \times 10^7$ s
\end{itemize}

\textbf{Mercurio:}
\begin{itemize}
    \item \textbf{Radio orbital:} $R_M = 5.79 \times 10^7$ km $= 5.79 \times 10^{10}$ m
    \item \textbf{Periodo orbital:} $T_M = 88.0$ días $= 7.603 \times 10^6$ s
\end{itemize}

\end{datosbox}

\section{Marco Teórico}

\subsection{Velocidad Orbital}

Para un objeto en órbita circular de radio $R$ con periodo $T$:

\begin{equation*}
    v = \frac{2\pi R}{T}
\end{equation*}

\subsection{Aceleración Centrípeta}

La aceleración centrípeta dirigida hacia el centro (en este caso, hacia el Sol) es:

\begin{equation*}
    a_{\text{rad}} = \frac{v^2}{R} = \frac{4\pi^2 R}{T^2}
\end{equation*}

\subsection{Conversión de Unidades}

\textbf{Días a segundos:}
\begin{equation*}
    1 \text{ día} = 24 \text{ h} \times 3600 \text{ s/h} = 86\,400 \text{ s}
\end{equation*}

\section{Desarrollo de la Solución}

\begin{solucionbox}

\subsection*{Conversiones previas}

\textbf{Tierra:}
\begin{align*}
    T_T &= 365 \times 86\,400 = 31\,536\,000 \text{ s} = 3.1536 \times 10^7 \text{ s}
\end{align*}

\textbf{Mercurio:}
\begin{align*}
    T_M &= 88.0 \times 86\,400 = 7\,603\,200 \text{ s} = 7.603 \times 10^6 \text{ s}
\end{align*}

\subsection*{Parte a) Velocidad orbital de la Tierra}

\subsubsection*{Paso 1: Aplicar la fórmula}

\begin{equation*}
    v_T = \frac{2\pi R_T}{T_T}
\end{equation*}

\subsubsection*{Paso 2: Sustituir valores}

\begin{align*}
    v_T &= \frac{2\pi \times 1.50 \times 10^{11} \text{ m}}{3.1536 \times 10^7 \text{ s}} \\
    &= \frac{9.425 \times 10^{11} \text{ m}}{3.1536 \times 10^7 \text{ s}} \\
    &= 2.989 \times 10^4 \text{ m/s} \\
    &\approx 29\,900 \text{ m/s} \\
    &\approx 30\,000 \text{ m/s} = 30 \text{ km/s}
\end{align*}

\textbf{En km/h:}
\begin{equation*}
    v_T = 30\,000 \text{ m/s} \times 3.6 = 108\,000 \text{ km/h}
\end{equation*}

\subsection*{Parte b) Aceleración radial de la Tierra}

\subsubsection*{Método 1: Usando $v$ y $R$}

\begin{align*}
    a_T &= \frac{v_T^2}{R_T} \\
    &= \frac{(2.99 \times 10^4)^2}{1.50 \times 10^{11}} \\
    &= \frac{8.94 \times 10^8}{1.50 \times 10^{11}} \\
    &= 5.96 \times 10^{-3} \text{ m/s}^2 \\
    &\approx 0.006 \text{ m/s}^2
\end{align*}

\subsubsection*{Método 2: Usando la fórmula directa (verificación)}

\begin{align*}
    a_T &= \frac{4\pi^2 R_T}{T_T^2} \\
    &= \frac{4\pi^2 \times 1.50 \times 10^{11}}{(3.1536 \times 10^7)^2} \\
    &= \frac{5.921 \times 10^{12}}{9.945 \times 10^{14}} \\
    &= 5.95 \times 10^{-3} \text{ m/s}^2 \quad \checkmark
\end{align*}

\subsection*{Parte c) Mercurio}

\subsubsection*{Velocidad orbital de Mercurio}

\begin{align*}
    v_M &= \frac{2\pi R_M}{T_M} \\
    &= \frac{2\pi \times 5.79 \times 10^{10}}{7.603 \times 10^6} \\
    &= \frac{3.638 \times 10^{11}}{7.603 \times 10^6} \\
    &= 4.78 \times 10^4 \text{ m/s} \\
    &\approx 47\,800 \text{ m/s} = 47.8 \text{ km/s}
\end{align*}

\textbf{En km/h:}
\begin{equation*}
    v_M = 47\,800 \times 3.6 = 172\,000 \text{ km/h}
\end{equation*}

\subsubsection*{Aceleración radial de Mercurio}

\begin{align*}
    a_M &= \frac{v_M^2}{R_M} \\
    &= \frac{(4.78 \times 10^4)^2}{5.79 \times 10^{10}} \\
    &= \frac{2.285 \times 10^9}{5.79 \times 10^{10}} \\
    &= 3.95 \times 10^{-2} \text{ m/s}^2 \\
    &\approx 0.040 \text{ m/s}^2
\end{align*}

\textbf{Verificación:}
\begin{align*}
    a_M &= \frac{4\pi^2 \times 5.79 \times 10^{10}}{(7.603 \times 10^6)^2} \\
    &= 3.95 \times 10^{-2} \text{ m/s}^2 \quad \checkmark
\end{align*}

\end{solucionbox}

\section{Resultados Finales}

\begin{resultadobox}

\subsection*{TIERRA}

\textbf{a) Velocidad orbital:}
\begin{equation*}
    \boxed{v_T = 30\,000 \text{ m/s} = 30 \text{ km/s} = 108\,000 \text{ km/h}}
\end{equation*}

\textbf{b) Aceleración radial:}
\begin{equation*}
    \boxed{a_T = 0.006 \text{ m/s}^2 = 6.0 \times 10^{-3} \text{ m/s}^2}
\end{equation*}

\vspace{0.5cm}

\subsection*{MERCURIO}

\textbf{c.a) Velocidad orbital:}
\begin{equation*}
    \boxed{v_M = 47\,800 \text{ m/s} = 47.8 \text{ km/s} = 172\,000 \text{ km/h}}
\end{equation*}

\textbf{c.b) Aceleración radial:}
\begin{equation*}
    \boxed{a_M = 0.040 \text{ m/s}^2 = 4.0 \times 10^{-2} \text{ m/s}^2}
\end{equation*}

\end{resultadobox}

\section{Análisis Comparativo}

\subsection{Comparación entre Tierra y Mercurio}

\begin{center}
\begin{tabular}{|l|c|c|c|}
\hline
\textbf{Parámetro} & \textbf{Tierra} & \textbf{Mercurio} & \textbf{Razón M/T} \\
\hline
Radio orbital (km) & $1.50 \times 10^8$ & $5.79 \times 10^7$ & 0.39 \\
\hline
Periodo (días) & 365 & 88.0 & 0.24 \\
\hline
Velocidad (km/s) & 30.0 & 47.8 & 1.59 \\
\hline
Aceleración (m/s$^2$) & 0.006 & 0.040 & 6.67 \\
\hline
\end{tabular}
\end{center}

\textbf{Observaciones:}
\begin{itemize}
    \item Mercurio está más cerca del Sol ($\sim$ 39\% de la distancia de la Tierra)
    \item Mercurio se mueve más rápido ($\sim$ 1.6 veces la velocidad de la Tierra)
    \item Mercurio completa su órbita en solo 88 días
    \item La aceleración de Mercurio es $\sim$ 6.7 veces mayor que la de la Tierra
\end{itemize}

\subsection{Gráfica comparativa}

\begin{center}
\begin{tikzpicture}
\begin{axis}[
    ybar,
    width=0.9\textwidth,
    height=7cm,
    ylabel={Velocidad orbital (km/s)},
    symbolic x coords={Mercurio, Venus, Tierra, Marte, Júpiter},
    xtick=data,
    ymin=0, ymax=50,
    bar width=0.8cm,
    title={Velocidades orbitales planetarias (planetas interiores)},
    grid=major
]

\addplot coordinates {
    (Mercurio,47.8)
    (Venus,35.0)
    (Tierra,30.0)
    (Marte,24.1)
    (Júpiter,13.1)
};

\end{axis}
\end{tikzpicture}
\end{center}

\textbf{Patrón:} Los planetas más cercanos al Sol se mueven más rápido.

\section{Conexión con las Leyes de Kepler}

\subsection{Tercera Ley de Kepler}

La tercera ley de Kepler establece que:

\begin{equation*}
    T^2 \propto R^3
\end{equation*}

Verifiquemos con nuestros datos:

\textbf{Tierra:}
\begin{equation*}
    \frac{T^2}{R^3} = \frac{(3.154 \times 10^7)^2}{(1.50 \times 10^{11})^3} = 2.95 \times 10^{-19}
\end{equation*}

\textbf{Mercurio:}
\begin{equation*}
    \frac{T^2}{R^3} = \frac{(7.603 \times 10^6)^2}{(5.79 \times 10^{10})^3} = 2.98 \times 10^{-19}
\end{equation*}

¡Los valores son casi idénticos! Esto confirma la tercera ley de Kepler.

\subsection{Relación entre velocidad y radio}

De $v = 2\pi R/T$ y la tercera ley de Kepler, podemos deducir:

\begin{equation*}
    v \propto \frac{1}{\sqrt{R}}
\end{equation*}

Es decir, la velocidad es inversamente proporcional a la raíz cuadrada del radio.

\textbf{Verificación:}
\begin{equation*}
    \frac{v_M}{v_T} = \sqrt{\frac{R_T}{R_M}} = \sqrt{\frac{1.50 \times 10^{11}}{5.79 \times 10^{10}}} = \sqrt{2.59} = 1.61
\end{equation*}

Y de nuestros cálculos: $\frac{47.8}{30.0} = 1.59$ ¡Coincide!

\section{Perspectiva}

\subsection{Velocidades orbitales en contexto}

\begin{itemize}
    \item \textbf{Avión comercial:} $\sim$ 900 km/h $\approx$ 0.25 km/s
    \item \textbf{Bala de rifle:} $\sim$ 1 km/s
    \item \textbf{Velocidad de escape terrestre:} 11.2 km/s
    \item \textbf{Tierra alrededor del Sol:} 30 km/s
    \item \textbf{Mercurio alrededor del Sol:} 47.8 km/s
    \item \textbf{Sol alrededor de la galaxia:} $\sim$ 220 km/s
\end{itemize}

\subsection{Aceleraciones en contexto}

Aunque las aceleraciones centrípetas son muy pequeñas (milésimas de m/s$^2$), son suficientes para mantener a los planetas en órbita debido a:
\begin{enumerate}
    \item La gran masa del Sol proporciona la fuerza gravitacional necesaria
    \item Las órbitas son muy grandes (millones de kilómetros)
    \item No hay fricción en el espacio que detenga el movimiento
\end{enumerate}

\section{Conceptos Clave}

\begin{enumerate}
    \item Los planetas más cercanos al Sol tienen velocidades orbitales mayores

    \item La aceleración centrípeta aumenta dramáticamente al acercarse al Sol (proporcional a $1/R^2$)

    \item Las velocidades orbitales planetarias son del orden de decenas de km/s

    \item La tercera ley de Kepler relaciona el periodo con el radio orbital: $T^2 \propto R^3$

    \item Aunque las aceleraciones parecen pequeñas, son suficientes para mantener órbitas estables durante miles de millones de años
\end{enumerate}

\end{document}
