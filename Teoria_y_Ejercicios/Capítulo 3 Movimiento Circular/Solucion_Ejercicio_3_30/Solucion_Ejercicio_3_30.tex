\documentclass[11pt,a4paper]{article}

% Paquetes necesarios
\usepackage[utf8]{inputenc}
\usepackage[T1]{fontenc}
\usepackage[spanish]{babel}
\usepackage[margin=2.5cm]{geometry}
\usepackage{amsmath}
\usepackage{amssymb}
\usepackage{xcolor}
\usepackage{tcolorbox}
\usepackage{graphicx}
\usepackage{tikz}
\usepackage{pgfplots}
\pgfplotsset{compat=1.18}
\usetikzlibrary{arrows.meta,patterns,decorations.pathmorphing,calc}
\usepackage{wrapfig}
\usepackage[export]{adjustbox} % (opcional) claves extra para \includegraphics
\usepackage{xparse}

% Definición de colores
\definecolor{azuloscuro}{RGB}{0,51,102}
\definecolor{azulclaro}{RGB}{230,240,250}
\definecolor{verdeoscuro}{RGB}{0,100,0}
\definecolor{rojoclaro}{RGB}{255,230,230}

% Configuración de cajas
\tcbuselibrary{theorems,skins,breakable}

\newtcolorbox{datosbox}{
    colback=azulclaro,
    colframe=azuloscuro,
    fonttitle=\bfseries,
    title=Datos del Problema,
    sharp corners,
    boxrule=1pt
}

\newtcolorbox{solucionbox}{
    colback=white,
    colframe=verdeoscuro,
    fonttitle=\bfseries,
    title=Desarrollo de la Solución,
    sharp corners,
    boxrule=1pt,
    breakable
}

\newtcolorbox{resultadobox}{
    colback=rojoclaro,
    colframe=red!70!black,
    fonttitle=\bfseries,
    title=Resultado Final,
    sharp corners,
    boxrule=2pt
}

% Título y autor
\title{\textbf{Solución del Ejercicio 3.30} \\
\large Movimiento Circular: Rotor de Helicóptero}
\author{Física Universitaria - Sears y Zemansky \\ Capítulo 3: Movimiento en Dos o Tres Dimensiones}
\date{\today}

\begin{document}

\maketitle

\section{Enunciado del Problema}

\begin{wrapfigure}[14]{r}{.6\textwidth}  % Ajusta el ancho según necesites
	\centering
	\vspace{-\baselineskip}
	%\begin{center}
\begin{tikzpicture}[scale=0.85]
    % Eje central del rotor
    \filldraw[gray!50] (0,0) circle (0.3);

    % Cuatro aspas del rotor
    % Aspa 1 (horizontal derecha)
    \fill[blue!40, opacity=0.7] (0.3,0.15) -- (3.4,0.15) -- (3.4,-0.15) -- (0.3,-0.15) -- cycle;
    \draw[blue!70!black, very thick] (0.3,0.15) -- (3.4,0.15) -- (3.4,-0.15) -- (0.3,-0.15) -- cycle;

    % Aspa 2 (horizontal izquierda)
    \fill[blue!40, opacity=0.7] (-0.3,0.15) -- (-3.4,0.15) -- (-3.4,-0.15) -- (-0.3,-0.15) -- cycle;
    \draw[blue!70!black, very thick] (-0.3,0.15) -- (-3.4,0.15) -- (-3.4,-0.15) -- (-0.3,-0.15) -- cycle;

    % Aspa 3 (vertical arriba)
    \fill[blue!40, opacity=0.7] (0.15,0.3) -- (0.15,3.4) -- (-0.15,3.4) -- (-0.15,0.3) -- cycle;
    \draw[blue!70!black, very thick] (0.15,0.3) -- (0.15,3.4) -- (-0.15,3.4) -- (-0.15,0.3) -- cycle;

    % Aspa 4 (vertical abajo)
    \fill[blue!40, opacity=0.7] (0.15,-0.3) -- (0.15,-3.4) -- (-0.15,-3.4) -- (-0.15,-0.3) -- cycle;
    \draw[blue!70!black, very thick] (0.15,-0.3) -- (0.15,-3.4) -- (-0.15,-3.4) -- (-0.15,-0.3) -- cycle;

    % Flecha de rotación
    \draw[-{Latex[length=4mm]}, red!70!black, ultra thick]
        (0.5,3.6) arc (45:145:1);
    \node[red!70!black] at (1.5,3.2) {$\omega = 550$ rpm};

    % Radio marcado
    \draw[{Latex[length=2mm]}-{Latex[length=2mm]}, thick, orange!80!black]
        (0,0) -- (3.4,0);
    \node[orange!80!black, above] at (1.7,0) {$R = 3.40$ m};

    % Punta del aspa destacada
    \filldraw[green!60!black] (3.4,0) circle (0.12);
    \node[green!60!black, right] at (3.5,0) {Punta del aspa};

    % Vector de velocidad en la punta
    \draw[-{Latex[length=3mm]}, purple!70!black, ultra thick]
        (3.4,0) -- (3.4,1.3);
    \node[purple!70!black, right] at (3.4,0.7) {$\vec{v}$};

    % Vector de aceleración centrípeta
    \draw[-{Latex[length=3mm]}, red!70!black, ultra thick]
        (3.4,0) -- (2.2,0);
    \node[red!70!black, below] at (2.8,-0.1) {$\vec{a}_{\text{rad}}$};

    % Etiqueta
    \node[below, align=center] at (0,-4.2) {\textbf{Vista superior del rotor} \\ (4 aspas)};

\end{tikzpicture}
%\end{center}
\vspace{-\baselineskip} % Reduce espacio después
\end{wrapfigure}
Un modelo de rotor de helicóptero tiene cuatro aspas, cada una de 3.40 m de longitud desde el eje central hasta la punta. El modelo se gira en un túnel de viento a 550 rpm. \\[.5mm]

a) ?`Qué rapidez lineal tiene la punta del aspa en m/s? \\[.5mm]

b) ?`Qué aceleración radial tiene la punta del aspa, expresada como un múltiplo de la aceleración debida a la gravedad, es decir, $g$? \\[.5mm]

\section{Datos del Problema}

\begin{datosbox}
\begin{itemize}
    \item \textbf{Número de aspas:} 4
    \item \textbf{Longitud de cada aspa:} $L = 3.40$ m
    \item \textbf{Radio de rotación:} $R = 3.40$ m (desde el eje hasta la punta)
    \item \textbf{Frecuencia de rotación:} $f = 550$ rpm (revoluciones por minuto)
    \item \textbf{Aceleración gravitacional:} $g = 9.8$ m/s$^2$
\end{itemize}
\end{datosbox}

\section{Marco Teórico}

\subsection{Rapidez Lineal en Movimiento Circular}

La rapidez lineal $v$ de un punto que gira a una distancia $R$ del eje de rotación con frecuencia $f$ (en revoluciones por unidad de tiempo) es:

\begin{equation*}
    v = 2\pi R f
\end{equation*}

donde $f$ debe estar en revoluciones por segundo (Hz o rps).

\subsection{Aceleración Centrípeta}

La aceleración centrípeta puede calcularse de dos formas equivalentes:

\begin{equation*}
    a_{\text{rad}} = \frac{v^2}{R} \quad \text{o} \quad a_{\text{rad}} = 4\pi^2 R f^2
\end{equation*}

\subsection{Conversión de Unidades}

Para convertir rpm (revoluciones por minuto) a rps (revoluciones por segundo):

\begin{equation*}
    f_{\text{(rps)}} = \frac{f_{\text{(rpm)}}}{60}
\end{equation*}

\section{Desarrollo de la Solución}

\begin{solucionbox}

\subsection*{Conversión de unidades}

Primero convertimos la frecuencia de rpm a rps:

\begin{align*}
    f &= 550 \text{ rpm} = \frac{550 \text{ rev}}{60 \text{ s}} \\
    &= 9.167 \text{ rps} \\
    &\approx 9.17 \text{ Hz}
\end{align*}

\subsection*{Parte a) Rapidez lineal de la punta}

\subsubsection*{Paso 1: Aplicar la fórmula}

La rapidez lineal en la punta del aspa es:

\begin{equation*}
    v = 2\pi R f
\end{equation*}

\subsubsection*{Paso 2: Sustituir valores}

\begin{align*}
    v &= 2\pi \times 3.40 \text{ m} \times 9.167 \text{ rps} \\
    &= 2\pi \times 31.17 \text{ m/s} \\
    &= 195.8 \text{ m/s} \\
    &\approx 196 \text{ m/s}
\end{align*}

\textbf{Conversión a km/h:}

\begin{equation*}
    v = 196 \text{ m/s} \times \frac{3.6 \text{ km/h}}{1 \text{ m/s}} = 706 \text{ km/h}
\end{equation*}

\textbf{Interpretación:} La punta del aspa se mueve a aproximadamente 196 m/s o 706 km/h. ¡Esta es una velocidad muy alta, cercana a la velocidad del sonido (343 m/s)!

\subsection*{Parte b) Aceleración radial como múltiplo de $g$}

\subsubsection*{Método 1: Usando $v$ y $R$}

\begin{align*}
    a_{\text{rad}} &= \frac{v^2}{R} \\
    &= \frac{(196 \text{ m/s})^2}{3.40 \text{ m}} \\
    &= \frac{38\,416 \text{ m}^2\text{/s}^2}{3.40 \text{ m}} \\
    &= 11\,299 \text{ m/s}^2 \\
    &\approx 11\,300 \text{ m/s}^2
\end{align*}

\subsubsection*{Paso 2: Expresar como múltiplo de $g$}

\begin{align*}
    \frac{a_{\text{rad}}}{g} &= \frac{11\,300 \text{ m/s}^2}{9.8 \text{ m/s}^2} \\
    &= 1153 \\
    &\approx 1150
\end{align*}

Por lo tanto:
\begin{equation*}
    a_{\text{rad}} \approx 1150\,g
\end{equation*}

\subsubsection*{Método 2: Usando la fórmula directa (verificación)}

También podemos usar:

\begin{align*}
    a_{\text{rad}} &= 4\pi^2 R f^2 \\
    &= 4\pi^2 \times 3.40 \text{ m} \times (9.167 \text{ rps})^2 \\
    &= 39.48 \times 3.40 \times 84.03 \\
    &= 11\,281 \text{ m/s}^2 \\
    &\approx 11\,300 \text{ m/s}^2 \quad \checkmark
\end{align*}

Los resultados coinciden, confirmando nuestro cálculo.

\textbf{Interpretación:} La aceleración en la punta del aspa es aproximadamente \textbf{1150 veces} la aceleración gravitacional. ¡Esta es una aceleración extremadamente grande!

\end{solucionbox}

\section{Resultados Finales}

\begin{resultadobox}

\subsection*{Parte a) Rapidez lineal}

\begin{equation*}
    \boxed{v = 196 \text{ m/s} = 706 \text{ km/h}}
\end{equation*}

\subsection*{Parte b) Aceleración radial}

\textbf{En m/s$^2$:}
\begin{equation*}
    \boxed{a_{\text{rad}} = 11\,300 \text{ m/s}^2}
\end{equation*}

\textbf{Como múltiplo de $g$:}
\begin{equation*}
    \boxed{a_{\text{rad}} = 1150\,g}
\end{equation*}

Esto significa que la aceleración en la punta del aspa es \textbf{mil ciento cincuenta veces} mayor que la aceleración gravitacional en la superficie terrestre.

\end{resultadobox}

\section{Análisis Adicional}

\subsection{Comparación con otras situaciones}

\begin{center}
\begin{tabular}{|l|c|}
\hline
\textbf{Situación} & \textbf{Aceleración (múltiplos de $g$)} \\
\hline
Persona caminando & $< 0.1\,g$ \\
\hline
Automóvil deportivo acelerando & $\sim 1.0\,g$ \\
\hline
Montaña rusa & $3 - 6\,g$ \\
\hline
Piloto de avión caza en maniobra & $\sim 9\,g$ \\
\hline
Centrifugadora de laboratorio & $100 - 1000\,g$ \\
\hline
\textbf{Punta de aspa de helicóptero} & $\mathbf{1150\,g}$ \\
\hline
Ultracentrífuga & $> 100\,000\,g$ \\
\hline
\end{tabular}
\end{center}

\subsection{Variación de velocidad y aceleración a lo largo del aspa}

La velocidad y aceleración no son constantes a lo largo del aspa, sino que dependen de la distancia al eje:

\begin{center}
\begin{tikzpicture}
\begin{axis}[
    width=0.85\textwidth,
    height=7cm,
    xlabel={Distancia al eje $r$ (m)},
    ylabel style={align=center},
    ylabel={Velocidad $v$ (m/s) \\ Aceleración $a_{\text{rad}}/100$ (m/s$^2$)},
    xmin=0, xmax=3.5,
    ymin=0, ymax=200,
    grid=major,
    legend pos=north west
]

% Velocidad lineal v = 2πrf
\addplot[blue, thick, domain=0:3.4] {2*pi*x*9.167};

% Aceleración centrípeta / 100 para escalar
\addplot[red, thick, domain=0:3.4] {4*pi^2*x*9.167^2/100};

\legend{$v = 2\pi r f$, $a_{\text{rad}}/100$}

\end{axis}
\end{tikzpicture}
\end{center}

\textbf{Observaciones:}
\begin{itemize}
    \item La velocidad aumenta linealmente con la distancia al eje
    \item La aceleración también aumenta linealmente con la distancia
    \item En el eje central ($r = 0$): $v = 0$ y $a_{\text{rad}} = 0$
    \item En la punta ($r = 3.40$ m): valores máximos
\end{itemize}

\subsection{Esfuerzos en el material del aspa}

Con aceleraciones tan grandes, el material del aspa experimenta fuerzas enormes:

\textbf{Ejemplo:} Si cada aspa tiene una masa de 10 kg concentrada en su punto medio (1.70 m del eje):

\begin{align*}
    a_{\text{medio}} &= 4\pi^2 \times 1.70 \times (9.167)^2 = 5650 \text{ m/s}^2 = 577\,g \\
    F &= m \times a = 10 \text{ kg} \times 5650 \text{ m/s}^2 = 56\,500 \text{ N}
\end{align*}

¡Una fuerza de más de 56,000 Newtons (equivalente al peso de 5.8 toneladas)!

\subsection{Relación con la velocidad del sonido}

La velocidad del sonido en el aire a nivel del mar es aproximadamente 343 m/s. La punta del aspa se mueve a:

\begin{equation*}
    \frac{v_{\text{aspa}}}{v_{\text{sonido}}} = \frac{196}{343} = 0.57
\end{equation*}

El aspa se mueve a \textbf{57\% de la velocidad del sonido} (Mach 0.57). A velocidades cercanas o superiores a Mach 1, se generan ondas de choque y aumenta dramáticamente la resistencia del aire.

\section{Conceptos Clave}

\begin{enumerate}
    \item La rapidez lineal en movimiento circular aumenta proporcionalmente con la distancia al eje: $v = 2\pi r f$

    \item La aceleración centrípeta también aumenta con la distancia al eje: $a_{\text{rad}} = 4\pi^2 r f^2$

    \item A 550 rpm, las puntas de las aspas experimentan aceleraciones superiores a 1000$g$

    \item Estos valores explican por qué las aspas de helicóptero deben fabricarse con materiales muy resistentes (aleaciones especiales, fibra de carbono)

    \item La velocidad de las puntas se acerca a la velocidad del sonido, lo cual limita la velocidad máxima de rotación práctica

    \item Conversión importante: $1 \text{ rpm} = \frac{1}{60}$ Hz $= 0.01667$ Hz
\end{enumerate}

\section{Aplicaciones Prácticas}

\subsection{Límites de diseño}

Los diseñadores de helicópteros deben considerar:

\begin{enumerate}
    \item \textbf{Resistencia estructural:} El material debe soportar fuerzas centrífugas enormes

    \item \textbf{Efectos aerodinámicos:} A velocidades cercanas a la del sonido, aumenta la resistencia del aire

    \item \textbf{Vibración:} Las aspas deben estar perfectamente balanceadas

    \item \textbf{Fatiga del material:} Cada revolución somete las aspas a ciclos de esfuerzo
\end{enumerate}

\subsection{Helicópteros reales}

En helicópteros de tamaño real:
\begin{itemize}
    \item Radio del rotor: 5-10 m
    \item Velocidad de rotación: 200-400 rpm
    \item Velocidad de las puntas: limitada a Mach 0.9 (~ 300 m/s) para evitar ondas de choque
\end{itemize}

\end{document}
