\documentclass[11pt,a4paper]{article}

% Paquetes necesarios
\usepackage[utf8]{inputenc}
\usepackage[T1]{fontenc}
\usepackage[spanish]{babel}
\usepackage[margin=2.5cm]{geometry}
\usepackage{amsmath}
\usepackage{amssymb}
\usepackage{xcolor}
\usepackage{tcolorbox}
\usepackage{graphicx}
\usepackage{tikz}
\usepackage{pgfplots}
\pgfplotsset{compat=1.18}
\usetikzlibrary{arrows.meta,patterns,decorations.pathmorphing,calc}
\usepackage{wrapfig}
\usepackage[export]{adjustbox} % (opcional) claves extra para \includegraphics
\usepackage{xparse}

% Definición de colores
\definecolor{azuloscuro}{RGB}{0,51,102}
\definecolor{azulclaro}{RGB}{230,240,250}
\definecolor{verdeoscuro}{RGB}{0,100,0}
\definecolor{rojoclaro}{RGB}{255,230,230}

% Configuración de cajas
\tcbuselibrary{theorems,skins,breakable}

\newtcolorbox{datosbox}{
    colback=azulclaro,
    colframe=azuloscuro,
    fonttitle=\bfseries,
    title=Datos del Problema,
    sharp corners,
    boxrule=1pt
}

\newtcolorbox{solucionbox}{
    colback=white,
    colframe=verdeoscuro,
    fonttitle=\bfseries,
    title=Desarrollo de la Solución,
    sharp corners,
    boxrule=1pt,
    breakable
}

\newtcolorbox{resultadobox}{
    colback=rojoclaro,
    colframe=red!70!black,
    fonttitle=\bfseries,
    title=Resultado Final,
    sharp corners,
    boxrule=2pt
}

% Título y autor
\title{\textbf{Solución del Ejercicio 3.33} \\
\large Movimiento Circular: Rueda de la Fortuna}
\author{Física Universitaria - Sears y Zemansky \\ Capítulo 3: Movimiento en Dos o Tres Dimensiones}
\date{\today}

\begin{document}

\maketitle

\section{Enunciado del Problema}

\begin{wrapfigure}[15]{r}{.55\textwidth}  % Ajusta el ancho según necesites
	\centering
	\vspace{-\baselineskip}
	%\begin{center}
\begin{tikzpicture}[scale=0.75]
    % Estructura de la rueda
    \draw[blue!50!black, very thick] (0,0) circle (4);

    % Radios de la rueda
    \foreach \angle in {0,45,...,315} {
        \draw[blue!30, thick] (0,0) -- (\angle:4);
    }

    % Eje central
    \filldraw[gray!60] (0,0) circle (0.2);

    % Pasajero en punto más bajo
    \filldraw[red!70!black] (0,-4) circle (0.2);
    \node[red!70!black, below] at (0,-4.5) {Punto más bajo};

    % Vector velocidad (punto más bajo)
    \draw[-{Latex[length=3mm]}, green!60!black, ultra thick]
        (0,-4) -- (-1.2,-4);
    \node[green!60!black, below] at (-0.85,-4) {$\vec{v}$};

    % Vector aceleración centrípeta (punto más bajo)
    \draw[-{Latex[length=3mm]}, purple!70!black, ultra thick]
        (0,-4) -- (0,-3);
    \node[purple!70!black, right] at (0.1,-3.5) {$\vec{a}_{\text{rad}}$};

    % Pasajero en punto más alto
    \filldraw[orange!80!black] (0,4) circle (0.2);
    \node[orange!80!black, above] at (0,4.5) {Punto más alto};

    % Vector velocidad (punto más alto)
    \draw[-{Latex[length=3mm]}, green!60!black, ultra thick]
        (0,4) -- (1.2,4);
    \node[green!60!black, above] at (0.6,3.95) {$\vec{v}$};

    % Vector aceleración centrípeta (punto más alto)
    \draw[-{Latex[length=3mm]}, purple!70!black, ultra thick]
        (0,4) -- (0,3);
    \node[purple!70!black, left] at (-0.1,3.5) {$\vec{a}_{\text{rad}}$};

    % Radio marcado
    \draw[{Latex[length=2mm]}-{Latex[length=2mm]}, thick]
        (0.3,0) -- (0.3,-4);
    \node[right] at (0.4,-2) {$R = 14.0$ m};

    % Flecha de rotación
    \draw[-{Latex[length=4mm]}, red!70!black, very thick]
        (-3,3) arc (135:90:4.2);

    \node[below, align=center] at (0,-5) {\textbf{Rueda de la fortuna} \\ (movimiento circular uniforme)};

\end{tikzpicture}
%\end{center}
\vspace{-\baselineskip} % Reduce espacio después
\end{wrapfigure}
Una rueda de la fortuna de 14.0 m de radio gira sobre un eje horizontal en el centro (figura 3.42). La rapidez lineal de un pasajero en el borde es constante e igual a 7.00 m/s. \\[.1mm]

?`Qué magnitud y dirección tiene la aceleración del pasajero al pasar: \\[.1mm]

a) por el punto más bajo de su movimiento circular? \\[.1mm]

b) ?`Por el punto más alto de su movimiento circular? \\[.1mm]

c) ?`Cuánto tarda una revolución de la rueda? \\[.1mm]

\section{Datos del Problema}

\begin{datosbox}
\begin{itemize}
    \item \textbf{Radio de la rueda:} $R = 14.0$ m
    \item \textbf{Rapidez lineal (constante):} $v = 7.00$ m/s
    \item \textbf{Aceleración gravitacional:} $g = 9.8$ m/s$^2$
\end{itemize}
\end{datosbox}

\section{Marco Teórico}

\subsection{Movimiento Circular Uniforme}

En movimiento circular uniforme:
\begin{itemize}
    \item La rapidez $v$ es constante
    \item La aceleración centrípeta tiene magnitud constante: $a_{\text{rad}} = v^2/R$
    \item La aceleración siempre apunta hacia el centro del círculo
    \item No hay aceleración tangencial ($a_{\text{tan}} = 0$)
\end{itemize}

\subsection{Dirección de la Aceleración}

\begin{itemize}
    \item \textbf{Punto más bajo:} La aceleración apunta hacia arriba (hacia el centro)
    \item \textbf{Punto más alto:} La aceleración apunta hacia abajo (hacia el centro)
    \item \textbf{Puntos laterales:} La aceleración apunta horizontalmente hacia el centro
\end{itemize}

\section{Desarrollo de la Solución}

\begin{solucionbox}

\subsection*{Cálculo de la aceleración centrípeta}

La magnitud de la aceleración centrípeta es la misma en todos los puntos:

\begin{align*}
    a_{\text{rad}} &= \frac{v^2}{R} \\
    &= \frac{(7.00 \text{ m/s})^2}{14.0 \text{ m}} \\
    &= \frac{49.0 \text{ m}^2\text{/s}^2}{14.0 \text{ m}} \\
    &= 3.50 \text{ m/s}^2
\end{align*}

\subsection*{Parte a) Aceleración en el punto más bajo}

\textbf{Magnitud:}
\begin{equation*}
    |{\vec{a}}| = 3.50 \text{ m/s}^2
\end{equation*}

\textbf{Dirección:}

La aceleración apunta hacia el centro de la rueda, es decir, \textbf{verticalmente hacia arriba}.

\textbf{Expresión vectorial:}

Si tomamos el eje $y$ positivo hacia arriba:
\begin{equation*}
    \vec{a} = 3.50 \text{ m/s}^2 \, \hat{j} \quad \text{(hacia arriba)}
\end{equation*}

\subsection*{Parte b) Aceleración en el punto más alto}

\textbf{Magnitud:}
\begin{equation*}
    |\vec{a}| = 3.50 \text{ m/s}^2
\end{equation*}

\textbf{Dirección:}

La aceleración apunta hacia el centro de la rueda, es decir, \textbf{verticalmente hacia abajo}.

\textbf{Expresión vectorial:}

\begin{equation*}
    \vec{a} = -3.50 \text{ m/s}^2 \, \hat{j} \quad \text{(hacia abajo)}
\end{equation*}

\subsection*{Parte c) Tiempo de una revolución}

El periodo $T$ es el tiempo para completar una vuelta:

\begin{align*}
    T &= \frac{2\pi R}{v} \\
    &= \frac{2\pi \times 14.0 \text{ m}}{7.00 \text{ m/s}} \\
    &= \frac{87.96 \text{ m}}{7.00 \text{ m/s}} \\
    &= 12.6 \text{ s}
\end{align*}

\textbf{Verificación usando la fórmula alternativa:}

También podemos usar:
\begin{align*}
    a_{\text{rad}} &= \frac{4\pi^2 R}{T^2} \\
    T &= 2\pi\sqrt{\frac{R}{a_{\text{rad}}}} \\
    &= 2\pi\sqrt{\frac{14.0}{3.50}} \\
    &= 2\pi\sqrt{4.0} \\
    &= 2\pi \times 2.0 \\
    &= 12.6 \text{ s} \quad \checkmark
\end{align*}

\end{solucionbox}

\section{Resultados Finales}

\begin{resultadobox}

\subsection*{Parte a) Punto más bajo}

\textbf{Magnitud:}
\begin{equation*}
    \boxed{|\vec{a}| = 3.50 \text{ m/s}^2}
\end{equation*}

\textbf{Dirección:}
\begin{equation*}
    \boxed{\text{Hacia arriba (hacia el centro)}}
\end{equation*}

\subsection*{Parte b) Punto más alto}

\textbf{Magnitud:}
\begin{equation*}
    \boxed{|\vec{a}| = 3.50 \text{ m/s}^2}
\end{equation*}

\textbf{Dirección:}
\begin{equation*}
    \boxed{\text{Hacia abajo (hacia el centro)}}
\end{equation*}

\subsection*{Parte c) Periodo}

\begin{equation*}
    \boxed{T = 12.6 \text{ s}}
\end{equation*}

La rueda completa una vuelta cada \textbf{12.6 segundos}.

\end{resultadobox}

\section{Análisis Adicional}

\subsection{Comparación con la gravedad}

La aceleración centrípeta en la rueda es:
\begin{equation*}
    \frac{a_{\text{rad}}}{g} = \frac{3.50}{9.8} = 0.36 = 36\%
\end{equation*}

La aceleración es aproximadamente el 36\% de $g$.

\subsection{Sensación del pasajero}

\textbf{En el punto más bajo:}
\begin{itemize}
    \item La aceleración centrípeta apunta hacia arriba
    \item El pasajero siente presión adicional en el asiento
    \item Peso aparente: $W_{\text{aparente}} = W + ma_{\text{rad}} \approx 1.36W$
\end{itemize}

\textbf{En el punto más alto:}
\begin{itemize}
    \item La aceleración centrípeta apunta hacia abajo
    \item El pasajero siente menos presión en el asiento
    \item Peso aparente: $W_{\text{aparente}} = W - ma_{\text{rad}} \approx 0.64W$
\end{itemize}

\subsection{Diagrama de vectores en diferentes puntos}

\begin{wrapfigure}[16]{r}{.6\textwidth}  % Ajusta el ancho según necesites
	\centering
	\vspace{-\baselineskip}
	%\begin{center}
\begin{tikzpicture}[scale=1]
    % Círculo de la rueda
    \draw[blue!30, thick] (0,0) circle (3);
    \filldraw[gray!60] (0,0) circle (0.15);

    % Pasajero y vectores en 8 posiciones
    \foreach \angle/\label in {0/0°, 45/45°, 90/90°, 135/135°, 180/180°, 225/225°, 270/270°, 315/315°} {
        \filldraw[red!60] (\angle:3) circle (0.1);
        % Vector de aceleración (hacia el centro)
        \draw[-{Latex[length=2mm]}, purple, very thick]
            (\angle:3) -- (\angle:2);
    }

    % Etiquetas
    \node[below] at (0,-3.2) {Punto más bajo};
    \node[above] at (0,3.2) {Punto más alto};

    \node[below, align=center] at (0,-3.7) {En todos los puntos: \\ $|\vec{a}| = 3.50$ m/s$^2$ hacia el centro};

\end{tikzpicture}
%\end{center}
\vspace{-\baselineskip} % Reduce espacio después
\end{wrapfigure}

\subsection{Velocidad angular}

También podemos expresar la rotación en términos de velocidad angular:

\begin{equation*}
    \omega = \frac{v}{R} = \frac{7.00}{14.0} = 0.50 \text{ rad/s}
\end{equation*}

Y la aceleración centrípeta se puede escribir como:
\begin{equation*}
    a_{\text{rad}} = \omega^2 R = (0.50)^2 \times 14.0 = 3.50 \text{ m/s}^2
\end{equation*}

\section{Conceptos Clave}

\begin{enumerate}
    \item En movimiento circular uniforme, la magnitud de la aceleración es constante

    \item La aceleración siempre apunta hacia el centro del círculo (aceleración centrípeta)
\end{enumerate}


\begin{enumerate}
    \item[3] La dirección de la aceleración cambia continuamente conforme el objeto gira

    \item[4] En el punto más bajo, la aceleración apunta hacia arriba; en el punto más alto, apunta hacia abajo

    \item[5] El pasajero experimenta cambios en el peso aparente a lo largo del recorrido

    \item[6] La aceleración centrípeta no cambia la rapidez, solo la dirección del movimiento
\end{enumerate}

\end{document}
