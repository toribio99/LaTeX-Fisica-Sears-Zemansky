\documentclass[11pt,a4paper]{article}

% Paquetes necesarios
\usepackage[utf8]{inputenc}
\usepackage[T1]{fontenc}
\usepackage[spanish]{babel}
\usepackage[margin=2.5cm]{geometry}
\usepackage{amsmath}
\usepackage{amssymb}
\usepackage{xcolor}
\usepackage{tcolorbox}
\usepackage{graphicx}
\usepackage{tikz}
\usepackage{pgfplots}
\pgfplotsset{compat=1.18}
\usetikzlibrary{arrows.meta,patterns,decorations.pathmorphing,calc}
\usepackage{wrapfig}
\usepackage[export]{adjustbox} % (opcional) claves extra para \includegraphics
\usepackage{xparse}
%\setlength{\columnsep}{5pt}
%\setlength{\intextsep}{5pt}

% Definición de colores
\definecolor{azuloscuro}{RGB}{0,51,102}
\definecolor{azulclaro}{RGB}{230,240,250}
\definecolor{verdeoscuro}{RGB}{0,100,0}
\definecolor{rojoclaro}{RGB}{255,230,230}

% Configuración de cajas
\tcbuselibrary{theorems,skins,breakable}

\newtcolorbox{datosbox}{
    colback=azulclaro,
    colframe=azuloscuro,
    fonttitle=\bfseries,
    title=Datos del Problema,
    sharp corners,
    boxrule=1pt
}

\newtcolorbox{solucionbox}{
    colback=white,
    colframe=verdeoscuro,
    fonttitle=\bfseries,
    title=Desarrollo de la Solución,
    sharp corners,
    boxrule=1pt,
    breakable
}

\newtcolorbox{resultadobox}{
    colback=rojoclaro,
    colframe=red!70!black,
    fonttitle=\bfseries,
    title=Resultado Final,
    sharp corners,
    boxrule=2pt
}

% Título y autor
\title{\textbf{Solución del Ejercicio 3.28} \\
\large Movimiento Circular: Periodo de Rotación de una Lavadora}
\author{Física Universitaria - Sears y Zemansky \\ Capítulo 3: Movimiento en Dos o Tres Dimensiones}
\date{\today}

\begin{document}

\maketitle

\section{Enunciado del Problema}


\begingroup
\setlength{\columnsep}{0pt} % Muy cerca del texto
\setlength{\intextsep}{0pt} % Mucho espacio arriba/

\begin{wrapfigure}[8]{r}{.5\textwidth}  % Ajusta el ancho según necesites
	\centering
	\vspace{-\baselineskip}
	%\begin{center}
\begin{tikzpicture}[scale=.6]
    % Vista superior de la lavadora (tambor circular)
    % Tambor exterior
    \draw[fill=gray!30, thick] (0,0) circle (3);

    % Tambor interior (en rotación)
    \draw[fill=blue!20, very thick] (0,0) circle (2.5);

    % Perforaciones del tambor
    \foreach \angle in {0,30,...,330} {
        \foreach \r in {1.8,2.2} {
            \filldraw[gray!60] (\angle:\r) circle (0.08);
        }
    }

    % Centro de rotación
    \filldraw[black] (0,0) circle (0.15);
    \node[below] at (0,-0.3) {\small Eje};

    % Ropa dentro de la lavadora
    \filldraw[red!60] (1.5,0.5) circle (0.3);
    \filldraw[blue!60] (-1.2,0.8) circle (0.25);
    \filldraw[green!60] (0.3,-1.6) circle (0.28);
    \filldraw[yellow!60] (-1.4,-0.9) circle (0.22);

    % Flecha de rotación
    \draw[-{Latex[length=3mm]},very thick, red!70!black]
        (2,1.5) arc (37:323:2.5cm);
    \node[red!70!black, right] at (2.4,1.8) {$\omega$ (rotación)};

    % Radio
    \draw[{Latex[length=2mm]}-,thick,blue!70!black] (0,0) -- (1.5,0.5);
    \node[blue!70!black,above] at (0.75,0.25) {$R$};

    % Vector de aceleración centrípeta en un punto
    \draw[-{Latex[length=3mm]},ultra thick,orange!80!black]
        (1.5,0.5) -- (0.75,0.25);
    \node[violet!80!black,right] at (1.1,-0.2) {$\vec{a}_{\text{rad}}$};

    % Etiqueta
	\node[below, align=center] at (0,-3.3) {\textbf{Vista superior del tambor} \\ \textbf{de la lavadora}};

\end{tikzpicture}
%\end{center}
\vspace{-\baselineskip} % Reduce espacio después
\end{wrapfigure}
Imagine que, en su primer día de trabajo para un fabricante de electrodomésticos, le piden que averigüe qué hacerle al periodo de rotación de una lavadora para triplicar la aceleración centrípeta, y usted impresiona a su jefa contestando inmediatamente. ?`Qué le contesta?

\vspace{3mm}

\endgroup

\section{Datos del Problema}

\begin{datosbox}
\begin{itemize}
    \item \textbf{Objetivo:} Triplicar la aceleración centrípeta
    \item \textbf{Incógnita:} ?`Cómo modificar el periodo $T$ para lograr $a_{\text{rad,nuevo}} = 3a_{\text{rad,viejo}}$?
    \item \textbf{Radio del tambor:} $R$ (constante, no cambia)
    \item \textbf{Periodo inicial:} $T_1$ (desconocido)
    \item \textbf{Periodo final:} $T_2 = ?$ (a determinar)
\end{itemize}
\end{datosbox}

\section{Marco Teórico}

\subsection{Aceleración Centrípeta en Movimiento Circular Uniforme}

Para un objeto que se mueve en una trayectoria circular de radio $R$ con rapidez constante, la aceleración centrípeta está dirigida hacia el centro del círculo y tiene magnitud:

\begin{equation*}
    a_{\text{rad}} = \frac{v^2}{R}
\end{equation*}

\subsection{Relación entre Velocidad y Periodo}

En un movimiento circular uniforme, la partícula recorre una distancia igual a la circunferencia $2\pi R$ en un tiempo igual al periodo $T$. Por lo tanto:

\begin{equation*}
    v = \frac{2\pi R}{T}
\end{equation*}

\subsection{Aceleración Centrípeta en Términos del Periodo}

Sustituyendo la ecuación de velocidad en la ecuación de aceleración:

\begin{equation*}
    a_{\text{rad}} = \frac{v^2}{R} = \frac{(2\pi R/T)^2}{R} = \frac{4\pi^2 R^2}{T^2 R} = \frac{4\pi^2 R}{T^2}
\end{equation*}

Esta es la ecuación clave: la aceleración centrípeta es \textbf{inversamente proporcional al cuadrado del periodo}.

\section{Análisis Conceptual}

De la ecuación anterior observamos que:

\begin{equation*}
    a_{\text{rad}} \propto \frac{1}{T^2}
\end{equation*}

Esto significa:
\begin{itemize}
    \item Si el periodo \textbf{disminuye}, la aceleración centrípeta \textbf{aumenta}
    \item Si el periodo \textbf{aumenta}, la aceleración centrípeta \textbf{disminuye}
    \item La relación es cuadrática: cambios pequeños en $T$ producen cambios grandes en $a_{\text{rad}}$
\end{itemize}

\section{Desarrollo de la Solución}

\begin{solucionbox}

\subsection*{Paso 1: Establecer la relación entre aceleraciones y periodos}

Sean:
\begin{itemize}
    \item $T_1$ = periodo inicial de rotación
    \item $a_1$ = aceleración centrípeta inicial
    \item $T_2$ = periodo final de rotación (desconocido)
    \item $a_2$ = aceleración centrípeta final
\end{itemize}

Usando la ecuación de aceleración en términos del periodo:
\begin{align*}
    a_1 &= \frac{4\pi^2 R}{T_1^2} \\
    a_2 &= \frac{4\pi^2 R}{T_2^2}
\end{align*}

\subsection*{Paso 2: Aplicar la condición del problema}

El problema pide que $a_2 = 3a_1$ (triplicar la aceleración). Sustituyendo:

\begin{equation*}
    3a_1 = a_2
\end{equation*}

\begin{equation*}
    3 \cdot \frac{4\pi^2 R}{T_1^2} = \frac{4\pi^2 R}{T_2^2}
\end{equation*}

\subsection*{Paso 3: Simplificar y despejar $T_2$}

Podemos cancelar $4\pi^2 R$ en ambos lados:

\begin{equation*}
    \frac{3}{T_1^2} = \frac{1}{T_2^2}
\end{equation*}

Despejando $T_2^2$:

\begin{equation*}
    T_2^2 = \frac{T_1^2}{3}
\end{equation*}

Tomando raíz cuadrada:

\begin{equation*}
    T_2 = \frac{T_1}{\sqrt{3}}
\end{equation*}

O bien, expresándolo de otra forma:

\begin{equation*}
    T_2 = \frac{T_1}{\sqrt{3}} \approx 0.577 \, T_1
\end{equation*}

\subsection*{Interpretación}

Para triplicar la aceleración centrípeta, el periodo de rotación debe \textbf{dividirse entre} $\sqrt{3}$ (o multiplicarse por $1/\sqrt{3}$).

Esto significa que la lavadora debe girar \textbf{más rápido}, completando cada ciclo en aproximadamente el 57.7\% del tiempo original.

\subsection*{Verificación}

Verifiquemos que esta respuesta es correcta:

Si $T_2 = T_1/\sqrt{3}$, entonces:

\begin{align*}
    \frac{a_2}{a_1} &= \frac{4\pi^2 R/T_2^2}{4\pi^2 R/T_1^2} \\
    &= \frac{T_1^2}{T_2^2} \\
    &= \frac{T_1^2}{(T_1/\sqrt{3})^2} \\
    &= \frac{T_1^2}{T_1^2/3} \\
    &= 3 \quad \checkmark
\end{align*}

La verificación confirma que $a_2 = 3a_1$.

\end{solucionbox}

\section{Resultado Final}

\begin{resultadobox}

\textbf{Respuesta que daría a la jefa:}

\vspace{0.5cm}

\textit{"Para triplicar la aceleración centrípeta, debemos reducir el periodo de rotación dividiendo el periodo actual entre $\sqrt{3}$, o equivalentemente, multiplicando la frecuencia (revoluciones por minuto) por $\sqrt{3}$."}

\vspace{0.5cm}

\textbf{Expresión matemática:}

\begin{equation*}
    \boxed{T_{\text{nuevo}} = \frac{T_{\text{actual}}}{\sqrt{3}} \approx 0.577 \, T_{\text{actual}}}
\end{equation*}

\textbf{En términos de frecuencia:}

Si trabajamos con RPM (revoluciones por minuto), donde $f = 1/T$:

\begin{equation*}
    \boxed{f_{\text{nueva}} = \sqrt{3} \cdot f_{\text{actual}} \approx 1.732 \, f_{\text{actual}}}
\end{equation*}

\textbf{Ejemplo práctico:}

Si la lavadora gira actualmente a 600 RPM:
\begin{itemize}
    \item Nueva velocidad = $600 \times \sqrt{3} \approx 1039$ RPM
    \item Esto triplicaría la aceleración centrípeta
\end{itemize}

\end{resultadobox}

\section{Relación General}

Es útil conocer la relación general entre el factor de cambio en la aceleración y el factor de cambio en el periodo:

\begin{tcolorbox}[colback=yellow!10!white,colframe=orange!75!black,title=Fórmula General]

Si queremos que la aceleración centrípeta cambie por un factor $k$:

\begin{equation*}
    a_{\text{nueva}} = k \cdot a_{\text{vieja}}
\end{equation*}

Entonces el periodo debe cambiar por un factor:

\begin{equation*}
    T_{\text{nuevo}} = \frac{T_{\text{viejo}}}{\sqrt{k}}
\end{equation*}

Y la frecuencia por:

\begin{equation*}
    f_{\text{nueva}} = \sqrt{k} \cdot f_{\text{vieja}}
\end{equation*}

\end{tcolorbox}

\section{Casos Particulares}

\begin{center}
\begin{tabular}{|c|c|c|c|}
\hline
\textbf{Factor de } & \textbf{Factor de} & \textbf{Factor de} & \textbf{Ejemplo} \\
\textbf{aceleración ($k$)} & \textbf{periodo} & \textbf{frecuencia} & \textbf{(de 600 RPM)} \\
\hline
2 (duplicar) & $1/\sqrt{2} \approx 0.707$ & $\sqrt{2} \approx 1.414$ & 849 RPM \\
3 (triplicar) & $1/\sqrt{3} \approx 0.577$ & $\sqrt{3} \approx 1.732$ & 1039 RPM \\
4 (cuadruplicar) & $1/2 = 0.500$ & $2$ & 1200 RPM \\
9 (9 veces) & $1/3 \approx 0.333$ & $3$ & 1800 RPM \\
\hline
\end{tabular}
\end{center}

\section{Consideraciones Prácticas}

\subsection{Límites físicos}

En una lavadora real, no se puede aumentar indefinidamente la velocidad de rotación porque:

\begin{enumerate}
    \item \textbf{Resistencia mecánica:} El tambor y los componentes pueden no soportar fuerzas muy altas
    \item \textbf{Ruido y vibración:} A velocidades altas, la vibración se vuelve excesiva
    \item \textbf{Seguridad:} La ropa y el agua ejercen fuerzas mayores que podrían dañar la máquina
    \item \textbf{Eficiencia energética:} Mayor velocidad requiere más potencia del motor
\end{enumerate}

\subsection{Aplicación en ciclos de centrifugado}

Las lavadoras modernas típicamente tienen:
\begin{itemize}
    \item Ciclo de lavado: 50-100 RPM (baja aceleración)
    \item Ciclo de centrifugado: 800-1600 RPM (alta aceleración para extraer agua)
\end{itemize}

La diferencia de velocidad entre estos ciclos produce diferencias mucho mayores en la aceleración centrípeta.

\section{Conceptos Clave}

\begin{enumerate}
    \item La aceleración centrípeta es \textbf{inversamente proporcional al cuadrado del periodo}: $a_{\text{rad}} \propto 1/T^2$

    \item Para multiplicar la aceleración por un factor $k$, el periodo debe dividirse entre $\sqrt{k}$

    \item La relación cuadrática implica que pequeños cambios en el periodo producen cambios significativos en la aceleración

    \item En términos de frecuencia, para multiplicar la aceleración por $k$, la frecuencia debe multiplicarse por $\sqrt{k}$

    \item Este principio se aplica a cualquier movimiento circular uniforme: centrifugadoras, ruedas de la fortuna, satélites en órbita, etc.
\end{enumerate}

\end{document}
